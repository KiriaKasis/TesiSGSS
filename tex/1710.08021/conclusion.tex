

\chapter*{Conclusion}
\addcontentsline{toc}{chapter}{Conclusion}\markboth{Conclusion}{}



		\fancyhead{} % cancella tutti i campi
			\fancyhead[LE]{\scshape \leftmark}
			\fancyhead[RO]{\scshape \rightmark}
			\fancyfoot[LE,RO]{\thepage}
			\fancyfoot[LO,CE]{ }
			\fancyfoot[CO,RE]{ }
		 		\renewcommand{\headrulewidth}{0.4pt}
			\renewcommand{\footrulewidth}{0.4pt}
		\pagestyle{fancy}
		\renewcommand{\sectionmark}[1]{\markright{\thesection.\ #1}}
		\renewcommand{\chaptermark}[1]{\markboth{\chaptername\ \thechapter.\ #1}{}}


In this work we studied the one-loop renormalization properties of a \ym{} theory whose quadratic term in the Lagrangian contains, besides the conventional kinetic term, also an extra quadratic operator weighted with a coefficient with negative mass dimension. We also considered another dimension-six interaction cubic in the field strength weighted with a dimensionless parameter, and then we extended the theory to make it supersymmetric.


We first gave a rapid review of the path integral formalism, presenting the method of background field quantization to compute one-loop effective actions. In order to evaluate of the divergence in the effective action, the heat kernel technique was also introduced.


We then analysed the basic aspects of the theory, in particular considering its renormalization properties. This theory turned out to be renormalizable, since the extra derivatives present in the Lagrangian density induce another factor of squared momentum in the propagator, improving the ultraviolet properties of the theory. 
We computed the one-loop effective action, thus deriving the necessary renormalization of the gauge coupling and the $\beta$ function. We showed that the $\beta$ function is a quadratic function of the parameter weighting the cubic contribution, and can acquire either sign, or even vanish. Our computation corrects a mistake of an earlier work in the literature, confirming other computations with a  diagrammatic approach.
We also considered the coupling with this kind of higher derivative matter fields, including their contribution in the $\beta$ function, confirming some literature results. We observed that such a model solves the hierarchy problem because the mass of the scalar does not get quadratic corrections on the cut-off.
The alternative formulation of higher derivative theories in terms of usual-derivative ones was also discussed, showing that one ghost field appears. 


The supersymmetric generalization of the theory presents some technical difficulties, that can be partially overcame in formulating it in six spacetime dimensions and then performing a trivial dimensional reduction.  Six dimensional spacetime seems to be more a more natural framework in which to formulate this kind of higher derivative theories, as the coupling is dimensionless and the component field expression of the Lagrangian is much simpler than the four dimensional ones. This allowed us to get the $N=1$ and $N=2$ supersymmetric extension of the original theory. We also managed to gain enough information to reconstruct, at least at the linearised level, the $N=4$ supersymmetric model. The supersymmetric case is much different from those previously considered. To begin with, supersymmetry prohibits the presence of the contribution cubic in the field strength; then, the matter field contribution is such that the theory has a positive $\beta$ function, growing with the number of supersymmetries.



To conclude with, higher derivative theories are interesting from many points of view and they arise in many different contexts. Even though they suffer of some inconsistencies and ambiguities, and it is unlikely that the theories we studied actually are a fundamental theory, they may serve as a toy model for them and might as well describe the low energy behaviour of some fundamental theory. Surely this is enough to motivate further studies in this area, and in particularly much work is to be done in understanding the connection between supersymmetry and higher-derivative theories. For example, it would be interesting to study in greater detail how the supermultiplet structure relates with the higher derivative operators; also, the relation between supersymmetry breaking mechanisms and the presence of extra derivatives should be understood with more attention. 










