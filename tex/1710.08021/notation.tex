

	\fancyfoot[CE,CO]{\thepage}
	\fancyfoot[LO,RO]{ }
	\fancyfoot[LE,RE]{ }
\chapter{Notation}

%%In this work we study relativistic quantum field theory in the Minkovsky spacetime metric $\tilde \eta^{\mu\nu} = \text{diag}(-,+,+,+)$.


%%In this work we consider a field theory in Euclidean spacetime. 
For definiteness of the functional integral we consider the Wick-rotated space with metric therefore
\(
\delta^{\mu\nu}
= \1_4
=
\text{diag}(++++)
\)
.
However, as we mentioned in the introduction, the Wick rotation is ill-defines for the higher derivative theories we study in this work. Such problems are not of interest for this thesis and this operation might be thought as a formal operation done to the theory defined in Minkovsky spacetime with metric \(\eta^{\mu\nu}=\text{diag}(-+++)\) to slightly simplify the notation. The formal differences between the two treatments is an overall sign in the Lagrangian density and factors of $i$ in the path integral formulation.
Einstein's index convention applies otherwise stated.
Our notation for spinors and supersymmetry is basically that of \cite{WB}.

Considering gauge theories, the symmetry group will generically be indicated with $\GroupName{G}$ and assumed to be a compact Lie group. Its structure constants are
\begin{equation}
	[T^a,T^b] = f^{amb} T^m,
\end{equation}
where $T^a$ are a set of generators of any representation. 
The generators \(t^a\) of the fundamental representation are normalized so so that \begin{equation}\label{notation-tr-generatord-fundam}
\tr t^a t^b = - \delta^{ab}/2 ;
\end{equation}  those of the adjoint representation have components
\(
(T_\ad^m)^{ab} = f^{amb}
\).
The normalization of any representation $R$ are normalized such that
\begin{equation}
\tr T_R^a  T_R^b = - C_R \delta^{ab} 
\end{equation}
defining the Casimir invariants $C_R$.
In a given representation $R$ we will therefore have
\begin{equation}\label{notation-trFmunuFmunu}
\tr F_{\mu\nu} F_{\mu\nu}
	=
- C_R F_{\mu\nu}^a F_{\mu\nu}^a.
\end{equation}
The Casimir element for the adjoint representation $C_2$ can be expressed in terms of the structure constants as
\begin{equation}
f^{mna} f^{mnb} =  C_2 \delta^{ab}.
\end{equation}
We assume Einstein's convention to hold also for gauge indices.


The gauge field $A_\mu$ induces in the minimal coupling prescription the covariant derivative
\begin{equation}
\covD_\mu = \partial_\mu + A_\mu
\end{equation}
and the field strength tensor reads
\begin{equation}
F_{\mu\nu} = [\covD_\mu, \covD_\nu]
= \partial_\mu A_\nu - \partial_\nu A_\mu + [A_\mu , A_\nu ].
\end{equation}
The field strength tensor satisfy the Bianchi's identity
\begin{equation}
\covD_{[\mu} F_{\nu\rho]} = 0
\end{equation}
that can be used to prove the relation
\begin{equation}\label{CovDF-dentity}
2 \left( \covD_\mu F_{\mu\nu} \right)^2 
	=
\left( \covD_\mu F_{\nu\rho} \right)^2 
+ F_{\mu\nu} \left[ F_{ \nu \rho } , F_{ \rho \mu } \right].
\end{equation}




We will also employ natural units in which the speed of light $c$ and Planck's constant $\hbar$ are taken to 1. All quantities will then be measured in units of mass. The only exception is the first chapter, where Planck's constant is used as a formal parameter to study loop-expansions.



The trace over gauge indices is $\tr$; $\tilde \tr$ is that on spinor indices and $\Tr$ indicates the trace over all indices after the symbol.





\section[Spinors in \texorpdfstring{${d=4}$}{{{d=4}}} ]{Spinors in \texorpdfstring{$\boldsymbol{d=4}$}{{{d=4}}} }


We use here two-component Weyl spinors $\psi^\alpha$.
The conjugate spinor is \( \bar \psi^{\dot \alpha} := (\psi^\alpha)^* \) where \(*\) denotes complex conjugation.

The antisymmetric tensors are
\[ \varepsilon^{21} = \varepsilon_{12} = +1 ,
\hspace{5em}
%%\text{and}
%%\hspace{5em}
\varepsilon^{\dot 2 \dot 1} = \varepsilon_{\dot 1 \dot 2} = +1
\]
which moreover they satisfy
\[
\varepsilon_{\alpha\beta} \varepsilon^{\beta\gamma} = \delta^\gamma_\alpha,
\hspace{5em}
%%\text{and}
%%\hspace{5em}
\varepsilon_{\dot\alpha\dot\beta} \varepsilon^{\dot\beta\dot\gamma} = \delta^\gamma_\alpha
.
\]
Sigma matrices read
\begin{equation}\label{notation-sigma-matrices}
	\sigma^\mu_{\alpha \dot\beta}
		=
	( \mathbbm{1} , \sigma^i )_{\alpha \dot \beta},
\hspace{4em}
%%\text{and}
%%\hspace{4em}
	\bar\sigma^{\mu\; \dot \alpha \beta}
		=
	( \mathbbm{1} ,  -\sigma^i )^{\dot \alpha \beta}
		=
	\varepsilon^{\dot \alpha \dot \beta}
	\varepsilon^{\beta \alpha}
	\sigma^\mu_{\alpha \dot \beta}
\end{equation}
Lorentz generators of the spinor representation are
\begin{equation}\label{notation-gener-lorentz-spinor}
\begin{split}
	{\sigma^{\mu \nu \; }}\indices{ _{ \alpha} ^{{\beta}} }
& =
	 {\sigma^{[\mu}\bar\sigma^{\nu]}}%
		\indices{ _\alpha ^\beta }
=
	\frac{1}{2} \left(
				\sigma^\mu \bar \sigma^\nu
				-
				\sigma^\nu \bar \sigma^\mu 
			\right)\indices{ _\alpha ^\beta },
\\
	{{\bar{\sigma}}}\indices{ ^{\mu \nu \; } ^{\dot \alpha} _{\dot{\beta}} }
& =
	 {\bar\sigma^{[\mu}\sigma^{\nu]}}
		\indices{ ^{\dot{\alpha}} _{\dot{\beta}}}
=
	\frac{1}{2} \left(
				\bar \sigma^\mu \sigma^\nu
				-
				\bar \sigma^\nu \sigma^\mu 
			\right)\indices{ ^{\dot{\alpha}} _{\dot{\beta}}}.
\end{split}
\end{equation}


The conventions for contracting spinor indices are
\begin{equation}
\psi \chi = \psi^\alpha \chi_\alpha,
\hspace{5em}
\bar \psi \bar \chi = \bar \psi_{\dot \alpha} \bar \chi^{\dot \alpha}.
\end{equation}


\subsection{Identities}
For a comprehensive treatment of the two-component spinor notation we found \cite{Dreiner:2008tw} very useful. In our notation, we recall that the following identities hold. 
\begin{align}
\label{identity-1}
\str \sigma^\mu \bar \sigma^\nu
& 	= \str \bar \sigma^\mu \sigma^\nu 
	=
		- 2 \eta^{\mu\nu}
\\
\label{identity-2}
(\sigma^{\mu\nu})\indices{_\alpha^\beta}
&	=
		\varepsilon_{\alpha\tau} \varepsilon^{\beta\kappa} 	
 			( \sigma^{\mu\nu} )\indices{_\kappa^\tau}
\\ 
\label{identity-3}
\sigma^{\mu}_{\alpha\dot\alpha} \bar\sigma^{\dot \beta \beta}_\mu 
&	=
	-	2 \delta_\alpha^\beta\delta^{\dot\beta}_{\dot\alpha}
\\
\label{identity-4}
\sigma^{\mu\; \alpha\dot\alpha} \sigma_\mu^{ \beta \dot \beta}
&	=
	-	2\varepsilon_{\alpha\beta}\varepsilon_{\dot\alpha\dot\beta}
\\
\label{identity-5}
\bar \sigma^{\mu\;\dot \alpha\alpha} \bar\sigma^{\dot \beta \beta}_\mu 
&	=
	-	2\varepsilon^{\alpha\beta}\varepsilon^{\dot\alpha\dot\beta}
\\
\label{identity-6}
[\sigma^{(\mu} \bar\sigma^{\nu)}]\indices{_\alpha^\beta}
&	=
	-	\eta^{\mu\nu} \delta^\beta_\alpha
\\
\label{identity-7}
[\bar \sigma^{(\mu} \sigma^{\nu)}]
	\indices{^{\dot \alpha}_{\dot \beta}}
&	=
	-	\eta^{\mu\nu} \delta^{\dot\alpha}_{\dot\beta}
\\
\label{identity-8}
\str \sigma^{\mu\nu} \sigma^{\tau\kappa}
&	= 
		-4 \left( \eta^{\mu\tau} \eta^{\nu\kappa}
			-	\eta^{\mu\kappa} \eta^{\tau\nu}
			- i \varepsilon^{\mu\nu\tau\kappa}
			\right)
\\
\label{identity-9}
\str \bar \sigma^{\mu\nu} \bar \sigma^{\tau\kappa}
&	= 
		-4 \left( \eta^{\mu\tau} \eta^{\nu\kappa}
			-	\eta^{\mu\kappa} \eta^{\tau\nu}
			+ i \varepsilon^{\mu\nu\tau\kappa}
			\right)
\end{align}
In particular that the last two identities imply
\begin{equation}\label{identity-spinor_trace-double-sigmamn}
\Tr{F_{\mu\nu} F_{\tau\kappa}}
		\sigma\indices{^{\mu \nu}} 
		\sigma\indices{^{\tau \kappa}} 
	=
\tr{F_{\mu\nu} F_{\tau\kappa}}
\str		\sigma\indices{^{\mu \nu}} 
		\sigma\indices{^{\tau \kappa}} 
	=
	- 4 \tr F_{\mu\nu} F^{\mu\nu}
\end{equation}
and
\begin{equation}\label{identity-spinor_trace-double-sigmamn-V}
\Tr{F_{\mu\nu} F_{\tau\kappa}}
		{\bar \sigma}\indices{^{\mu \nu}} 
		{\bar \sigma}\indices{^{\tau \kappa} }
		=
\tr{F_{\mu\nu} F_{\tau\kappa}}
\str	{\bar \sigma}\indices{^{\mu \nu} } 
		{\bar \sigma}\indices{^{\tau \kappa} }
	=
	- 4 \tr F_{\mu\nu} F^{\mu\nu}.
\end{equation}



\section[Supersymmetry in \texorpdfstring{${d=4}$}{{{d=4}}} ]{Supersymmetry in \texorpdfstring{$\boldsymbol{d=4}$}{{{d=4}}} }

The Poincar\'{e} algebra, consisting of generators of translations $P_\mu$ and Lorentz transformations $M_{\mu\nu}$  reads
\begin{align}
\label{susy1}
[M_{\mu\nu}, M_{\rho \sigma}] &= 
- i \eta_{\mu\rho} M_{\nu\sigma}
- i \eta_{\nu\sigma} M_{\mu\rho}
+ i \eta_{\mu\sigma} M_{\nu\rho}
+ i \eta_{\nu\rho} M_{\mu\sigma} 
\\
[M_{\mu\nu} , P_\rho]
&  =
-i \eta_{\rho\mu}  P_\nu + i \eta_{\rho\nu} P_\mu.
\end{align} 
The supersymmetric extension, in the cases of vanishing central charge, is
\begin{align}
\label{pedissequo0000}
[M_{\mu\nu}, Q^i_\alpha] 
& = i {\sigma}\indices{_{\mu\nu \, \alpha}^\beta} Q^i_\beta
\\
[M_{\mu\nu}, \bar{Q}_{i}^{\dot \alpha}] 
&  = i {\bar{\sigma}}%
\indices{_{\mu\nu \,} ^{\dot{\alpha}}_{\dot{\beta}}} \bar Q_{i}^{\dot{\beta}}
\\
\label{pedissequo}
\{Q^i_\alpha , \bar{Q}_{j \dot \beta} \} & = 2 \sigma^\mu_{\alpha \dot{\beta}}
P_\mu \delta^{i}_j,
\end{align}
where $Q^i_\alpha$ are the supercharges, with $i$ being the $R$-symmetry index and $\alpha$ the spinor index; $(Q^i_\alpha)^\dagger = \bar Q_{i\dot\alpha}$.

The covariant derivatives read
\begin{equation}
\covDs_{\alpha i} = \frac{\partial}{\partial \theta^{\alpha i}}
+ i \sigma^\mu_{\alpha\dot\alpha} \bar\theta^{\dot \alpha}_i \frac{\partial}{\partial x^\mu},
\hspace{3em}
\bar \covDs_{\dot \alpha}^i = - \frac{\partial}{\partial \bar{\theta}^{\dot \alpha}_i}
- i \theta^{\alpha i } \sigma^\mu_{\alpha \dot\alpha} \frac{\partial}{\partial x^\mu};
\end{equation}
they commute with the generators $Q$, $\bar Q$ and  satisfy the algebra
\begin{equation}
\{
\covDs_{\alpha i},
\bar \covDs_{\dot \alpha} ^ i\}
=
 - 2 \delta_i^j \sigma^\mu_{\alpha \dot \alpha} \, \partial_\mu.
\end{equation}


The $N=1$ relevant supermultiplets are the chiral and the gauge one. For the latter, the degrees of freedom can be collected into the gauge superfield
\begin{equation}\label{wz-V}
V(x,\theta,\bar\theta) = - \theta \sigma^\mu \bar \theta A_\mu(x) + i (\theta\theta\bar \theta\bar\lambda(x) - \bar \theta\bar \theta\theta\lambda(x))
+ \frac{1}{2} \theta \theta \bar \theta \bar \theta D(x)
\end{equation}
that can be used to define the superfield strength
\begin{equation}
W_\alpha = - \frac{1}{4} \bar \covDs \bar \covDs \left( e^{-V} \covDs_\alpha e^V \right).
\end{equation}
The action can then be written as
\begin{align}\label{N1sym-4d}
S^{N=1}_{\SYM}
& = \frac{1}{2}
\int \dd{^4x} \dd{^2\theta}\tr [ W W + \text{h.c.}]
=
\int \dd{^4x}
\Lagr^{N=1}_\SYM
\\
\Lagr^{N=1}_\SYM
&=
\tr
\left[
\frac{1}{2} F_{\mu\nu} F^{\mu\nu} + 2 i \bar \lambda \bar \sigma^\mu \covD_\mu  \lambda -  D^2
\right]
\end{align}
with $F_{\mu\nu} = [\covD_\mu, \covD_\nu]$, $\covD_\mu $ being in the adjoint representation.
We also recall that the Wick-rotated Lagrangian, with a suitable rescaling of the fields, can be written as
\begin{align}
\label{lagr-N1sym-4d}
\Lagr^{N=1}_\SYM
&=
-
\frac{1}{g^2}
\tr
\left[
\frac{1}{2} F_{\mu\nu} F^{\mu\nu} + 2 i \bar \lambda \bar \sigma^\mu \covD_\mu  \lambda -  D^2
\right]
\end{align}



In the Abelian case the superfield strength reads
\begin{equation}
W_\alpha = - \frac{1}{4} \bar \covDs \bar \covDs \covDs_\alpha V 
\end{equation}
and the Lagrangian for super-Maxwell theory is
\begin{align}
\label{lagr-n=1-smaxwell}
 \Lagr^{N=1}_{\text{sM}} &=
-\frac{1}{4} F_{\mu\nu} F^{\mu\nu} - i \bar \lambda \bar \sigma^\mu \partial_\mu  \lambda + \frac{1}{2} D^2
\end{align}
with $F_{\mu\nu} = \partial_\mu A_\nu - \partial_\nu A_\mu.$

The gauge theory with extended supersymmetry can be realised by adding to \eqref{N1sym-4d} one or three $N=1$  chiral multiplets in the adjoint representations of the gauge group, with the correct coefficients in order to have the required $R$-symmetry between spinors and scalars.



