
\chapter{Higher-derivative gauge theory}

This Chapter is organised as follows. In Section 2.1 we define the higher-derivative gauge theory that we want to consider and study some of its properties; then, in Section~2.2 we explicitly compute the one-loop $\beta$ function for the \ym{} coupling. In Section~2.3 we consider also the presence of matter fields, allowing for a higher-derivative Lagrangian for them as well.






\section{Pure-gauge higher derivative theory}

Let us consider a pure-gauge theory invariant under the action of a semisimple gauge group $\GroupName{G}$; for the notation and the definition of the mathematical structures, see Appendix~A. The theory that we are considering describes a set of gauge fields $A_\mu^a$, where $a=1,\ldots,\dim\mathrm{Lie}\, \GroupName{G}$ is the `color' index.


The higher-derivative extension of the \ym{} Euclidean Lagrangian density that we are going to consider is
\begin{equation}\label{lagr-hdym}
\Lagr^\HD_{\YM}
	=
- \frac{1}{2 g^2} \tr  F_{\mu\nu} F_{\mu\nu} %\right)
	- \frac{1}{ m^2 g^2 }\tr
		\left[
			 \left( \covD_\mu F_{\mu\nu} \right)^2
			+ \gamma   \: F_{\mu\nu}
				\left[ F_{\mu\lambda},
				F_{\nu\lambda} \right] 
			\right],
\end{equation}
where $\covD_\mu$ is the covariant derivative of the gauge field $A_\mu$ and, as usual,
\(
F_{\mu\nu} 
	=
[\covD_\mu, \covD_\nu] 
\); $ \gamma$ is a real dimensionless parameter, whereas $m$ has dimension of mass, as the name suggests. 
In components, the Lagrangian density \eqref{lagr-hdym} reads
\begin{equation}\label{lagr-hdym-comp}
\Lagr^\HD_{\YM}
	=
\frac{1}{4 g^2}  F^a_{\mu\nu} F^a_{\mu\nu}
	+ \frac{1}{ 2 m^2 g^2 }
		\left[
			 \left( \covD_\mu F^a_{\mu\nu} \right)^2
			+ \gamma  f^{abc}  F^a_{\mu\nu}
				F^b_{\mu\lambda}
				F^c_{\nu\lambda}  
			\right],
\end{equation}




In \eqref{lagr-hdym} we added to the \ym{} Lagrangian density the most general contribution that can be written from terms of mass dimension six  preserving gauge invariance.
Such new terms enjoy interesting properties. This can be understood observing that $(\covD_\mu F_{\mu\nu})^2$ is a kinetic-like term, because it contains contributions that are quadratic in the fields. Indeed, this contribution is not studied in perturbation theory: It can be resummed in an additional term modifying the propagator, that, after a proper gauge fixing, for high momenta scales with $p^{-4}$. Such UV behaviour results in an improved scaling of loop integrals that is a first hint to the fact that this kind of theories have, in general, better renormalization properties than usual-derivative theories.
%
As we are going to discuss, despite having a parameter with negative mass dimension (\ie $1/m^2$), the scaling of the propagator at high energies is such that the theory turns out to be renormalizable.



%%This closely resembles the Pauli-Villars renormalization scheme; indeed the propagator can be decomposed in the sum of two conventional terms:
%%
%%
%%Schematically, with some properly chosen gauge fixing term, 
%%
%%This fact suggests a way to the numbers of degrees of freedom, that will be investigated.
%%





\subsection{Degrees of freedom}

In this subsection we will comment on the structure of degrees of freedom of the theory \eqref{lagr-hdym}. Since the kinetic term is unconventional, we first specify what we actually mean when we consider the degrees of freedom. We will distinguish between `off-shell' and `on-shell'  degrees of freedom. The former is the number of independent real fields that are used to define the Lagrangian density, regardless of the dynamics; the latter is half the number of initial real data necessary to specify a solution of the equation of motions. These definitions are motivated by the comparison with ordinary-derivative theories.




The theory describes a vector field with gauge invariance, hence it has $ 4 - 1= 3 $ off-shell degrees of freedom for each color index. 

Taking the dynamics into consideration, the Lagrangian density contains terms with up to four derivatives, so that the equations of motions derived from it are fourth-order partial differential equations.  The component $A^0$  of the gauge field is still non-dynamical, since, as in the usual \ym{} case, $F^{00}=0$.  The number of on-shell degrees of freedom is therefore $ 2 \cdot 3 - 1 = 5 $ for each gauge index.

For comparison, recall that the conventional Yang-Mills action $\sim  \tr F_{\mu\nu}F_{\mu\nu}$ propagates $3 - 1=2$ on-shell degrees of freedom for color index. The additional number of degrees of freedom in $\Lagr_\YM^\HD$ can be made manifest rewriting the Lagrangian density in a suitable fashion; as we are going to show, the degrees of freedom can be realized through a conventional Yang-Mills field interacting with a massive vector field conveying the $3$ missing degrees of freedom.

For simplicity, and since we are interested only in the structure of the degrees of freedom, in our discussion we will consider only the extended kinetic term
\begin{equation}\label{lagr-hdym-kinetc}
\left. \Lagr^\HD_{\YM} \right|_{\text{kin}}
	=
-\frac{1}{2 g^2} \tr  F_{\mu\nu} F_{\mu\nu} %\right)
	- \frac{1}{ m^2 g^2 }
			\tr \left( \covD_\mu F_{\mu\nu} \right)^2
\end{equation}
Let us introduce an auxiliary field $A'_\mu$  transforming in the adjoint representation of the gauge group. Let $\covD_\mu$ and $F_{\mu\nu}$ denote the covariant derivative and the field strength tensor of the field $A_\mu$. The Lagrangian density
\begin{equation}
\Lagr =
	-\tr \left[
			\frac{1}{2} F_{\mu\nu} F_{\mu\nu} 
			- m^2 A'_\mu A'_\mu
			- 2 F_{\mu\nu} \covD_\mu A'_\nu
		\right]
\end{equation}
gives back the original \eqref{lagr-hdym-kinetc} on equations of motion of the auxiliary field 
\begin{equation}
  m^2 A'_\mu = \covD_\mu F_{\mu\nu} .
\end{equation} Shifting $ A_\mu \rightarrow A_\mu + A'_\mu $, the auxiliary field becomes dynamical and the mixing $\sim \tr F \covD A'$ exactly cancels:
\begin{equation}\label{lagr-HD-ym-splitted-dof}
\begin{split}
\Lagr
& =
	-\tr \bigg[
			\frac{1}{2} \left(
				F_{\mu\nu} + \covD_\mu A'_\nu - \covD_\nu A'_\mu + [A'_\mu,A'_\nu]
				\right)^2 
			- m^2 A'_\mu A'_\mu
\\
&
\hspace{3em}
			- 2 \left(
				F_{\mu\nu} + \covD_\mu A'_\nu - \covD_\nu A'_\mu + [A'_\mu,A'_\nu]
				\right)
				(\covD_\mu  A'_\nu + [A'_\mu,A'_\nu]) 
		\bigg]
\\
& 
	 =
	-\tr \bigg[
		\frac{1}{2}	F_{\mu\nu}F_{\mu\nu} 
		- \frac{1}{2} \left( \covD_\mu A'_\nu - \covD_\nu A'_\mu \right)^2 
		- m^2 A'_\mu A'_\mu 
		+ \text{interactions} 
		\bigg]
\end{split}
\end{equation} 
where in the last equality we focused on the kinetic term.




Notice that, in this decomposition, both the kinetic and the mass term for the massive fields have the wrong sign. Already at the classical level this is known to be a problem because the equations of motion allow for modes growing exponentially at infinity, that upon canonical quantization result in negative norm states in the Hilbert state space.


These aspects of the theory could lead to a violation of causality; this problem can be solved by imposing future boundary conditions, or modifying the contour of integration in the definition of the propagator so that the particles associated to negative-norm decay in every physical process. It can also be shown that the theory with $\gamma = 0$, at least in the non-abelian case or at low energies (smaller than the scale dictated by $m$) is perturbatively unitary. Anyway, dealing with  these problems goes much beyond the aim of the present work, since we are interested in the one-loop renormalization properties, therefore we will not investigate further these issues.










\subsection{Renormalizability}

We are now going to discuss the properties of renormalizability of the theory \eqref{lagr-hdym}, showing that, as anticipated, it turns out to be renormalizable despite having a parameter with negative mass dimension. We will do so by performing an explicit power counting of the divergences in Feynman diagrams and without considering the background field approach. Since this approach is somehow tangential to the main topics of the present work, we will outline the main steps in the derivation without all details and explicit expressions. Notice that we are not using the background field technique outlined in the previous Chapter.

We choose the gauge fixing to be
\begin{equation}
G[A] = \partial_\mu A_\mu,
\end{equation}
integrated with weight 
\begin{equation}
 H = - \frac{1}{m^2 g^2}\partial_\mu \partial_\mu + \frac{1}{g^2},
\end{equation}
so that the Lagrangian density picks up the contribution
\begin{equation}\label{hdym-renorm-gauge-fixing}
\Delta_G \Lagr
	=
	-
\frac{1}{2}
\tr \left[ 
	\partial_\mu A_\mu 
	\left( 1 - \frac{1}{m^2}\partial_\nu \partial_\nu \right)
	\partial_\rho A_\rho
\right]
\end{equation}
Notice that the first term, \ie the `$1$'  in the expression of $H$, is the same Lorenz gauge-fixing contribution in the case of the \ym{} Lagrangian cancelling the contribution $\partial_\mu A_\nu \, \partial_\nu A_\mu$ from the expansion of $F_{\mu\nu}F_{\mu\nu}$; the second cancels the analogous term originating from $(\partial_\mu F_{\mu\nu})^2 \sim \partial_\mu \partial_\mu A_\nu \, \partial_\rho  \partial_\rho A_\nu  $.

The factor $\det H$ is just a numerical contribution independent of the fields, therefore it can be reabsorbed in the definition of the integration measure.
This choice of $G$ gives the Jacobian contribution \eqref{M-lorenz-gauge}
\begin{equation}
M = - \partial_\mu  \covD_\mu 
.
\end{equation}
%%because 
%%\(
%%\delta_\omega G 
%%	=
%%\partial_\mu \delta_\omega A_\mu 
%%	=
%%\partial_\mu \covD_\mu \omega.
%%\)
In order to evaluate the determinant in a diagrammatic framework, we represent it by introducing (anticommuting) Lorentz-scalar ghost fields  $c$ and  $\bar c$ via
\begin{equation}
\det \hat M = \int \DD{c} \DD{\bar c} 
	\exp \left[ {\int \dd{^4x}  \bar c \ M \ c } \right]
\end{equation}
that effectively adds a new contribution to the Lagrangian density
\begin{equation}
\Delta_c \Lagr
	=
\tr \left[ - \bar c \ \partial_\mu \covD_\mu c \right].
\end{equation}

Considering the kinetic term \eqref{lagr-hdym-kinetc} with the gauge fixing contribution \eqref{hdym-renorm-gauge-fixing}, it is apparent that at high momenta $p$ the Feynman propagator for the gauge field scales as
\begin{equation}
D^{ab}_{\mu\nu}(p) \sim  \frac{ \delta^{ab} \delta_{\mu\nu} }{(p^2)^2},
\end{equation}
while the ghost fields have the usual propagator behaving at high energies like
\begin{equation}
D^{ab}(p) \sim  \frac{ \delta^{ab} }{{p}^2}.
\end{equation}


Let us consider a  Feynman diagram with $n_A$ internal gauge lines, $n_c$ internal ghost lines, $V_n$ vertices with exactly $n$ gauge field lines and $V_c$ ghost vertices. From the interaction terms in \eqref{lagr-hdym}, it is clear that  $n$ takes the values $3$, $4$, $5$ and $6$. 
The number of loops in the diagram is 
\begin{equation}
L = n_A + n_c - \sum_{n=3}^6 V_n - V_c + 1;
\end{equation}
the number of external particles is
\begin{equation}
E_A
	=
\sum_{n=3}^6 n V_n + V_c - 2n_A,
\hspace{4em}
E_c
	=
2V_c - 2n_c.
\end{equation}
Combining these results the superficial degree of divergence is
\begin{equation}
\begin{split}
d 
	& =
4L - 4n_A - 2 n_c + \sum_{n=3}^6 (6-n) V_n + V_c
\\
	& =
6 - 2L - E_A - 2E_c
\end{split}
\end{equation}
This result is enough to prove the renormalizability of the theory. Indeed, at one-loop level the only candidates for a non-renormalizable divergent result are the two-point function of the gauge field and the ghost two-point function. The former is ruled out by gauge invariance; the latter, in principle logarithmically divergent, is actually finite since one derivative acts on one external ghost line.
All other diagrams with two or more loops are finite. This proves that the theory is super-renormalizable, in the sense that only a finite number of diagrams are divergent.
%%  Spiegazione: il diagramma e'
%%    __			and the derivative of the vertex 
%%   /  \			is on the external line
%%--*    *---		so no higher divergence
%%   \__/			in the integreal





\section{One-loop renormalization of the gauge sector}



In this Section we will compute the one-loop quantum correction to the gauge coupling $g$. We follow the path outlined in the first Chapter using the background field method. As discussed for the usual \ym{} theory in Section~1.5, we have to compute the divergence proportional to $\tr F_{\mu\nu} F_{\mu\nu} $, and therefore a clever choice of background solution is one with vanishing matter fields. The expansion therefore reads, as in \eqref{ym-shift-field},
\begin{equation}
	A_\mu \rightarrow A_\mu + B_\mu 
\end{equation}
with the same notation as before; in particular the covariant derivative $\covD_\mu$ and the field strength tensor $F_{\mu\nu}$ are expressed in terms of the background field only. Their expressions have been given in \eqref{ym-shift-covD} and \eqref{ym-shift-Fmunu}.





We now expand the Lagrangian density about the classical solution. 
The expansion of the  kinetic term $\sim \tr F_{\mu\nu} F_{\mu\nu}$ can be straightforwardly read from the \ym{} case studied in Section~1.5.1; up to a total derivative, \eqref{YM-operator-implicit-indices} reads %% the weight $H$ used that time is indeed comprised in REF, and the result is therefore
\begin{equation}
\label{LW-ym-expansion}
\left( F^a_{\mu\nu} \right)^2
\rightarrow
2  A_\mu \cdot   \left[
  	- ( \covD^2 ) \delta_{ \alpha \beta }  
	- 2 F_{ \mu\nu }
  	\right]
 A_\nu 
-
2  (\covD_\mu A_\mu ) \cdot (\covD_\nu A_\nu) 
\end{equation}
where all the fields are written in the adjoint representation.



The expansion of the other two terms is quite lengthy and technical; the results are
\begin{align}
\label{LW-DF2-expansion}
\left(\covD_\mu F_{\mu \beta}^a \right)^2  
& \rightarrow 
 	A_\alpha \cdot \delta_{\alpha \beta} \covD^4 A_\beta
 	+ (\covD_\mu A_\mu) \cdot \covD^2 (\covD_\nu A_\nu )  \\\nonumber
& \hspace{6em}
 	+ 4\ A_\alpha \cdot F_{ \alpha \beta} \covD^2 A_\beta
 	+ 4\ A_\alpha \cdot F_{ \alpha \mu} F_{\mu \beta } A_\beta
\\\nonumber
\\
\label{LW-FFF-expansion}
f^{abc} F_{\mu\nu}^a F_{\mu\lambda}^b F_{\nu\lambda}^c
& 
\rightarrow
A_{\alpha} \cdot
	\left[
		3 F_{ \mu \nu } \delta_{ \alpha \beta }  
		+  3 F_{ \alpha \beta} \delta_{ \mu \nu } 
		- 6 F_{ \mu \beta }\delta_{ \alpha \nu } 
	\right]
	 \covD_{ \mu } \covD_{ \nu } A_{\beta}.
\end{align}
These formulae are derived in Appendix~C; in the rest of this Section we manipulate these expressions in order to get the final operator and perform the computation of the divergence.



Putting together the contributions \eqref{LW-DF2-expansion}, \eqref{LW-FFF-expansion} and \eqref{LW-ym-expansion}, the part of the Lagrangian that is quadratic in the fluctuation reads
\begin{equation}\label{LW-expansion-intermeriate}
\begin{split}
{\Lagr_{\YM,A^2}^\HD}
	= 		\frac{1}{2 g^2 m^2 } A^a_\alpha   &
			\bigg[
				\delta_{\alpha \beta} \covD^4
  				+ W^{\mu\nu}_{\alpha\beta} \covD_{ \mu } \covD_{ \nu } 
  				+ U_{\alpha\beta}
 			\bigg]^{ab}
			 A^b_\beta \\
 	& - \frac{1}{2 g^2 m^2}( \covD_\mu A_\mu )^a
 			\left[
 				- \covD^2 +  m^2	 
 			\right]^{ab}
 			( \covD_\nu A_\nu )^b,
\end{split}
\end{equation}
where
\begin{align}
\label{V}
W^{\mu\nu}_{\alpha\beta} & =  \left[
  					\left(
  						(4 + 3 \gamma) F_{ \alpha \beta} 
	  					- m^2 \delta_{\alpha \beta} 
	  				\right) \delta_{\mu\nu}
  					- 6 \gamma F_{(\mu|  \beta }\delta_{|\nu ) \alpha }
  				\right],
  				\\
\label{U}
U_{\alpha \beta} & = \frac{3}{2} \gamma
 					F_{ \mu \nu } F_{ \mu \nu  }\delta_{ \alpha \beta }
	  			- 2 m^2 F_{\alpha \beta }
				+ \left(
				 		4 + 3 \gamma
				 	\right) F_{ \alpha \mu} F_{\mu \beta }.
\end{align}

With this result, we are now ready to proceed to the quantization of the theory and to the determination of the divergences in the one-loop effective action.
The gauge fixing functional that we will use to perform the computation is again 
\begin{equation}
G\left[A+\mathcal{A}\right](x) = \covD_\mu A_\mu.
\end{equation}
The form of the expanded Lagrangian density \eqref{LW-expansion-intermeriate} suggests to use the integration weight functional
\begin{equation}
H = -\frac{1}{g^2 m^2}\covD^2 + \frac{1}{g^2} ,
\end{equation}
that cancels exactly the terms written in the second line of \eqref{LW-expansion-intermeriate}. The `$1$' is the same contribution for the \ym{} case with the background field approach; the Laplacian cancels the analogous term originating from $(\covD F)^2$. As we anticipated before, the choice that we made for the functional $G$ and the operator $H$ is motivated by the fact that they make the Lagrangian density simpler. Indeed, it exactly cancels against the term in the second line of the expanded Lagrangian \eqref{LW-expansion-intermeriate}. 
The contribution of the Jacobian determinant of the gauge fixing functional has already been evaluated in \eqref{M0}, hence here we recall that the relevant operator is
\begin{equation}
M_0
	=
-\covD_\mu \covD_\mu .
\end{equation}

Considering he Lagrangian density (2.2.5),
the quantity in brackets in the first line of  closely resembles an operator of the form \eqref{hk-fourth-order-generic}. In order to apply the procedure described in Section~1.2 and 1.3 to compute the divergences and the $\beta$ function, we have to make sure that the operators involved are self-adjoint, up to total derivatives: we have to define the coefficients with the symmetries described in \eqref{hk-fourth-order-generic-coefficeints}.


Let us consider the coefficients of dimension two, \ie
\begin{equation}\label{LW-operator-D2-prov-coeffs}
W^{\mu \nu}_{\alpha\beta} =
			\left[
  				(4 + 3 \gamma) F_{ \alpha \beta} 
	  			- m^2 \delta_{\alpha \beta} 
	  			\right] \delta^{\mu\nu}
  				- 6 \gamma F\indices{^{(\mu}_\beta }\delta^{\nu )}_{ \alpha };
\end{equation}
some indices have been raised for notational purposes but without any meaning since we are in Euclidean spacetime. We have to implement symmetry with respect to the transposition  of the \( A_\mu \)'s;  this corresponds to swapping \emph{simultaneously} both the gauge indices and the spacetime indices -- with the notation of  \eqref{LW-expansion-intermeriate}, the couples \( (a,\alpha) \) and \( (b,\beta ) \). This operation has different consequences on the three terms of $W^{\mu \nu} $. Indeed, the first and the third terms are antisymmetric under the exchange of the gauge indices, whereas the second one is symmetric; hence, the spacetime indices must be respectively antisymmetrized and symmetrized.



Therefore the coefficients of dimensions two for the self-adjoint operator read
\begin{equation}\label{LW-V-final}
\left(V^{\mu \nu}\right)_{\alpha \beta}
 	= 
 	\left[
  	(4 + 3 \gamma) F_{ \alpha \beta} 
	- m^2 \delta_{\alpha \beta} 
	\right] \delta^{\mu\nu}
  	- 6 \gamma F\indices{^{(\mu}_{[ \beta } } \delta^{\nu )}_{{ \alpha] } }.
\end{equation}
The non derivative term, on the other hand, is already symmetric, and hence
\eqref{U}
is the correct coefficient.



 The relevant operator for the gauge field therefore reads
\begin{equation}
\Delta_A = \covD^4 + V^{\mu\nu} \covD_\mu \covD_\nu + U
\end{equation}
with $V^{\mu\nu}$ and $U$ given in \eqref{LW-V-final} and \eqref{U}, and we can proceed to the evaluation of its Seeley-deWitt coefficient. 





\subsection{Divergences}


We are now ready to perform the computation of the divergences of one-loop quantum effective action.

Comparing with the general expressions for $\Gamma_{(1)}$, we can see that our total coefficient \eqref{b4-total} reads the same of Yang-Mills Lagrangian
\begin{equation}\label{b4-tot-hdym}
b_4^\tot
	=
b_4(\Delta_A)
- b_4(H)
- 2 b_4(M_0).
\end{equation}

Let us now evaluate the first contribution, $b_4(\Delta_A)$, according to \eqref{b4-coeff-4order}.
 We start by computing the trace on `external' indices of \( V^{\mu \nu} \)
\begin{equation}
\begin{split}
V_{\alpha \beta} \equiv \left( V^{\mu\mu} \right)_{\alpha \beta } 
	& =
		4 (4 + 3 \gamma) F_{ \alpha \beta} 
		- 4  m^2 \delta_{\alpha \beta} 
		- 6 \gamma F\indices{^{\mu}_{[ \beta } } \delta^{\mu }_{{ \alpha] } }\\*
	& = 
		(16 + 6 \gamma) F_{ \alpha \beta} 
		- 4  m^2 \delta_{\alpha \beta} .
\end{split}
\end{equation}
Then the the relevant contributions read
\begin{align}
\begin{split}
\Tr V^2 
	& 	= \tr
			\left[ 
				(16+6\gamma)^2 F_{ \alpha \beta } F_{ \beta \alpha }
				+ 4 \cdot 16 m^2 \ 1_{\text{ad}}^G
			\right] \\
	&	= 4(8 + 3 \gamma)^2 C_2 F^a_{\mu\nu}F^a_{\mu\nu} +  64 m^4 \ d_\ad 
\end{split}
	\\ \nonumber \\
\begin{split}
\Tr V^{\mu \nu} V^{\mu \nu}		
	&	= \tr 
			\big[ 
				4 (4+3\gamma)^2 F_{ \alpha \beta } F_{ \beta \alpha }
				+ 36 \gamma^2 F_{\alpha \beta  } F_{ \beta  \alpha}
			\\
			& \qquad\qquad
				- 12\gamma (4 + 3 \gamma ) F_{\alpha \beta} F_{\beta \alpha}
				+ 16 m^4 \  1_{\text{ad}}^{\GroupName{G}} 
			\big]\\
	&	= (36 \gamma^2 + 64 \gamma + 48)C_2 F^a_{\mu\nu}F^a_{\mu\nu}  +  16 m^4 \ d_{\ad}
\end{split}
	\\ \nonumber \\
\begin{split}
\Tr U	
	&	= \tr
			\left[
				6 \gamma F_{ \mu \nu } F_{ \mu \nu }
				+ (4 + 3 \gamma ) F_{\alpha \mu} F_{ \mu \alpha}
			\right] \\
	& = ( 4 - 3 \gamma)C_2 F^a_{\mu\nu}F^a_{\mu\nu}  
\end{split}
\end{align}
where we used
\begin{equation}
F\indices{^{\mu}_\beta }\delta^{\nu }_{ \alpha }
F\indices{^{(\mu}_{ [ \alpha } }\delta^{\nu )}_{\beta ]  }
	= 
F\indices{^{\mu}_\beta }
F\indices{^{(\mu}_{ [ \alpha } }\delta^{ \alpha )}_{\beta ]  }
	=
- F_{ \mu \beta } F_{\mu \beta}
\end{equation}
and \eqref{notation-trFmunuFmunu}. \(d_\ad = \tr 1_{\text{ad}}^{\GroupName{G}} = \dim \text{Lie}\ \GroupName{G}\) is the dimension of the adjoint representation, \ie the number of colors.

We can now sum up the contributions with the correct weight in order to get the coefficient the fourth order operator, obtaining
\begin{align}\label{b4-operator-hdym}
b_4(\Delta_{A})
%% =\phantom{.}  &
%%	  \frac{1}{6} (-4 C^2 F^2)   \\
%%	& + \frac{1}{24} \left[ C_2 F^2 (36 \gamma^2 + 48 \gamma + 64) +  16 m^4 \ \tr_G 1_{\text{ad}}^G \right]\\
%%	& + \frac{1}{48}  \left[ 4(8 + 3 \gamma)^2 C_2 F^2 +  64 m^4 \ \tr_G 1_{\text{ad}}^G \right] \\
%%	& - C_2 F^2  ( 4 - 3 \gamma) \\
 =\phantom{.} &		
 	 \left( \frac{9}{4} \gamma^2 + 9 \gamma + \frac{10}{3} \right) C_2 F^a_{\mu\nu}F^a_{\mu\nu} 
 	+ 2 m^4 d_\ad
\end{align}
The gauge fixing contribution $M$ has been evaluated in \eqref{b4-M0-operator}
\begin{equation}
b_4 ({ M } ) = - \frac{1}{12} C_2 F^a_{\mu\nu}F^a_{\mu\nu}    ;
\end{equation} 
the integration weight $H$ contributes, using \eqref{b4-second-order}, with
\begin{align}\label{b4-H-HD-operator}
b_4 (H) & = - \frac{1}{12} C_2 F^a_{\mu\nu}F^a_{\mu\nu}    + \frac{m^4}{2} d_\ad.
\end{align}



The total coefficient \eqref{b4-tot-hdym} reads
\begin{equation}\label{b4-tot-hd-gauge}
b_4^\tot  =  \left( \frac{9}{4} \gamma^2 + 9 \gamma + \frac{43}{12} \right) C_2 F^a_{\mu\nu}F^a_{\mu\nu} + \frac{3}{2} m^4 d_\ad.
\end{equation}
A remarkable feature of this result is that the the one-loop effective action contains only a contribution proportional to usual the Yang-Mills  kinetic; this implies that only a renormalization of $g$ is necessary. The second contribution is just a constant independent of the fields contributing to the vacuum energy.

We can then identify, according to \eqref{ren-1loop-contribution} and \eqref{running-coupling}, the required renormalization
\begin{equation}
g^{- 2 }_\mu = g^{- 2 }_\Lambda  - \frac{ \bar \beta }{ 16 \pi^2 } \log \frac{\Lambda}{\mu},
\end{equation}
where
\begin{equation}
\bar \beta  = \left( 9 \gamma + 36 \gamma + \frac{43}{3}\right)C_2
\end{equation}
that gives, applying \eqref{beta-function-generic}, the beta function for the \ym{} coupling $g$
\begin{equation}\label{beta-hd-ym}
\beta(g)  = - \frac{g^3}{32\pi^2}\bar \beta =  - \frac{g^3 C_2}{16 \pi^2} \left( \frac{9}{2} \gamma^2 + 18 \gamma + \frac{43}{6}\right);
\end{equation}
Figure~\ref{FIG_plot-beta} shows a plot of $\beta$ as a function of the parameter $\gamma$.


\begin{figure*}[!ht]
\centering
\begin{tikzpicture}
\begin{axis}[ 
	title = $\beta$ function for $\Lagr^\HD_\YM$, 
	xlabel = $\gamma$,
	ylabel = $\beta \left/ \left( \dfrac{g^3 C_2 }{ 16\pi^2 }\right) \right. $ ,
	xmax=1.1,
	ymax=12.5,
	xmin=-5.1,
	ymin=-12.5, 
	samples=500,
	minor x tick num = 1,
	width=12cm,
	height=7cm
	]
  \addplot[blue, ultra thick] (x,-4.5*x*x - 18*x - 43/6);
  \addplot[dashed, red,  thick, domain=-20:20] (-3.55,x);
  \addplot[dashed, red,  thick, domain=-20:20] (-0.45,x);
  \addplot[gray, domain= -10:10] {0};
\end{axis}
\end{tikzpicture}
 \captionof{figure}{$\beta$ as a function of the parameter $\gamma$. External to vertical lines: $\beta<0$, \ie asymptotic freedom; internal: $\beta >0$, \ie UV Landau pole. }
 \label{FIG_plot-beta}
\end{figure*}


A few comments on the one loop result are in order. As already mentioned, the divergent contribution turns out to be proportional to the term $\tr F^2$ only, so that $g$ gets renormalized but the product $gm$ and $\gamma$ do not. This implies that $m$ runs in the opposite way than $g$.

Then, the $\beta$ function is independent of $m$, but depends on the parameter $\gamma$ that can be fine-tuned to make the former zero; in this case the flow of the renormalization group is trivial and conformal symmetry is not broken at quantum level. The values of $\gamma$ that make the $\beta$ function vanish are 
\begin{equation}
\gamma_{\pm} = -2 \pm \sqrt{\frac{65}{27}} \approx -3.55  \text{,}\ -0.45;
\end{equation}
we can therefore distinguish the two regimes
\( 
	\gamma_- < \gamma < \gamma_+
\)
where $\beta < 0$, and 
\(
\gamma < \gamma_- \vee \gamma > \gamma_+
\)
that gives
\(
\beta > 0
\). We also observe that $\beta$ is maximized when \( \gamma = -2 \) and its value is
\(
\beta(g)  =  {65 g^3 C_2} / {93 \pi^2}  
\)
but this does not seem to have any deeper implication.
These considerations are actually meaningful since, as mentioned, $\gamma$ does not run under renormalization.


In the first case the theory is asymptotically free, that is the same situation of the usual Yang-Mills theory; the coupling decreases towards zero with increasing scale, and according to this one loop result has an infrared Landau pole. Conversely, in  the second case, the coupling $g$ grows with energy and the theory has a UV Landau pole.

It is also interesting to notice that for $\gamma = 0 $ the $\beta$ function differs from the Yang-Mills case. Indeed, the coupling runs roughly twice as fast; this is nothing but a consequence of the additional degrees of freedom in the Lagrangian discussed in \eqref{lagr-HD-ym-splitted-dof}.

In the literature two different values for the beta function \eqref{beta-hd-ym} were reported. In \cite{Fradkin:1981iu} the computation techniques discussed here was employed, but the reported result is different. In its preprint \cite{Fradkin:preprint}  few more details are given, and some inconsistencies can be found already observing the symmetry properties of the coefficients of the derived operators.
On the other hand, \cite{Grinstein:2008qq} and \cite{Schuster}, using a conventional diagrammatic approach to the computation, obtained a result equivalent to ours, confirming the computation.



\section{Gauge theory with matter fields}

In this Section we consider the higher-derivative gauge theory with Lagrangian \eqref{lagr-hdym} coupled to matter fields. We will consider both (real) scalar and Weyl fermionic fields, studying the renormalization properties of the gauge sector only.

We are interested in the renormalization properties of the gauge sector, hence the computation we are going to do here is in fact similar to that we did for the usual \ym{} field in Section 1.5.2; for the same reasons we need just the divergence proportional to $\tr F_{ \mu \nu } F_{ \mu \nu } $  and therefore we choose a background with vanishing matter fields and non-zero gauge fields $B_\mu$. The expansion therefore will be performed as in \eqref{bkg-with-matter} that we repeat here for clarity, 
\begin{equation}
A_\mu \rightarrow A_\mu + B_\mu, 
\hspace{3.5em} 
\phi_i \rightarrow \phi_i,
\hspace{3.5em}
\psi_j \rightarrow \psi_j,
\end{equation}
with the same notation as before, that is $\phi$, $\psi$ and $A_\mu$ are the quantum fluctuation. Again, $\covD_\mu$ and $F_{\mu\nu}$ are expressed in terms of the background field only.


%% We extend the previous computation in order to consider the matter fields as well; since the condition REF is still valid, we consider . This will greatly simplify the expression we will be dealing with. In particular, the Lagrangian density for matter fields is at least quadratic in them, so that the expansion to get the term quadratic in the fluctuation is immediate: all higher order terms can be truncated away and the gauge field is directly evaluated at its background value.





The structure of the Lagrangian that we are going to consider is then
\begin{equation}
\Lagr =
	\Lagr_{\YM}^\HD + \Lagr_\phi + \Lagr_\psi,
\end{equation}
where \(\Lagr_{\YM}^\HD\) is \eqref{lagr-hdym}; \( \Lagr_\phi \) and \( \Lagr_\psi \) are the matter field Lagrangian densities with at most quadratic contributions in matter fields and with minimal coupling with gauge fields.
We are ignoring interactions between matter fields because they would be terms at least of third order in the fields themselves, and for the background about which we expand the Lagrangian they do not contribute to the terms quadratic in the fluctuations.


Considering usual-derivative matter fields, power counting arguments analogous to those performed in Section 2.1.2 lead to the conclusion that such theories are renormalizable if all the coefficients of the terms containing $\phi$ or $\psi$ have mass dimension grater or equal than zero. For higher-derivative matter of the type we are considering here, the mass dimension of any coefficient can be greater or equal than $-2$ without spoiling renormalizability. Going through these results would be long and not very instructive since it is just matter of playing with identities about graph topology, therefore we just accept this result and explicitly verify the renormalizability of the gauge sector under one-loop corrections.


We quantize the theory with the same gauge-fixing functional used above, \ie $G[A + \mathcal{A}]= \covD_\mu A_\mu$, and we will consider the same integration weight \( H = 1/g^2 - \covD^2/g^2m^2 \). 
As a consequence, the heat kernel coefficients  $b_4$ for the gauge fixing terms and for the gauge field are unaffected by the matter sector, and therefore are again those computed in \eqref{b4-M0-operator}, \eqref{b4-H-HD-operator} and \eqref{b4-operator-hdym}.


Considerations similar to those outlined for the \ym field coupled matter, bring us to the conclusion that, applying \eqref{eff-act-1loop-gauge}, we have to compute the total coefficient
\begin{equation}\label{b4-total-most-general}
b_4^\tot = 
b_4(\Delta_A)
- 2 b_4(M_0)
- b_4(H)
+ \sum_i b_4(\Delta_{\phi,i})
-2 \sum_j b_4(\Delta_{\psi,j}).
\end{equation}



%%Applying the formulae derived in Chapter MM, we can compute the effective action as
%%\begin{equation}
%%\Gamma_{(1)} =
%%	\frac{1}{2}
%%		\left[
%%			\log \det \hat \Delta_A
%%			+
%%			\sum_i \log \det \hat \Delta_{\phi,i}
%%			-
%%			2 \sum_j \log \det \hat \Delta_{\psi,j}
%%			-
%%			\log \det H
%%			-
%%			2 \log \det M|_\cl
%%		\right].
%%\end{equation}
%%to which the total coefficient $b^{\tot}_4$ reads



In the following we analyse the cases of usual-derivative matter and higher derivative matter. Mixed combinations could also be considered, but the number of free parameters makes the situation quite involved. For simplicity we will consider massless matter only.





\subsection{Usual derivative matter fields}


First we briefly cover the somewhat trivial theory describing higher-derivative gauge fields coupled with usual-derivative matter.

Given the restrictions that we imposed on the terms that can be present in the Lagrangian density, the contribution for scalar fields can only read
\begin{equation}
\Lagr_\phi = 
	\left( \covD_\mu \phi_i \right) \left( \covD_\mu \phi_i \right) 
\end{equation}
and that of the spinor field is
\begin{equation}
\Lagr_\psi = 
	i \psi_j \sigma^\mu \covD_\mu \bar \psi_j.
\end{equation}
$i$ and $j$ are generic indices enumerating the matter fields $\phi_i$ and $\psi_j$; no  structure related to the gauge symmetry is associated with these indices, that are summed when repeated. The fields might transform under any representation of the gauge group, that means that $\phi_i$ or $\psi_j$ could carry internal indices, whose contraction is understood.

The Langrangian densities are the same of the matter sectors studied for the usual \ym{} theories in Section~1.5, so that we can directly take the results \eqref{b4-scalar-usual-der} and \eqref{b4-scalar-higher-der}, \ie  
\begin{equation}
b_4( \Delta_{\phi,i} ) = - \frac{1 }{12}  C_{\phi,i} F^a_{\mu\nu} F^a_{\mu\nu} 
\end{equation}
for scalar fields, and
\begin{equation}
b_4( \Delta_{\psi,j} ) = \frac{1}{6} C_{\psi,j} F_{\mu\nu}^a F_{\mu\nu}^a  
\end{equation}
for spinor fields.


The total coefficient \eqref{b4-total-most-general} now reads
\begin{equation}
\begin{split}
b_4^\tot
	& =
F_{\mu\nu}^a F_{\mu\nu}^a
\bigg[
\frac{9}{4} \gamma^2 + 9 \gamma + \frac{43}{12}  
- \frac{1}{12} \sum_i C_{\phi,i}
- \frac{1}{3}  \sum_j C_{\psi,j}
\bigg]
%% \\
%%&\qquad
+ \frac{3}{2} m^4 d_\ad
\end{split}
\end{equation}
that again implies a renormalization of the \ym{} coupling $g$ only, and includes the vacuum energy contribution. Such divergence gives the $\beta$ function
\begin{equation}
\beta(g)
	=
- \frac{g^3}{8 \pi^2}
\bigg[
\frac{9}{4} \gamma^2 + 9 \gamma + \frac{43}{12}  
- \frac{1}{12} \sum_i C_{\phi,i}
- \frac{1}{3}  \sum_j C_{\psi,j}
\bigg]
\end{equation}
This result is of course just the same as in the case \ym{} with matter with the difference that the gauge field contribution is given by \eqref{b4-tot-hd-gauge}.
Depending on the value of $\gamma$ and the representation in which the matter fields are, the computed $\beta$ function can cause different renormalization group flow behaviours.


%%
%%The Lagrangian densities REF and REF are already quadratic in the matter fields, so in order to expand it about the background solution REF it is enough to evaluate the covariant derivative for the background value of the gauge field. Upon integration by parts of the scalar fields, the operators read
%%\begin{equation}
%%\Delta_{\phi,i} = - \covD^2,
%%\hspace{5em}
%%\Delta_{\psi,j} = i \sigma^\mu \covD_\mu .
%%\end{equation}
%%Their determinant is ready to evaluated using REF and REF, and the result is the 
%%




\subsection{Higher derivative matter fields}


Now we extend the Lagrangian densities for the matter fields  in order to include higher derivative contributions.

The most general contribution that we can have in the scalar sector is
\begin{equation}\label{lagr-hd-scalars}
\Lagr^\HD_\phi = 
- \frac{1}{2 g^2} \phi_i \covD^2 \phi_i  
+ \frac{\delta_{1,i}}{2 g^2	m^2}  \phi_i \left[  ( \covD^2 )^2 
						+ \delta_{2,i} F_{\mu\nu} F_{\mu\nu}
						\right] \phi_i
			,
\end{equation} where $\delta_{k,i}$ are generic real coefficients.
This is a quite general expression at least for our purposes: There cannot be any term \( \sim \covD_\mu \covD_\nu \), since no rank-2 symmetric Lorentz-scalar tensor of mass dimension 2 is available to contract those indices;\footnote{Of course a choice could be $m^2 \delta^{\mu\nu}$, but it is actually the ordinary-derivative contribution already taken into account in the Lagrangian density \eqref{lagr-hd-scalars}.} any mixing like $ \phi_i \Sigma_{ij} F_{\mu\nu} F_{\mu\nu} \phi_j $ can be eliminated by diagonalising $\Sigma_{ij}$ since the kinetic term is proportional to $\delta_{ij}$; a contribution of the form $\sim \covD_\mu F_{\mu\nu} \covD_\nu $ is a $N_\mu$ term (in the notation of \eqref{hk-fourth-order-generic}) so it does not contribute to the divergence. 

The operator for (Weyl) spinor fields that we will consider is
\begin{equation}\label{lagr-hd-spinors}
\begin{split}
\Lagr^\HD_\psi =  
& \frac{1}{g^2}  \psi_j i \sigma^\mu \covD_\mu \bar \psi_j \\
& + \frac{\tau_{1,j}}{ g^2 m^2 }  \psi_j \bigg[ 
		i \sigma^\mu  \covD_\mu \bar\sigma^\nu\covD_\nu \sigma^\rho \covD_\rho
		-i  \tau_{2,j} F_{ \mu \tau } \sigma^{\mu \tau}  \sigma^\rho \covD_\rho 
		-i \tau_{3,j} F_{ \mu \nu } \sigma_\nu \covD_\mu 
		\bigg]  \bar \psi_j
\end{split}
\end{equation}
where $\tau_{k,j}$ are generic real coefficients.
This is less general than the Lagrangian for the scalar field: the spinor index structure allow also for other combinations of sigma matrices, and we are ignoring possible mixing between the spinors;  \eqref{lagr-hd-spinors} is a general expression and we dropped total derivatives or ineffective contributions as described for the scalar field.




\subsection{General aspects}

Similarly to the pure-gauge case, the theory defined by 
\( \Lagr =	\Lagr_{\YM}^\HD + \Lagr^\HD_\phi + \Lagr^\HD_\psi \)
enjoys some interesting renormalization properties. In particular we will see that a class of this kind of theories is not affected by the hierarchy problem.


%%%
%%%\subsubsection{Conformal symmetry}
%%%
%%%
%%%Conformal symmetry, 
%%%




\subsubsection{Degrees of freedom}

As done for the pure gauge Lagrangian, it is interesting to count the degrees of freedom conveyed by these kind of theories and see how they can be thought of in a different way. As outlined in \cite{CONFORMAL, Grinstein:2008qq}  We will consider scalars first, and then spinors; we will focus only on the kinetic term, since we are just considering the structure of the degrees of freedom.
We will consider only one field at time.

\paragraph{Scalar fields.} 
The higher derivative Lagrangian density reads
\begin{equation}
\Lagr_\phi = 
	\frac{1}{2} \partial_\mu \phi \partial_\mu \phi	
	+ \frac{1}{2 m^2} (\partial^2 \phi)^2.
\end{equation}
Off-shell each real scalar field describes, of course, just one degree of freedom; on shell, since the equation of motions are fourth order differential equations, one has to specify four initial data for each real field, corresponding to two degrees of freedom.

These additional degrees of freedom can be made manifest by introducing a real auxiliary field $\phi'$:
\begin{equation}\label{lagr-HD-scalar-splitted-dof-intermediate}
\Lagr_\phi = 
	\frac{1}{2} \partial_\mu \phi \partial_\mu \phi	
	- \frac{1}{2 } m^2 \phi'^2
	- \phi' \square \phi,
\end{equation}
upon equation of motion \( \phi' =  \partial^2  \phi / m^2 \), \eqref{lagr-HD-scalar-splitted-dof-intermediate} gives back the original kinetic term. Shifting the field  \( \phi \rightarrow \phi - \phi' \) in order to make the derivative term diagonal, we obtain
\begin{equation}\label{lagr-HD-scalar-splitted-dof}
\Lagr_\phi = 
	\frac{1}{2} \partial_\mu \phi \partial_\mu \phi	
	- \frac{1}{2} \partial_\mu \phi' \partial_\mu \phi'
	- \frac{1}{2} m^2 \phi'^2.
\end{equation}
In this Lagrangian density, the degrees of freedom are decomposed in a massless scalar field and a massive scalar field (with mass $m$), the latter having an unusual sign in front of its kinetic and mass terms. The number of degrees of freedom is conserved, since in the theory defined by \eqref{lagr-HD-scalar-splitted-dof} each scalar field propagates one degree of freedom. Similarly to the gauge field case, the massive particle is a ghost corresponding to the exponentially growing modes of the higher-derivative  formulation of the theory.


\paragraph{Spinor fields.} Off-shell there are 4 degrees of freedom; the higher derivative equations of motion are third order partial differential equations, so that the number of propagating degrees of freedom is $ 4 \cdot 3 / 2 = 6  $. In order to make such degrees of freedom manifest, we rewrite the Lagrangian as
\begin{equation}
\begin{split}
\Lagr_{\psi} =
&\phantom{.}	\bar \psi i \bar \sigma^\mu \partial_\mu \psi
	- m (\bar{\psi'} \bar{\psi''} + \psi' \psi'')%%\\
%%&
	+ \bar{\psi'} i \bar \sigma^\mu  \partial_\mu \psi 	+ \bar{\psi} i \bar \sigma^\mu \partial_\mu \psi' 
	- \psi'' i \sigma^\mu \partial_\mu \bar{\psi''}
\end{split}
\end{equation}
where we introduced {two} Weyl spinors \(\psi'\) and \(\bar \psi'' \), whose classical equations of motion read
\begin{align}
i \bar \sigma^\mu \partial_\mu \psi' - m \bar \psi''  = 0
		, \hspace{2em}
i \sigma^\mu \partial_\mu \bar \psi'' +  m \psi'  = 0 .
\end{align} A diagonal kinetic term is obtained by shifting $\psi \rightarrow \psi - \psi' $, so that the Lagrangian now reads
\begin{equation}\label{lagr-HD-spinor-splitted-dof}
\begin{split}
\Lagr_{\psi} & 
	= \bar \psi   i \bar \sigma^\mu \partial_\mu \psi 
	- \bar \psi' i \bar \sigma^\mu \partial_\mu \psi' 
	- \psi'' i \sigma^\mu \partial_\mu \bar{\psi''}
%%\\
%%&
%%\quad
	- m (\bar{\psi'} \bar{\psi''} + \psi' \psi'')
\\
&
	=  \bar \psi   i \bar \sigma^\mu \partial_\mu \psi 
	- \bar \Psi i \slashed{\partial} \Psi
	- m \bar \Psi \Psi
\end{split}
\end{equation}
where we emphasized that the two new spinors group together into a (massive) Dirac spinor $\Psi := (\psi' , \bar \psi'')$ as it is required for charged spinors in order to have a mass term.  Since a Dirac spinor propagates $4$ degrees of freedom, and the Weyl spinor 2, the total number of dynamical degrees of freedom is conserved.
Again, the auxiliary massive field has an extra minus sign in the Lagrangian that classify it as a ghost particle.






\subsubsection{On the hierarchy problem}

A particular class of such theories, moreover, solves the hierarchy problem, because the mass of the scalar grows only logarithmically with the cut-off, while the scalar mass in usual theories gets also quadratic corrections. The theory that we will consider here are those in which the higher derivative contribution is only on the kinetic term, so that the scalar field has a Lagrangian of the form 
\begin{equation}\label{hd1l-lagr-scalar-hierarchy}
\tilde \Lagr_{\phi} \sim (\covD_\mu \phi) (\covD_\mu \phi) - \frac{1}{m^2}  (\covD^2 \phi) (\covD^2 \phi) + V(\phi) 
\end{equation}
where the potential satisfies the usual renormalization prescriptions for conventional-derivative theories, \ie it is a polynomial of degree at most four. We ignore spinor contributions for simplicity, even though it is easy to see that Yukawa couplings would not change the result.

This remarkable feature is again a consequence of the improved UV behaviour of the propagator, scaling with $p^{-4}$ instead of $p^{-2}$. The proof of this fact relies on an explicit power counting argument on Feynman diagrams, that we will not go through.
From \cite{Grinstein:2008qq} we quote that the superficial degree of divergence is, denoting $E_X$ the number of external lines for the field $X$,
\begin{equation}
d = 6 - 2 L - E_A - 2 E_c - E_\phi,
\end{equation}
so that the only possible quadratic divergence for the two-point function of the scalar field is at one loop.
We will now explicitly show the vanishing of the coefficient of $\Lambda^2$ in the one-loop effective action in the context of the formalism that we are using in this thesis.

We need to compute the divergences for the coefficient of $\phi^2$ and the possible wavefunction renormalization; we therefore need to choose a background with non-vanishing scalar $\phi_{\text{bkg}}$, while fermion and gauge fields can be set to  zero. The operator obtained expanding the fields will be a fourth-order differential operator of the form \eqref{hk-fourth-order-generic}; the power divergence is weighted with the heat kernel coefficient $a_1$ (equivalently $ b_2$) that can be computed using the techniques described in Section 1.2 and turns out to be proportional to $V_{\mu\mu}$. But let us check what could contribute to $V_{\mu\nu}$: We would need a rank-two symmetric tensor with mass dimension 2 containing $\phi_{\text{bkg}}^2$, but the Lagrangian density \eqref{hd1l-lagr-scalar-hierarchy} does not allow for it, since it requires a contribution with the structure $\phi^2 \covD \phi \covD \phi$ to appear. For the same reason, since no divergence proportional to $(\covD \phi)^2$ can be present, the wavefunction does not get renormalized, at least at one-loop level, and this shows that there is no mass counterterm with power-law dependence on the cut-off.




\subsection{One-loop renormalization}
As  mentioned above, being the Lagrangian densities \eqref{lagr-hd-scalars} and \eqref{lagr-hd-spinors} already quadratic in the matter fields, the expansion about the background solution is immediate. 


\paragraph{Scalar fields.} The operator relevant to the scalar field reads
\begin{align}
\Delta_{\phi,i}
	& = ( \covD^2 )^2 
						+ \frac{m^2}{\delta_{1,i}}\delta^{\mu\nu} \covD_\mu\covD_\nu
						+ \delta_{3,i} F_{\mu\nu} F_{\mu\nu}
\end{align}
 The $b_4$ coefficient is ready to be evaluated using \eqref{b4-coeff-4order}, and the result is
\begin{equation}\label{b4-hd-scalar}
b_4(\Delta_{\phi,i})
	=
	- \left( \frac{1}{6} + \delta_{3,i} \right)C_{\phi,i} F^a_{\mu\nu} F^a_{\mu\nu}
	+ \frac{ m^4 }{ 2 ( \delta_{1,i} )^2 } d_{\phi_i}.
\end{equation}
where
\(
d_{\phi_i}
	=
\tr \1_{\phi,i}
\)
is the dimension of the representation of the scalar $\phi_i$.




\paragraph{Spinor fields.} The operator for the spinor field reads
\begin{align}
\Delta_{\psi,j}
	& =
	i \sigma^\mu  \bar\sigma^\nu \sigma^\rho \covD_\mu\covD_\nu  \covD_\rho
	- i \tau_{2,j} F_{ \mu \tau } \sigma^{\mu \tau}  \sigma^\rho \covD_\rho %%\\ \nonumber
%%	&	\quad
	- i \tau_{3,j} F_{ \mu \nu } \sigma_\nu \covD_\mu
	+ \frac{m^2}{\tau_{1,j}} i \sigma^\mu \covD_\mu  
\end{align}
In order to compute its contribution to the divergence, we apply the procedure described in \eqref{b4-coeff-3order-implicit}. The Seeley-deWitt coefficient therefore reads
\begin{equation}\label{b4-hd-spinor}
\begin{split}
b_4(\Delta_{\psi,j})
& =
\left[
-\frac{1}{2}	
%%+ \tau_{2,j}
%%+ 2 \tau_{3,j}
%%- \frac{1}{2} \tau_{2,j} \tau_{3,j}
%%- \frac{1}{2}	(\tau_{2,j})^2
%%- \frac{1}{4} (\tau_{3,j})^2
+ \tau_{2,j}
- \frac{1}{2} \tau_{2,j} \tau_{3,j}
- \frac{1}{2}	(\tau_{2,j})^2
- \frac{1}{4} (\tau_{3,j})^2
\right]
\tr F_{\mu\nu} F_{\mu\nu}
%%\\
%%& \qquad 
+ \frac{m^4}{2(\tau_{1,j})^2} d_{\psi_j}
\end{split}
\end{equation}
where \(d_{\psi,j} = \tr 1_{\psi,j}\) is the dimension of the representation of the field $\psi_j.$
The first term is confirmed from the immediate calculation when the $\tau$'s are all zero, since the operator factorizes $ \Delta_\psi = \Delta_1 \bar \Delta_{1} \Delta_1 $, and therefore $b_4(\Delta_\psi) = 3 b_4(\Delta_1) = -\frac{1}{2} \tr F_{\mu\nu} F_{\mu\nu} $.



\subsubsection{Computation of $ b_4(\Delta_\psi)$}

In this section we go through the steps that bring to \eqref{b4-hd-spinor}. For simplicity of notation, we understand the index $j$.

Considering  the leading symbol in the operator the right choice of first-order differential operator to compose it with is \( \Delta_1 = - i \bar \sigma^\mu \covD_\mu \), and we get
\begin{equation}
\begin{split}
\Delta_{\psi+1} 
	& =
\Delta_\psi \cdot \Delta_1
	=
%	\\
%	& =
(\covD^2)^2
+ 
V^{\mu\nu} \covD_{\mu} \covD_{\nu}
+ U
\end{split}
\end{equation}
with coefficients
\begin{align}
V^{\rho \kappa} & =
	\left(-1 + \frac{\tau_2}{2}\right) 
	F_{\mu\nu} \sigma^{\mu} \bar \sigma^\nu \delta^{\rho\kappa}
	- \tau_3 F\indices{^{(\rho|}_\nu }\sigma^\nu \bar\sigma^{|\kappa)}
	+ \frac{m^2}{\tau_{1}} \delta^{ \rho \kappa }
	\\
U & =
\frac{1}{4} (1 - \tau_2) F_{\mu\nu} F_{\rho \tau} \sigma^\mu \bar \sigma^\nu \sigma^\rho \bar \sigma^\tau
-  \frac{\tau_3}{2} F_{\mu\nu} F_{\mu\rho} \sigma^\nu \bar \sigma^\kappa
+ \frac{m^2}{2 \tau_{1}} F_{\mu\nu} \sigma^\mu \bar \sigma^\nu
\end{align}
where symmetric and antisymmetric parts of the product of two covariant derivatives have been separated in order to ensure the correct symmetry properties.

As a starting point we perform the trace of $V$ over external indices
\begin{equation}
V = V^{\mu\mu}
	=
\left(-4 + 2{\tau_2} + \tau_3 \right) 
	F_{\mu\nu} \sigma^{\mu} \bar \sigma^\nu 
	+ 4 \frac{m^2}{\tau_{1}} 
\end{equation}
and now we have all the necessary ingredients to compute the relevant traces.
By using \eqref{identity-spinor_trace-double-sigmamn} we get
\begin{align}
\begin{split}
\Tr V^2
	& = \left(-4 + 2{\tau_2} + \tau_3 \right)^2
	\tr F_{\mu\nu} F_{\rho\kappa} \
	\str \sigma^{\rho}\sigma^{\mu} \bar \sigma^\nu \bar \sigma^\kappa
\\
	& = -4 \left( 16 + 4({\tau_2})^2 + (\tau_3)^2 - 16 \tau_2 + 4 \tau_2\tau_3 -8 \tau_3 \right) \tr F_{\mu\nu} F_{\mu\nu}
%%	\\
%%	& \qquad
+ 16 \frac{m^2}{\tau_{1}} d_{\psi}
\end{split}	
\end{align}
and using \eqref{identity-4}
\begin{align}
\Tr U 
	& = \frac{1}{4} (1 - \tau_2) \tr F_{\mu\nu} F_{\rho \tau} \str \sigma^\mu \bar \sigma^\nu \sigma^\rho \bar \sigma^\tau   +
 \frac{\tau_3}{2} 2 \tr F_{\mu\nu} F_{\mu\nu}
\\\nonumber
& = - (1 - \tau_2 - {\tau_3}) \tr F_{\mu\nu} F_{\mu \nu}.
\end{align}
\(\Tr V^{\mu\nu} V^{\mu\nu}\) is a trickier for its spinor index structure.
Expanding the contraction one gets
\begin{equation}
\begin{split}
\Tr V^{\mu\nu}V^{\mu\nu}
&	=
4 \left(-1 + \frac{\tau_2}{2}\right)^2
\tr F_{\mu\nu} F_{\rho\kappa} \str \sigma^\mu \bar \sigma^\nu \sigma^\rho \bar \sigma^\kappa
\\
& \qquad
+ (\tau_3)^2 \tr F\indices{_{\rho\nu }} F\indices{^{(\rho|}_\mu } 
\str \sigma^\nu \bar\sigma^{\lambda} \sigma^\mu \bar\sigma^{|\lambda)}
\\
& \qquad
+ \tau_3  \left(-2 + {\tau_2} \right) 
	\tr F_{\mu\nu} F_{\rho \nu}
	\str \sigma^\mu \bar \sigma^\nu \sigma^\rho \bar \sigma ^\nu
\\
& \qquad
+ \frac{m^2}{\tau_{1}} d_{\psi}
\end{split}
\end{equation}
The first term is of the kind discussed above.
The third term, using again \eqref{identity-6}, reads
%%\begin{equation}
%%\bar\sigma^{\lambda} \sigma^\mu \bar\sigma^{ \lambda } = 2 \bar \sigma^\mu
%%\end{equation}, reads
\begin{equation}
\tau_3  \left(-2 + {\tau_2} \right) 
	\tr F_{\mu\nu} F_{\rho \nu}
	\str \sigma^\mu \bar \sigma^\rho 
= 
- 4 \tau_3  \left(-2 + {\tau_2} \right)  \tr  F_{\mu\nu} F_{\mu \nu}.
\end{equation}
Splitting the symmetrization in the second one we get
\begin{equation}
\frac{(\tau_3)^2}{2} \tr F\indices{_{\rho\nu }} F\indices{_{ \rho }_\mu } 
\str \sigma^\nu \bar\sigma^{\lambda} \sigma^\mu \bar\sigma^{ \lambda }
+
\frac{(\tau_3)^2}{2}  \tr F\indices{_{\rho\nu }} F\indices{_{\lambda}_\mu } 
 \tr \bar\sigma^{\rho} \sigma^\nu \bar\sigma^{\lambda} \sigma^\mu ,
\end{equation}
where we also used the cyclicity of trace. They can be evaluated with the same techniques already seen; indeed we get
%% using REF and  REF
%%they become
%%\begin{equation}
%%{(\tau_3)^2} \tr F\indices{_{\rho\nu }} F\indices{_{ \rho }_\mu } 
%%\tr \sigma^\nu \bar \sigma^\mu
%%-2  {(\tau_3)^2} \tr F\indices{_{\mu\nu }} F\indices{_{\mu}_\nu } .
%%\end{equation}
%%and by the symmetry of $\tr F_{\rho\nu} F_{\rho\mu}$ upon exchange of $\mu$ and $\nu$, using (??)
\begin{equation}
- 2 {(\tau_3)^2} \tr F\indices{_{\mu\nu }} F\indices{_{ \mu }_\nu } 
- 2  {(\tau_3)^2} \tr F\indices{_{\mu\nu }} F\indices{_{\mu}_\nu }  
= - 4 (\tau_3)^2 \tr F\indices{_{\mu\nu }} F\indices{_{\mu}_\nu }  .
\end{equation}
The result is
\begin{equation}
\begin{split}
\Tr V^{\mu\nu}V^{\mu\nu}
&	=
-4 \left[
	4 
	-4 \tau_2  
	-2 \tau_3 
	+ \tau_2 \tau_3
	+ (\tau_2)^2
	+ (\tau_3)^2  
\right] 
\tr  F_{\mu\nu} F_{\mu \nu} 
%%\\
%%&
+ 4 \frac{m^4}{(\tau_{1})^2}d_{\psi} .
\end{split}
\end{equation}


Summing those contributions together according to \eqref{b4-coeff-4order} we arrive at the partial result
\begin{equation}
\begin{split}
b_4(\Delta_{\psi+1})
& =
\left[
-\frac{2}{3}	
+ \tau_2
- \frac{1}{2} \tau_2 \tau_3
- \frac{1}{2}	(\tau_2)^2
- \frac{1}{4} (\tau_3)^2
\right]
\tr F_{\mu\nu}  F_{\mu\nu}
%%\\
%%&
%%\qquad
+  \frac{m^4}{2(\tau_{1})^2}d_{\psi} 
;
\end{split}
\end{equation}
the coefficient for the spinor operator is obtained subtracting the coefficient for $\Delta_1$, and the result is \eqref{b4-hd-spinor}.



\subsubsection{$\boldsymbol \beta$ function}

In order to evaluate the divergence we just apply \eqref{b4-total-most-general} with  \eqref{b4-hd-scalar} and \eqref{b4-hd-spinor}. %%; the resulting coefficient reads
%%\begin{align}
%%b_4^{\tot}
%%	& = b_4(\Delta_A) - 2  \sum_j b_4(\Delta_{\psi,j}) 
%%			+ 4  \sum_i b_4( \Delta_{\phi,i} ) 
%%			- b_4( M_0 ) - 2    b_4( H )
%%%
%%\\
%%	& = \bigg[
%%	\\
%%	&	\qquad	\bigg] \tr F_{\mu\nu} F_{\mu\nu}
%%%
%%\end{align}
The resulting expression is long and not insightful.
Again, the divergence is proportional to the \ym{} contribution only, so that $g$ gets renormalized but $\gamma$, $gm$, $\tau$'s and $\delta$'s do not run. The $\beta$ function reads 
\begin{equation}\label{beta-hd-ym-matter}
\begin{split}
\beta  = 
 - \frac{g^3}{16 \pi^2}
& \bigg[
\left( \frac{9}{2} \gamma^2 
+ 18 \gamma + \frac{43}{6}\right) C_2
%%\\
%%& \quad
- \sum_i	 \left( \frac{1}{3} + 2 \delta_{3,i} \right) C_{\phi,i}
\\
& \quad
+ \sum_j 2 \left(
- 1	
+ 2 \tau_{2,j}
-  \tau_{2,j} \tau_{3,j}
-  (\tau_{2,j})^2
- \frac{1}{2} (\tau_{3,j})^2
\right) C_{\psi,j}
\bigg]
\end{split}
\end{equation}
The renormalization group for this theory is rich and variable depending on the values of the parameters, and a complete discussion is impossible. As in the pure-gauge case \eqref{beta-hd-ym}, there is the possibility of a fine-tuning of the constant in order to have a vanishing $\beta$ function; this is consistent because the expression \eqref{beta-hd-ym-matter} is an invariant of the flow of the renormalization group.

A diagrammatic computation for a similar system has been performed in \cite{Grinstein:2008qq}.
%%BIB.
The gauge field and scalar contributions agree with those obtained here; the same is true for the spinor contribution with $\tau_2=0$.












