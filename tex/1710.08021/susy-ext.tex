

\chapter{Supersymmetric higher-derivative theory }

\pagestyle{fancy}
	\fancyhead{} % cancella tutti i campi
	\fancyhead[LE]{\scshape \leftmark}
	\fancyhead[RO]{\scshape \rightmark}
	\fancyfoot[CE,CO]{\thepage}
	\fancyfoot[LO,RO]{ }
	\fancyfoot[LE,RE]{ }
	\renewcommand{\headrulewidth}{0.4pt}
	\renewcommand{\footrulewidth}{0.4pt}

In this chapter we are going to discuss the supersymmetric generalization of the higher-derivative \ym{} Lagrangian \eqref{lagr-hdym}. 

In the first Section we will give some introductory observations about the problem, and we will motivate the need of formulating the theory in six spacetime dimension and then dimensionally reduce it. In Section~3.2 supersymmetry in six dimension is briefly discussed, and the formalism of harmonic superspace is presented, being the natural framework in which six dimensional supersymmetry is realised. After that, gauge theories are discussed and the higher derivative action for the vector multiplet is formulated.
In Section~3.3, the dimensional reduction of the theory is performed, and in the following two Sections the $d=4$, $N=1$ and $2$ supersymmetric higher-derivative Yang-Mills Lagrangians are formulated and the $\beta$ function of the gauge coupling is computed. In Section~3.6 the linearised $d=4$, $N=4$ Lagrangian is discussed and the $\beta$ function evaluated. Section~3.7 is devoted to some concluding comment.

The basic facts and notation about supersymmetric theories in four spacetime dimensions are given in Appendices~A.




\section{Introductory considerations}

The Lagrangian \eqref{lagr-hdym} that we are considering contains higher derivative of the gauge field; since supersymmetry mixes the fields in the Lagrangian, we expect the matter fields to have higher-derivative contributions as well.
What may be less obvious is that also the auxiliary fields get an higher derivative contribution, therefore becoming dynamical, as we are going to motivate in a few paragraphs.



To start with, let us consider for simplicity the case of an Abelian symmetry group, namely Maxwell theory. This case yields a somewhat trivial supersymmetric extension, but it points out many relevant features and serves as a `base case' against which we will verify more advanced techniques.
The higher-derivative Maxwell Lagrangian can be obtained from \eqref{lagr-hdym-comp}  and reads
\begin{equation}\label{lagr-hdmaxwell}
\Lagr_{\text{M}} = 
\frac{1}{4 g^2} F_{\mu\nu} F_{\mu\nu} 
- \frac{1}{4 m^2 g^2}  F_{\mu\nu}  \square  F_{\mu\nu}.
\end{equation}
where we also used \eqref{CovDF-dentity} and integrated by parts to rewrite the higher derivative term in this fashion. The term from the commutator of course vanishes being the theory Abelian.
We already know the $N=1$ supersymmetric Lagrangian containing the first contribution, namely the Lagrangian for super-Maxwell theory,
\begin{equation}
\Lagr_{\text{sM}} = 
\frac{1}{2 g^2} \left( \frac{1}{2} F_{\mu\nu} F_{\mu\nu} 
+2 i \bar \psi \bar \sigma^\mu \partial_\mu \psi - D^2  \right),
\end{equation}
whose action can be expressed in terms of a superspace integral as
\begin{equation}\label{act-supermaxwell-superspace}
S_{\text{sM}} = 
\frac{1}{4}
\int \dd{^4x} \dd{^2 \theta}\left[ W W + \text{hc}% \bar W \bar W 
\right],
\end{equation}
$W$ being the superfield strength.

Let us now notice that the higher derivative contribution in \eqref{lagr-hdmaxwell} consists in a simple insertion of a $ - \square = - \partial_\mu \partial_\mu$ inside the ordinary kinetic term. This suggests to insert such operator between the two factors of $W$ in the superspace action \eqref{act-supermaxwell-superspace}
\begin{equation}
S'_{\text{sM}} = 
- \frac{1}{4}
\int \dd{^4x} \dd{^2 \theta}\left[ W \square W +  \text{hc} %\bar W \square \bar W 
\right].
\end{equation}
Since $W$ and $\square$ are all gauge-invariant operator, this action is gauge invariant.
In terms of component fields the Lagrangian density corresponding to $S_{\text{sM}} + S'_{\text{sM}} $ reads
\begin{equation}\label{lagr-N=1-supermax}
\begin{split}
\Lagr^{N=1}_{\HD\text{sM}}
=
&
\frac{1}{2 g^2} \left( \frac{1}{2} F_{\mu\nu} F_{\mu\nu} 
+2 i \bar \psi \bar \sigma^\mu \partial_\mu \psi - D^2  \right)
\\
&\quad + \frac{1}{2 m^2 g^2} \left(- \frac{1}{2} F_{\mu\nu} \square F_{\mu\nu} 
- 2 i \bar \psi \bar \sigma^\mu \square \partial_\mu \psi +  D\square D
\right),
\end{split}
\end{equation}
In \cite{Gama:2011ws}  this model was used to construct a higher-derivative extension of QED, also adding matter fields.

The Lagrangian \eqref{lagr-N=1-supermax} is quite interesting and points out a number of features that must be taken into account in writing a supersymmetric theory for the higher-derivative \ym{} field. Though, it is still trivial in its dynamics, since it is a free theory also at the higher-derivative level. 

As anticipated, the auxiliary field becomes dynamical. According to the computation of the degrees of freedom described in the previous Chapter, \eqref{lagr-N=1-supermax} describes 5 ($A_{\mu}$) + 1 ($D$) bosonic and 6 ($\psi$) fermionic degrees of freedom, and hence the propagation of $D$ is indeed necessary to make the number of bosonic and fermionic degrees of freedom equal on-shell. Its kinetic term, however, has the extra minus sign that indicates its ghost nature.\footnote{Remember that the euclidean Lagrangian density is the energy.}
Notice that the presence of unphysical degrees of freedom was already recognised in the formulation of the higher-derivative theories in terms of two-derivative fields; this kind of multiplet structure of the ghost fields will be discussed later in some detail.

The non-Abelian generalization of this result is highly nontrivial for many reasons. We used the identity \eqref{CovDF-dentity} that in the case of a non-Abelian gauge group adds also another contribution. Moreover, the partial derivative, and as a consequence $\square$, is not a gauge invariant object any more; nor is the \ym{} superfield $W_\alpha$, and it is not clear how a super-invariant can be constructed.


In the case of Maxwell theory the term proportional to $\gamma$ in \eqref{lagr-hdym} vanishes since the gauge group is Abelian. However, we can argue that in general for a supersymmetric theory there cannot arise any also in the non-Abelian gauge symmetry. This is a consequence of the superfield formulation of the theory: The vector superfield $W^\alpha$ cannot construct a scalar Lagrangian density that is cubic in itself. We will therefore restrict to the case $\gamma = 0$.


We want to find the $N=1$, $2$ and $4$ extension of \eqref{lagr-hdym}, or at least to evaluate the one-loop $\beta$ function for such theories.
Instead of trying to find a supersymmetric formulation in four spacetime dimension, we will obtain the corresponding theory in $5+1$ spacetime dimensions and then dimensionally reduce it to four dimensions. This is motivated by many reasons.



The unextended supersymmetry algebra in six spacetime dimensions is indeed very similar to the $N=2$ algebra in four spacetime dimension; this comes, of course, with no surprise, because the dimensional reduction of the former yields automatically the latter. In this context, working in six spacetime dimensions simplifies the multiplet that we have to deal with, because it consists only of the gauge field, of one spinor and of the auxiliary fields; the rest of the $N=2$, $d=4$ supermultiplet is  generated via dimensional reduction.
We will also see that studying $N=2$ higher derivative \sym{} will provide enough information to compute, at least at one-loop, the  $\beta$ function of the $\beta$ function for the extended $N=4$  case.







\section{Preliminaries}

In this Section we will give some generalities of the six dimensional spinors and supersymmetry algebra. We will then introduce a superspace that represents in a manifest way the group structure associated with the relevant fields, namely the harmonic superspace. This will allow us to formulate the gauge theory in six dimensions, then to be dimensionally reduced to four spacetime dimension.
Appendix~B gives a review of supersymmetry in four spacetime dimensions


As will turn out in the computation, the construction of a higher derivative Lagrangian in $d=6$ is much more straightforward than the formulation of the usual \ym{} action. This could be related to the fact that in six dimensions such theories might be more natural, since the coupling of the higher derivative term happens to be dimensionless (while in \eqref{lagr-hdym} it has dimension $-2$).


Six dimensional Minkovsky spacetime is the natural generalisation of the four dimensional one. It is described by coordinates $x^M$, $M=0,1,\ldots,5$; here we will be working with the mostly positive metric $\eta^{MN} = \text{diag}({}-{}+{}+{}+{}+{}+{})$, and Wick-rotate the Lagrangian at the end, in order to get a formally convergent 
functional integral to proceed as discussed in Chapter~1. As mentioned in the Introduction, though, this is not a well-defined operation and we use it as a formal tool to simplify the notation, but such subtleties are not of interest for this work.


For completeness and future reference, the covariant derivative and the field strength tensor are 
\begin{equation}\label{6dim-covd-fmunu}
\covD_M = \partial_M + A_M
\hspace{5em}
F_{MN} = [\covD_M, \covD_N ].
\end{equation}
The Laplacian is denoted with $\tilde \covD^2 = \covD^M \covD_M$, the tilde serving to distinguish it from the four dimensional one.
Gauge transformations act on the gauge fields as
\begin{equation}\label{6dim-covd-fmunu-gaugetrans}
\delta_\omega A_M = - \covD_M \omega
\hspace{5em}
\delta_\omega F_{MN} = [\omega, F_{MN}].
\end{equation}





\subsection{Six-dimensional spinors}



In this Section we recall some facts about spinors in the six-dimensional Minkovsky spacetime $\mathbb{R}^{1,5}$. A complete treatment of the spinor representation in six (and other) dimensions can be found in \cite{Kugo:1982bn,VanProeyen:1999ni}.


In order to describe spinor fields we start by considering the relevant Clifford algebra $\GroupName{Cl}(1,5)$ defined by the anticommutation relation
\begin{equation}
\left\{
\Gamma^M, \Gamma^{N}
\right\}
=
2 \eta^{MN} 
\end{equation}
for $M,N = 0,1,\ldots,5$. $\Gamma_M$ can be represented using $ 8 \times 8 $ complex matrices.
In full analogy with four dimensional spacetime, 
\begin{equation}
\Gamma^\dagger_0 = - \Gamma_0,
\qquad\qquad\qquad
\Gamma^\dagger_I = \Gamma_I,
\qquad\qquad
I = 1,\ldots,5,
\end{equation}
and
\begin{equation}
\Gamma_M^\dagger = - \Gamma_0 \Gamma_M \Gamma_0^{-1}.
\end{equation}
We also define the matrix $\Gamma^7$ as
\begin{equation}
\Gamma^7 = \Gamma^0 \Gamma^1 \cdots \Gamma^5
\end{equation}
such that \( (\Gamma^7)^2=\1\),
by means of which we can construct the chiral projectors
\begin{equation}
P_{\pm} = \frac{1}{2} \left( \1 \pm \Gamma^7  \right).
\end{equation}
As in four dimensional spacetime these are good definitions because $\Gamma^7$ anticommutes with all other $\Gamma^M$ matrices.

As it is well known, one can choose different representations for the $\Gamma^M$ matrices; since $\Gamma^{M*}$ still belongs to the same Clifford algebra (being the metric real), it always exists a matrix $B$ realising the change of basis, that is
\begin{equation}
\Gamma^M = -  B^{-1} \Gamma^{M *}  B
\end{equation}
with $ B$ such that
%\begin{equation}
\(  B^*  B = -\1 \).
%\end{equation}
In six spacetime dimensions the complex conjugation does not mix the two chiral projections; in other words, spinors do not change chirality under complex conjugation, nor with any other covariant operation. This is an important difference with the $d=4$ spinor representations.

All these considerations imply that the spinor representation is completely reducible, since chiral constraints can be applied covariantly, but no Majorana condition can be imposed, since it would require $B^*B = \1$. Dirac spinors have $8$ complex components; Weyl spinors have $4$ complex components.
From now on, we only consider Weyl spinors $\Psi^a$.



It is convenient to consider, instead of one chiral spinor, a couple of spinors $\Psi_i$ with $i=1,2$ and  $\GroupName{SU}(2)$ index, with the additional pseudo-Majorana %%\footnote{The name `pseudo' comes from the fact that the antisymemtric $\varepsilon$ is used; the conventional Majorana condition $\psi^* = B \psi$ is inconsistent since $B^2=-\1$.}
   constraint
%%\begin{equation}
%%( \Psi^i_a )^*  := \bar \Psi_{\dot a i} 
%%= \varepsilon_{ij} B\indices{_{\dot a}^b} \Psi^j_b.
%%\end{equation}
\begin{equation}\label{pMW-condition}
	\overline{\Psi^a_i}
		:=
	- C^a_{\dot{b}} (\Psi^b_i)^*
		=
	\Psi^{ai},
%
\qquad\qquad
%
	C^a_{\dot{b}}
		=
	\begin{pmatrix}
		0 & \mathbbm{1} \\
		-\mathbbm{1} & 0
	\end{pmatrix};
\end{equation}where $*$ is the complex conjugation and $C^*C=-\1$ ($C$ it is the restriction of $B$ to Weyl spinors), implying  that \( \overline{\overline{\Psi}^a } = - \Psi^a \) for any spinor $\Psi^a$.


This arises from two motivations. First, it allows us not to consider dotted indices, making explicit the fact that the two chiralities are not mixed under conjugation; second, after dimensional reduction it becomes the $R$-symmetry of the $N=2$, $d=4$ superPoincar\`e algebra. For this reason we will refer also to the $\GroupName{SU}(2)$ symmetry introduced in \eqref{pMW-condition} as $R$-symmetry. Notice that the pseudo-Majorana condition does not modify the number of degrees of freedom; that is, a pseudo-Majorana-Weyl spinor still contains, off-shell, 4 complex degrees of freedom.




A convenient realization of $\Gamma$ matrices is the chiral\footnote{The chiral character of this representation can be seen realising that \( \Gamma_1 \sim \sigma^1 \otimes \ldots \), \(\Gamma_I \sim \sigma^2 \otimes \ldots \) and therefore \(  \Gamma_7 \sim \sigma^1 (\sigma^2)^5 \otimes \ldots \sim \sigma^3 \otimes \ldots \).   } one
\begin{equation}\label{conjug-gamma-6d}
(\Gamma_M)_{AB} =
	\begin{pmatrix}
	0 & (\Sigma_M)_{a  b}  \\
	(\tilde \Sigma_M)^{ a b} & 0 
	\end{pmatrix},
\end{equation}
where
\begin{equation}
(\Sigma_M)_{a  b} = (\1, \Sigma^I)
\qquad\qquad
(\tilde \Sigma_M)^{ a b} = (\1, - \Sigma^I)
\end{equation}
with  $\Sigma^M_{ab}$ are the 6d antisymmetric Weyl matrices that, splitting the indices as $a \sim (\alpha,\dot{\alpha})$, read
\begin{equation*}
	\Sigma^\mu_{ab}
			=
		\begin{pmatrix}
			0 & \sigma^\mu_{\alpha  \dot{\beta} } \\
			-\bar \sigma^\mu_{\dot \alpha \beta} & 0
		\end{pmatrix},
%
\qquad
%
	\Sigma^4_{ab}
		=
	\begin{pmatrix}
		-i\varepsilon_{\alpha \beta} & 0 \\
		0 & +i \varepsilon_{\dot \alpha \dot \beta}
	\end{pmatrix},
%
\qquad
%
	\Sigma^5_{ab}
		=
	\begin{pmatrix}
		-\varepsilon_{\alpha \beta} & 0 \\
		0 & -\varepsilon_{\dot \alpha \dot \beta}
	\end{pmatrix};
\end{equation*}
the usual hermitian matrices are obtained contracting \( \Sigma^M_{ab} \) with \( C^b_{\dot{b}} \):
\begin{equation}
(\Sigma_M)_{ab} = -  C\indices{^{\dot b}_b} (\Sigma_M)_{a \dot b} .
\end{equation}

Then,
\begin{equation}
\tilde \Sigma^{M\, ab} = \frac{1}{2} \varepsilon^{abcd} \Sigma^M_{cd},
\end{equation}  where \(\varepsilon^{1234}=\varepsilon_{1234}=+1\), so that
\begin{equation*}
	\tilde \Sigma^{\mu \; ab}
			=
		\begin{pmatrix}
			0 & - \sigma^{\mu \; \alpha  \dot{\beta} } \\
			\bar \sigma^{ \mu \; \dot \alpha \beta} & 0
		\end{pmatrix},
%
\qquad\qquad
%
	\tilde \Sigma^{4 \; ab}
		=
	\begin{pmatrix}
		-i\varepsilon^{\alpha \beta} & 0 \\
		0 & i \varepsilon^{\dot \alpha \dot \beta}
	\end{pmatrix},
%
\qquad\qquad
%
	\tilde \Sigma^{5 \; ab}
		=
	\begin{pmatrix}
		\varepsilon^{\alpha \beta} & 0 \\
		0 & \varepsilon^{\dot \alpha \dot \beta}
	\end{pmatrix}.
\end{equation*}



The generators of the Lorentz group in the $(1,0)$ spinor representation are 
\begin{equation*}
	\left( \Sigma^{MN}\right)^b_a =
	\left( \tilde \Sigma^{[M} \Sigma^{N]} \right)^b_a =
	\frac{1}{2} \left( \tilde \Sigma^{M} \Sigma^{N}
		- \tilde \Sigma^{N} \Sigma^{M}  \right)^b_a
\end{equation*}
\ie\hspace{-0.5em}\footnote{In \( \Sigma\indices{^{\mu\nu \;} ^{\alpha}  _{\beta} } \) the extra minus sign is due to the conventions for contracting spinor indices in $\sigma^{\mu\nu}$.}
\begin{align*}
	\Sigma\indices{^{\mu\nu \; }^a_b}
			=
		\begin{pmatrix}
			- \sigma\indices{^{\mu\nu \;} ^\alpha  _\beta } & 0 \\
			0 & \bar\sigma\indices{^{\mu\nu \;} ^{\dot \alpha} _{\dot\beta}} 
		\end{pmatrix},
%
& \qquad
%
	\Sigma\indices{^{\mu 4 \; }^a_b}
		=
	\begin{pmatrix}
		0 & i \sigma\indices{^{\mu\; } ^\alpha _{\dot \beta}} \\
		i \bar \sigma\indices{^{\mu\; } ^{\dot \alpha} _{\beta}} & 0
	\end{pmatrix},
\\
%
	\Sigma\indices{^{\mu 5 \; }^a_b}
		=
	\begin{pmatrix}
		0 & - \sigma\indices{^{\mu\; } ^\alpha _{\dot \beta}} \\
		\bar \sigma\indices{^{\mu\; } ^{\dot \alpha} _{\beta}} & 0
	\end{pmatrix},
%
& \qquad
%
	\Sigma\indices{^{4 5 \; }^a_b}
		=
	\begin{pmatrix}
		i \delta^\alpha_\beta & 0 \\
		0 & -i \delta^{\dot \alpha}_{\dot \beta} 
	\end{pmatrix}.
\end{align*}




We conclude this review of spinors in six spacetime dimensions by underlying that the nontrivial aspects of this spinor algebra are the the fact that in the right-hand-side of \eqref{conjug-gamma-6d} there is a minus sign (instead of a plus) and the fact that $B^*B = -\1$ instead of $\1$. Such signs are determined by the signature of the metric, and they can be proven to be the only consistent combination in $5+1$ dimension. We could have as well  obtained these results by studying the representations of the universal covering group of $\GroupName{SO}(1,5)$, that is \(\GroupName{Spin}(1,5) \approx \GroupName{SL}(2,\mathbb{H}) \approx \GroupName{SU*}(4) \),\footnote{The notation $\GroupName{SU*}(4)$ identifies the subgroup of $\GroupName{SU}(4)$ to which \( \GroupName{SL}(2,\mathbb{H}) \) is injected through the conventional representation of quaternions via \( \GroupName{SU}(2) \) matrices.} that is equivalent to the approach outlined here.





%%
%%The universal covering of \(\GroupName{SO}(1,5)\) is \(\GroupName{Spin}(1,5) \approx \GroupName{SL}(2,\mathbb{H}) \approx \GroupName{SU*}(4) \); left handed Weyl spinors have one index \( \Psi^a \), \(a=1,2,3,4\); since the conjugate of the fundamental representation is equivalent to the representation we started with, it is convenient to introduce a couple of spinors \(\Psi_i^a\), being $i$ an \(\GroupName{SU}(2)\) index, obeying the pseudoreality (\ie (pseudo-)Majorana-Weyl) condition like


\subsection{Six-dimensional supersymmetry and harmonic superspace}

The six-dimensional supersymmetry algebra can be obtained as the immediate extension of the four-dimensional algebra \eqref{susy1}--\eqref{pedissequo}, by extending the Lorentz indices and imposing the pseudo-Majorana condition \eqref{pMW-condition}.
The commutator of supersymmetry generators then reads, following \cite[p.\ 32]{Galperin:book} and \cite{Howe:1983fr}
\begin{equation}\label{6d-susy-alg}
	\{Q_{a}^i, Q_{b}^j\} = 2 \varepsilon^{ij} \Sigma^M_{ab} P_M,
%
%\qquad
%
%	M = 0,1,2,3,4,5
\end{equation}

In full analogy with the four dimensional case, one can study the representations of the six-dimensional supersymmetry algebra. Since we are employing an explicit $\GroupName{SL}(2,\mathbb{C})\oplus \GroupName{SL}(2,\mathbb{C})$ notation, we will not go through all the derivation again, since the steps are basically the same up to a doubling of the indices. We briefly mention that, since we are studying massless representations, we can diagonalise the momentum operator to $P_M = (-E, 0, 0, E, 0, 0)$ and from this  follows that only one supercharge is nontrivially realised, and raises the helicity of $1/2$. The vector multiplet therefore is made of the gauge boson and the massless spinor, with both helicities because of CPT invariance. Notice that, off-shell, the spinor carries $8$ real degrees of freedom, while the vector boson only $6-1=5$ (one is killed by gauge invariance), therefore we expect $3$ bosonic real auxiliary fields to appear.




The algebra \eqref{6d-susy-alg} can be realised in the conventional superspace, called the real superspace \( \mathbb{R}^{1,5|1} \), where points are described with coordinates
\begin{equation}
\left( x^M, \theta^a_i \right) 
\end{equation} 
 where $x^M$ are (commuting) spacetime coordinates and $\theta^a_i$ are complex Grassmann numbers satisfying the pseudo-Majorana condition \eqref{pMW-condition}.
In this superspace the supersymmetry algebra is realised through the transformations parametrised by the pseudo-Majorana-Weyl spinor $\xi$
\begin{equation}
\delta x^M 
= i \left(
\xi^{ia} \, \Sigma^M_{ab} \, \theta^b_i
- \theta^{ia} \, \Sigma^M_{ab} \, \xi^b_i
\right),
\hspace{3em}
\delta\theta^a_i =  \xi^a_i. 
\end{equation}
In the real superspace we then introduce the spinor covariant derivatives
\begin{equation}\label{spinor-covD-usual}
\covDs_a^k  = \partial_a^k - i \theta^{bk} \partial_{ab}
\end{equation}
where the partial derivatives are defined as
\begin{equation}
\partial_{ab} 
= \frac{1}{2} \Sigma^M_{ab} \partial_M 
%
\qquad\qquad
%
\partial_a^k \, \theta^b_i = \delta^k_i \, \delta^b_a.
\end{equation}
We also introduce the notation, valid in general for vector indices,
\begin{equation}
x^M = \frac{1}{2}\Sigma^M_{ab} x^{ab}.
\end{equation}
The spinor derivatives satisfy the algebra
\begin{equation}
\left\{
\covDs^k_a   ,  \covDs^j_b
\right\}
	=
- 2 i \varepsilon^{kj} \partial_{ab}.
\end{equation}
The covariant derivatives (anti)commute with supersymmetry generators and therefore are used to construct objects that transform covariantly under supersymmetry. However this is of little help in this case, in contrast with the unextended supersymmetry in four dimensions, because it is not possible to formulate off-shell supersymmetric theories in six dimensional spacetime in this superspace, as follows from a degree-of-freedom counting argument as discussed in \cite{Howe:1983fr,Galperin:book}.
This no-go theorem drove physicists to look for other kind of superspaces. As we will see, the loophole to this result is to find a superspace that allows for the introduction of an infinite number of auxiliary fields.


There are also other choices of coordinates that one can make. In particular we will deal with the \emph{harmonic} coordinates, that we are now going to construct. For a complete discussion of the construction, of the properties and the superfieds that can be introduced in four and six dimension the reader should check \cite{Galperin:1984av, Galperin:book, Howe:1985ar}.


Let us introduce the bosonic variables $u^\pm_i$, $i$ being a $\GroupName{SU}(2)_R$ index, such that $u^-_i = (u^{+i})^*$ and
\begin{equation}\label{u-defining-property-1}
u^{+i} u^-_i = 1,
\end{equation}
\begin{equation}\label{u-defining-property-2}
u^{+i} u^+_i = 0 = u^{-i} u^-_i.
\end{equation}

They are charged with respect to a $\GroupName{U}(1)_R$ subgroup of $\GroupName{SU}(2)_R$ with charges $\pm 1$ according to the notation.
These properties allow us to define, for any spinor $\Psi^i_a$, its $\GroupName{U}(1)_R$ projections $\Psi^{\pm} = \Psi^i u^\pm_i$ and to decompose it according to
\begin{equation}
\psi^i = u^{+i} \Psi^- - u^{-i} \Psi^+.
\end{equation}

The harmonic variables effectively correspond to assigning an  \( \GroupName{SU}(2) \) matrix
\begin{equation}\label{matrix-u+u-}
\begin{pmatrix}
u^+_1 	&	u^-_1	\\
u_2^+	&	u^-_2	
\end{pmatrix}
\end{equation}
where the requirement of unit determinant is exactly the condition \eqref{u-defining-property-1}.
The action of $\GroupName{U}(1)$ can be then rewritten as
\begin{equation}\label{matrix-u+u--U1action}
\begin{pmatrix}
u^+_1 e^{i\psi}	&	u^-_1 e^{- i\psi}	\\
u_2^+ e^{i\psi}	&	u^-_2 e^{- i\psi}	
\end{pmatrix}
=
\begin{pmatrix}
u^+_1 	&	u^-_1	\\
u_2^+	&	u^-_2	
\end{pmatrix}
e^{i \psi \sigma^3}
\end{equation}
where $\sigma^3$ is the third Pauli matrix and \( e^{i \psi \sigma^3} \in \GroupName{U}(1)_R \). Not all $\GroupName{SU}(2)$ matrices can be represented as \eqref{matrix-u+u-}, while \eqref{matrix-u+u--U1action} parametrises the whole group; %%we require now the action to be manifestly invariant under $\GroupName{U}(1)$ transformations only, so that
 we are effectively considering the coset 
 $\GroupName{SU}(2)_R / \GroupName{U}(1)_R$, that is diffeomorphic to the two sphere $\mathbb{S}^2$. This also proves that $\GroupName{U}(1)_R$ is the only component of $ \GroupName{SU}(2)_R $ linearly realised, as a consequence of coset theory.
 

We also introduce the following integration rules for the harmonic variables
\begin{equation}\label{integratioon-harmonics}
\int \dd{u} 1 = 1,
\hspace{3em}
\int \dd{u} u^+_{(i_1} \cdots u^+_{i_n} u^-_{j_1} \cdots u^-_{j_m)} = 0,
\end{equation}
with the second one holding when $n + m \neq 0 $.

There are different kind of conjugations that we can define over the harmonic variables. We already considered complex conjugation $*$, that affects both the $\GroupName{SU}(2)_R$ and the $\GroupName{U}(1)_R$ indices; we now define the conjugation $\overline{\phantom{x}}$, that affects only the latter:
\begin{equation}
\overline{u^{+i}} = u^{-i}
\qquad
\overline{u^{-}_{i}} = -u^+_{i}.
\end{equation}
This conjugation preserves the defining properties of \eqref{u-defining-property-1} and  \eqref{u-defining-property-2} and squares to minus the identity.
Then, we can define the conjugation $\sim$, that is the composition of $*$ and $\overline{\phantom{x}}$; on scalars and Grassmann numbers it is the complex conjugation, whereas on the harmonic variables it takes the value
\begin{equation}
\widetilde{u^{\pm}_{i}} = u^{\pm i}
;
\end{equation}
it therefore squares to the identity for complex and Grassmann numbers, and to minus the identity for harmonic variables.
This implies that for an harmonic function of charge $q$, 
\begin{equation}
\widetilde{\widetilde{ f^{(q)}}} = (-)^q f^{(q)}.
\end{equation}
In the particular case of even $q=2k$, we have that we can impose the condition
\begin{equation}
\widetilde{f^{2k}} = f^{(2k)}
\end{equation}
or equivalently
\begin{equation}
\overline{f^{i_1 \cdots i_{2k}}} = \varepsilon_{i_1 j_1} \cdots \varepsilon_{i_{2k} j_{2k}} f^{{i_1 \cdots i_{2k}}}.
\end{equation}


We are now ready to introduce the \emph{real harmonic superspace}  $\HR$, that is the space described with the set of coordinates
\begin{equation}
(z,u) = \left( x^M, \theta^a_i, u^{\pm i} \right).
\end{equation}
This superspace is therefore diffeomorphic to $\mathbb{R}^{1,5|1} \times \mathbb{S}^2$, where only part of the $R$-symmetry (the $\GroupName{U}(1)_R$ subgroup) is now linearly realised on the coordinates, while in the conventional superspace the whole $\GroupName{SU}(2)_R$ is linearly realised. In the superspace  $\HR$ we still have the spinor covariant derivatives \eqref{spinor-covD-usual}; in addition we can introduce derivatives (automatically covariant, since the harmonics do not have spinor indices) for the harmonic variables,
\begin{equation}
%%D^{++} = 
\partial^{++} = u^{+i} \frac{\partial}{\partial u^{-i}},
\hspace{2em}
%%D^{--} = 
\partial^{--} = u^{-i} \frac{\partial }{\partial u^{+i}},
\hspace{2em}
D^0 = u^{+i} \frac{\partial}{\partial u^{+i}} - u^{-i} \frac{\partial}{\partial u^{-i}}.
\end{equation}
Notice that a superfield of definite $\GroupName{U}(1)$ charge is eigenvector of $D^0$ with eigenvalue the charge.

An alternative coordinate system on this superspace is represented by the `analytic' coordinates,
\begin{equation}
( Z_A , u) = (x^M_A, \theta^{\pm a}, u^\pm_i) 
\end{equation}
where
\begin{equation}
\theta^{\pm }_a = \theta^i_a u_i^\pm
\qquad
x^M_A = x^M + \frac{i}{2} \theta^a_k\, \Sigma^M_{ab} \, \theta^b_l \, u^{+(k|} u^{-|l)}.
\end{equation}






Another superspace that can be now introduced, whose importance will be clear soon, is the \emph{Grassmann-analytic} (G-analytic for short) \emph{superspace} $\HR^+$, that is spanned by the coordinates
\begin{equation}
(\zeta , u ) = \left( x^M_A, \theta^{+ a},  u^\pm_i \right) ,
\end{equation}
namely we dropped dependence on the coordinate $\theta^-$.
This new superspace turns out to be very useful because the restriction to it is supersimmetric invariant, as it happens in four spacetime dimensions with the chiral superspace. This can be seen by noticing that the supersymmetry transformations can be realised as 
\begin{equation}
\delta x^M_A
= -2 i \left(
\xi^{ia} \, \Sigma^M_{ab} \, \theta^{+b}
- \theta^{+a} \, \Sigma^M_{ab} \, \xi^{ib}
\right)u^-_i,
\hspace{1.5em}
\delta\theta^a_i =  \xi^{ia} u^+_i,
\hspace{1.5em}
\delta u^\pm_i=  0
,
\end{equation}
where again $\xi$ is the parameter of the supersymmetry transformation. It is clear that such transformation leave the G-analytic subspace invariant.


%
%The six-dimensional supersymemtry algebra can be realised in the conventional suerspace in a way fully similar to the four dimensional algebra.

%%The standard superspace approach for the six dimensonal supersymmetry REF employs coordinates \( (x^M, \theta^a_i) \) where $x^M$ are (commuting) spacetime coordinates and $\theta^a_i$ are complex Grassmann numbers satisfying the pseudo-Majorana condition REF.
%%%
%%%\footnote{This space is just the coset
%%%\begin{equation}
%%%\mathbb{R}^{6|1}
%%%=
%%%\frac{\text{Poincar\'e} \times }{\text{Lorentz} \times \GroupName{SU}(2)_R}.
%%%\end{equation}
%%%
%%%
%%%The structure of the algebra REF is complex enough to give us also different choices on which subgroup quotient. Another option is to take the quotient only with respect to a subgroup $\GroupName{U}(1)_R$ of $\GroupName{SU}(2)_R$. We just introduced the harmonic superspace
%%%\begin{equation}
%%%\HR 
%%%=
%%%\frac{\text{Poincar\'e} \times}{ \text{Lorentz} \times \GroupName{U}(1) }
%%%\end{equation}
%%%The coset \( \GroupName{SU}(2) / \GroupName{U}(1) \) is, as known, diffeomorphic to the two-sphere $ \mathbb{S}^2 $, so that the $\HR$ can be effectively represented as \( \mathbb{R}^{6|1N} \times \mathbb{S}^2 \). An effective way to parametrize the sphere is by using spherical harmonics $u^{\pm i}$. They are defined through the stereographic projection from north pole $N$, 
%%%\begin{equation}
%%%\begin{split}
%%%\pi_N \colon  \mathbb{S}^2  \setminus \{ N \} & \rightarrow \mathbb{C} \\
%%% y \qquad & \mapsto t
%%%\end{split}
%%%\end{equation}
%%%that allow for the definition of an $\GroupName{SU}(2)$ matrix whose components are the harmonics themselves:
%%%\begin{equation}
%%%\begin{pmatrix}
%%%u^+_1 	&	u^-_1	\\
%%%u_2^+	&	u^-_2	
%%%\end{pmatrix}
%%%	=
%%%\frac{1}{\sqrt{1 + t t^*}}
%%%\begin{pmatrix}
%%%1 	&	-t^*	\\
%%%t	&	1	
%%%\end{pmatrix}.
%%%\end{equation}
%%%In general an $\GroupName{SU}(2)$ matrix can be uniquely written as
%%%\begin{equation}
%%%\GroupName{SU}(2)
%%%	\ni
%%%\begin{pmatrix}
%%%\alpha 	&	-\beta^*	\\
%%%\beta	&	\alpha^*	
%%%\end{pmatrix}
%%%	\equiv
%%%\frac{1}{\sqrt{1 + t t^*}}
%%%\begin{pmatrix}
%%%e^{i\psi} 	&	-t^*e^{-i\psi}	\\
%%%te^{i\psi}	&	e^{-i\psi}	
%%%\end{pmatrix}
%%%\end{equation}
%%%with $\alpha \alpha^* + \beta \beta^* = 1$; the angle $\psi$ corresponds to the $\GroupName{U}(1)$ to be quotiented out.
%%%%%The indices of the harmonics are such that
%%%%%\begin{equation}
%%%%%	u^{+i} u^-_i 
%%%%%\equiv
%%%%%	\varepsilon^{ik} u^+_i u^-_k 
%%%%%%%\equiv
%%%%%%%	
%%%%%=
%%%%%	1,
%%%%%\qquad
%%%%%(u_i^+)^* = u^{i-},
%%%%%\qquad
%%%%%(u_i^-)^* = - u^{i+}.
%%%%%\end{equation}
%%%%%and the $\pm$ determines the representation ($\pm 1$) of $\GroupName{U}(1)$ under which they transform. 
%%%
%%%
%%%
%%%%%
%%%%%The harmonic superspace $\HR$ is then parametrized with coordinates
%%%%%\begin{equation}
%%%%%(z,u) = (x^M, \theta^a_i, u^{\pm i}).
%%%%%\end{equation}
%%%%%
%%%%%
%%%%%Let us now define the so-called analytic coordinates $Z_A = (x_A^M,\theta^{\pm a})$ as
%%%%%\begin{equation}
%%%%%	x^M_a
%%%%%=
%%%%%	x^m + \frac{i}{2} \theta^a_k \Sigma^M_{ab} \theta^b_l u^{+k} u^{-l},
%%%%%\qquad
%%%%%	\theta^{\pm a} = u^{\pm}_k \theta^{ak}.
%%%%%\end{equation}
%%%}

With respect to analytic coordinates, the covariant derivatives in the analytic basis become
\begin{align}
%\begin{split}
\label{spinor-covD-6-analytic-basis}
&	\covDs^+_a 	
% = u^+_i \covDs^i_a
 =
	\partial_{-a},
\\
%
\qquad
&	\covDs^{-}_a 	
= 
	- \partial_{+a} - 2 i \theta^{-k} \partial_{ab},
\\
%
\qquad
&	\covDs^0	
= 
	u^{+i} \frac{\partial}{\partial u^{+i}} 
	- u^{-i} \frac{\partial}{\partial u^{-i}}
	+ \theta^{+a} \partial_{+a} 
	- \theta^{-a} \partial_{-a},
%
\\
&	\covDs^{++}	
=
	\partial^{++} 
	+ i \theta^{+a} \theta^{+b} \partial_{ab} 
	+ \theta^{+a} \partial_{-a},
\\
%
\qquad
&	\covDs^{--}
=
	\partial^{--} 
	+ i \theta^{-a} \theta^{-b} \partial_{ab} 
	+ \theta^{-a} \partial_{+a},
%%\end{split}
\end{align}
where
%\begin{equation}
\( \partial_{\pm a} \theta^{\pm b} = \delta^b_a. \)
%\end{equation}
The following commutation relations hold:
\begin{equation}
\left\{ \covDs^+_a , \covDs^-_b \right\} = 2i \partial_{ab},
\qquad
[ \covDs^{++} , \covDs^{--} ] = \covDs^0,
\end{equation}
\begin{equation}\label{casa}
[\covDs^{--}, \covDs^-_a] = \covDs^+_a,
\qquad
[ \covDs^{--} , \covDs^+_a ] = \covDs^-_a,
\end{equation}
\begin{equation}\label{covDs++-commutator-1}
[\covDs^{++} , \covDs^+_a ] = 0 =  [ \covDs^{--} , \covDs^-_a ],
\end{equation}
as one can verify by direct computation.
Notice, however, that in the usual (non-analytic) $x^M$ coordinate we can write, with abuse of notation, $\covDs^{\pm}_a = u^{\pm}_i \covDs^i_a$.


We also introduce the notation
\begin{equation}
\left( \covDs^\pm \right)^4
	=
- \frac{1}{24}
	\varepsilon^{abcd} 
	\covDs_a^\pm \covDs_b^\pm \covDs_c^\pm \covDs_d^\pm
\end{equation}
and the integration measure
\begin{equation}
d \zeta^{(-4)} = d^6 x_A \left( \covDs^- \right)^4.
\end{equation}
Notice that with these definitions the only nonvanishing superspace integral is
\begin{equation}\label{integr-zeta(-4)}
\int d \zeta^{(-4)} (\theta^+)^4 = \int d^6 x_A .
\end{equation}


Consider now a superfield $\Phi$.
A first constraint that we can impose  is Grassmann-analiticity, that is said to be satisfied  if a field depends only on the variables
\(
\left( \zeta, u \right)
	=
\left( x^M_A, \theta^{+a}, u^{\pm i} \right),
\)
\ie it is independent of $\theta^{-a}$.  Notice that the constraint can be written in differential form as
\begin{equation}
\covDs^+_a \Phi = 0
\end{equation}
by virtue of \eqref{spinor-covD-6-analytic-basis}. This is analogous to the chiral constraint in the four dimensional superspace.

Since $\GroupName{U}(1)_R$ is linearly realised, superfields transform in representations of it, and this fact allows us to classify them according to their charge. Any covariant function $f^{(q)}$ with charge $q>0$ can then be decomposed in series of harmonics as
\begin{equation}
f^{(q)} = \sum_n^\infty f^{(i_1 \ldots i_{n+q} j_1 \ldots j_n )}
u^+_{i_1} \cdots u^+_{i_{n+q}} u^-_{i_1} \cdots  u^-_{i_n},
\end{equation}
an analogous decomposition holding for $q < 0$. Every $f^{\cdots}$ transforms in an irreducible representation of $\GroupName{SU}(2)_R$. Therefore, given a superfield of definite $\GroupName{U}(1)$ charge, it can be expanded according to the previous formula; this allows for the infinite number of auxiliary fields that we mentioned before.

We have now introduced all the superspace technology that we need in order to work in six dimensional supersymmetry.


\subsection{Gauge theory in harmonic superspace}

\subsubsection{General framework}

We want to consider fields that are not necessarily G-analytic, that is they are functions $\Phi$ of all the real superspace coordinates $(x^M,\theta^a_i)$. Notice that also this subspace is closed under the supersymmetry transformations considered above.  Consider a symmetry transformation of the form
\begin{equation}
\Phi'(x,\theta^i_a) =  e^{i \tau} \Phi(x,\theta^i_a),
\hspace{3em}
\tau = \tau^k \, T^k
\end{equation} 
with $\tau$ real and 
being $T^k$ the generators of the representation of the symmetry group $\GroupName{G}$ under which $\Phi$ transforms. We want to gauge such symmetry transformation allowing for $\tau = \tau(x,\theta^i_a)$, by consistency independent of the harmonic variables $u$, that is 
\begin{equation}
\partial^{++} \tau = 0 ,
\qquad\qquad
\partial^{--} \tau = 0 .
\end{equation} 


Let us start gauging the real-superspace derivatives $\covDs_{a}^i$ and $\partial_{ab}$ by introducing a gauge connection $A(x,\theta^a_i)$
\begin{align}
\covD^i_a = \covDs_{a}^i + A_{a}^i(x,\theta^i_a),
\qquad\quad
\covD_{ab} = \partial_{ab} + A_{ab}(x,\theta^i_a).
\end{align}
The field strength tensor in superspace is then constructed by taking (anti)commutators of such covariant derivatives
\begin{equation}
F_{IJ} = \left[ \covD_I, \covD_J \right\} - t\indices{_I_J^K} \covD_K,
\hspace{3em}
I,J,K = \left( ab \sim M,\ ^i_a \right)
\end{equation}
where $t$ is the  torsion coming from the commutators of the superspace (non-gauged) covariant derivative \eqref{covDs++-commutator-1}. The relevant components read
\begin{align}
\label{6d-gauge-F-with-torsion}
\left\{ \covD^i_a , \covD^j_b  \right\}
	& =
F^{ij}_{ab} 
+ i \varepsilon^{ij} 
	\delta^c_{[a} \delta^d_{b]} \covD^{}_{cd},
\\
\left\{ \covD_{ab}  , \covD_{cd}   \right\}
	& =
F_{ a b \, c d },
\\
\left[ \covD^i_a   ,  \covD^{}_{ b c } \right]
	& =
F^i_{ a \, b c }.
\end{align}

The field strength tensor obeys a number of Bianchi identities, but the representation still contains a big number of fields, as usual in such approach in superspace. Constraints must be imposed on the field strength in order to extract the desired multiplet.

The constraints to impose to describe the six dimensional gauge theory have been studied in the literature, see for instance \cite{Nilsson:1980nz, Koller:1982cs}, and read
\begin{equation}
\label{6d-gauge-constr1}
F_{ab}^{ij} = 0;
\end{equation}
even though we will not work in this superspace we solve the constraint to show the ideas that will be employed in the harmonic superspace.
The constraint is solved by
\begin{equation}
\label{6d-gauge-Aialpha}
A^i_a
	=
e^{-v} \covDs^i_a e^{v}
\end{equation}
for some real superfield $v=v(x^M,\theta^i_a) = v^k \, T^k$.
Notice  that $v$ %%in \eqref{6d-gauge-Aialpha} 
is not uniquely determined: The substitution $e^v \rightarrow e^{\kappa} e^{v}$ yields the same $A^i_a$ as long as $\covDs_a^i \kappa = 0 $; by consistency with the initial restriction to the subspace $\mathbb{R}^{1,5|1}$, this means that $\kappa$ is real and depends only on $x$.
Then, requiring covariance with respect to gauge transformations parametrized by $\tau$, one has the full gauge transformations
\begin{align}
e^{v'}
	& =
e^{\kappa} e^{v}e^{- \tau}
\\
{A_a^i} '
	& =
e^{\tau} A_a^i e^{-\tau} 
- 
( \covDs_a^i e^{\tau}) e^{-\tau},
\end{align}
where to verify the latter it might be useful to recall that $ \covDs%_\alpha^i 
e^{-\tau} = - e^{-\tau} \left( \covDs%_\alpha^i 
e^\tau \right) e^{-\tau} $.




\subsubsection{Harmonic superspace}




We are particularly interested in G-analytic superfields. The fundamental object that contains the degrees of freedom of the gauge theory is the connection connection $V^{++}$ for the derivative $\covDs^{++}$, so that
\begin{equation}\label{covds++}
\covD^{++} = \covDs^{++} + V^{++} 
\end{equation}
transforms covariantly under gauge transformations. This is achieved by requiring
\begin{equation}\label{v++-gage-intiial}
\left( V^{++} \right)' = e^{\lambda} \covDs^{++} e^{-\lambda} + e^{\lambda} V^{++} e^{-\lambda}
\end{equation}
begin $\lambda$ the gauge transformation parameter. We shall require that $\lambda$ is a $\sim$-even Grassmann-analytic superfield, that is $\tilde{\lambda}=\lambda$ and $\covDs^+_\alpha \lambda = 0$.

To understand this we would need to introduce matter multiplet and discuss gauging of internal symmetries. We will not go into this in detail since it would require a discussion much broader than the aim of this work, and we will just sketch here the main results. Indeed the matter multiplet can be embedded into a complex G-analytic superfield $\phi^{(1)}(\zeta,u)$ with $\GroupName{U}(1)$ charge $1$ whose action reads
\begin{equation}
S = \int \dd{u} \dd{\zeta^{(-4)}} {\tilde\phi}^{(1),\ell} \covDs^{++} \phi^{(1),\ell},
\end{equation}
being $\ell$ and internal index.
Gauging the invariance under phase transformations $\phi' = e^{ \lambda} \phi$, with $\sim$-even $\lambda = \lambda^k T^k $ being $T^k$ the generators $ \GroupName{G}$, this action is no longer invariant; introducing the $V^{++}$ connection satisfying \eqref{covds++} we can recover it. The G-analyticity of $\lambda$ comes requiring the consistency of the G-analyticity of $\phi^{(1)}$ under gauge transformations, that is a reasonable request if we recall that supersymmetry preserves it too. Notice that in the absence of \ym{} field, the derivative $\covDs^{++}$ preserves G-analyticity thanks to the commutators \eqref{covDs++-commutator-1}.
We therefore understand that all what is needed to encode \ym{} degrees of freedom is the connection $V^{++}$, that we are going to study in some greater detail.


Contracting  \eqref{6d-gauge-constr1} with the harmonics $u^+_i$ and $u^+_j$ we can rewrite it in the form
\begin{equation}
\left\{ 
\covD^+_a , \covD^+_b
\right\}
= 0
\end{equation}
whose solution reads
\begin{equation}
\label{6d-gauge-A+alpha}
A^+_a = e^{-b} \covDs^+_a e^b
\end{equation}
being now $b(x,\theta,u)$ dependent also on $u^\pm_i$ and transforming according to
\begin{equation}
e^{b'}
	=
e^{ \lambda} e^b e^{-\tau}
\end{equation}
where $\tau$ is again a generic superfield independent of $u$ while \(\lambda\) is now G-analytic, namely $\covDs^+_a \lambda =0 $.
Since the covariant derivative and $\lambda$ are $\sim$-even, we also require that $\widetilde{ b }= b$.



The meaning of the superfield $b$ can be understood as follows. A covariantly G-analytic superfield in the real superspace \( \Phi(x,\theta^i,u) \) (so that $\covD^+_a \Phi = 0$) undergoes the gauge transformation \( \Phi \rightarrow e^{\tau} \Phi  \); the field 
\begin{equation}\label{phi-Phi}
\phi := e^b \Phi,
\end{equation}
on the other hand, is G-analytic because
\begin{equation}
0 = \covD^+_a \Phi = e^{-b} \covDs^+_a \phi
\end{equation} 
 and transforms with $\lambda$ only:
\begin{equation}
\phi' =
	e^{b'}\Phi' =  
e^{ \lambda} e^b e^{-\tau} e^{\tau} \Phi = e^{\lambda} \phi,
\end{equation}
that also preserves G-analyticity of $\phi$.


In the analytic superspace we have also two other  derivatives: $\covDs^{++}$ and $\covDs^{--}$. Since we have more derivatives to make gauge-covariant, correspondingly we have to impose other constraints to get rid of the extra components of the superfield strength tensor. We know that $\covDs^{++}$ respects G-analyticity, that is it commutes with \( \covDs^+_a \),  we require this to be true for the gauge-covariant derivatives too:
\begin{equation}\label{6-gauge-commut-cvd++.covd+alpha}
\left[ \covD^{++}  ,  \covD^+_a  \right]
	=
0.
\end{equation}
We require \eqref{6-gauge-commut-cvd++.covd+alpha} to be valid for both the   gauge transformations, generated by $\tau$ and $\lambda$. Considering $\phi$-type fields, in the notation of \eqref{phi-Phi}, we get covariance under ($\lambda$-)gauge transformations if
\begin{equation}\label{6d-gauge-V++-def}
V^{++} = e^{b} \covDs^{++} e^{-b},
\end{equation}
transforming as
\begin{equation}\label{6d-gauge-V++-gauge-transf}
\left( V^{++} \right)' 
	=
e^{\lambda} \covDs^{++} e^{-\lambda} + e^{\lambda} V^{++} e^{-\lambda}
	=
e^{\lambda} \covD^{++} e^{-\lambda} 
\end{equation}
that is indeed our desired result (cf.\ \eqref{v++-gage-intiial}). Notice that $V^{++}$  that is a $\sim$-even function, as consequence of such being the $\covDs^{++}$ and $b$.
We know that for the $\tau$-gauge transformations $\covDs^{++}$ does not need any connection, as a consequence of $\covDs^{++} \tau = 0$; using this fact and plugging \eqref{6d-gauge-A+alpha} into the constraint \eqref{6-gauge-commut-cvd++.covd+alpha} one finds
\begin{equation}
0 = \covDs^{++} A^+_a % = D^{++} \left(  e^{-b} \covDs^+_\alpha e^b  \right)
= - e^{-b} \left[  \covDs^+_a \left(  e^b \covDs^{++} e^{-b}   \right)  \right] e^b
\end{equation}
that by comparison with \eqref{6d-gauge-V++-def} is a G-analyticity constraint for $V^{++}$, namely
\begin{equation}
\covDs^+_a V^{++} = 0.
\end{equation}

We  have just arrived to some conclusion. Indeed, if we accept that all the physical degrees of freedom of the vector multiplet are encoded in $V^{++}$ (\ie $A_\mu$ with gauge invariance and $\Psi^i$), through \eqref{6d-gauge-V++-def} can be determined $b$ and from it one can then compute $A^+_\alpha$ using \eqref{6d-gauge-A+alpha}, obtaining therefore the covariant derivative in $\HR^+$. However, in order to reconstruct the whole $A^i_a$, we still need to find $A^-_a$.


The superfield $V^{++}$ can be used to build a \sym{} action but the procedure is quite involved. We will directly study the higher derivative case, that is remarkably simpler. In order to do that, we still have to introduce some definitions. First of all, let us promote the commutation relation \eqref{covDs++-commutator-1} to the gauge-covariant derivatives,  by requiring
\begin{equation}
\label{6d-gauge-V--V++-relation}
\left[
	\covD^{++} ,  \covD^{--}
\right]
	= \covDs^{0};
\end{equation}
notice that $\covDs^0$ is already covariant since it only gives the $\GroupName{U}(1)$-weight and we work only with fields of definite charge.
This is obviously true for the $\tau$-gauge transformations, since $\covDs^{++} \tau = 0 = \covDs^{--} \tau $; to extend it to the $\lambda$-gauge transformations we have to introduce a connection $V^{--}$ for $\covDs^{--}$. \eqref{6d-gauge-V--V++-relation} turns then to an implicit definition of $V^{--}$ in terms of $V^{++}$, since it reads
\begin{equation}\label{6d-gauge-V--V++-differential-relation}
\covDs^{++} V^{--} - \covDs^{--} V^{++} + \left[ V^{++} , V^{--} \right] = 0.
\end{equation}
Notice that we can as well express the $V^{--}$ connection in terms of the superfield $b$ as we did for $V^{++}$  
\begin{equation}\label{6d-gauge-V---def}
V^{--} = e^{b} \covDs^{--} e^{-b}
\end{equation}
from which its gauge transformation property follows
\begin{equation}\label{6d-gauge-V---gauge-transf}
\left( V^{--} \right)' 
	=
e^{\lambda} \covDs^{--} e^{-\lambda} + e^{\lambda} V^{--} e^{-\lambda}
	=
e^{\lambda} \covD^{--} e^{-\lambda}.
\end{equation}
With $V^{--}$ we can reconstruct the connection $A^-_a = - \covDs^+_a V^{--}$ imposing the anti-commutation relation
\begin{equation}
\left\{
\covD^{--}, \covD^+_a
\right\} = \covD^-_a,
\end{equation}
that is the covariant generalization of \eqref{casa}.
We have therefore expressed all the gauge structure in terms of the field $V^{++}$, as promised, since we can reconstruct the whole $A^i_a$.


Following \cite{Ivanov:2005qf}, using \( V^{--} \) one can define the covariant spinor superfield strength
\begin{equation}
W^{+\alpha}
	=
- \frac{1}{6}
\varepsilon^{ \alpha \beta \gamma \delta } 
\covDs^+_\beta \covDs^+_\gamma \covDs^+_\delta V^{--}
\end{equation}
from which we can also build
\begin{equation}
F^{++} = \frac{1}{4} \covDs^+_a W^{+a} \equiv (\covDs^+)^4 V^{--}.
\end{equation}
%%Notice that $V^{--}$ roughly contains  the fields with no derivatives; $F^{++}$ therefore contains their second derivatives.
Notice that as a consequence of the gauge transformation rule for $V^{--}$, \eqref{6d-gauge-V---gauge-transf}, and since $\lambda$ is G-analytic, $F^{++}$ transforms in the adjoint representation of the group too, \ie
\begin{equation}
(F^{++})' = e^\lambda F^{++} e^{-\lambda}.
\end{equation}




Before proposing an action, let us spend a couple of words about dimensions. The harmonics $u$ are dimensionless as a consequence of the defining  properties \eqref{u-defining-property-1} and \eqref{u-defining-property-2}; this implies that the derivatives have dimension $[\covDs^\pm_\alpha] = - [ \theta^\alpha_i ] = 1/2$ and $[\covDs^{++}] = 0 = [\covDs^{--}]$.  Then, from \eqref{6d-gauge-V++-def} and \eqref{6d-gauge-V---def} $V^{\pm\pm}$ are dimensionless too, and therefore $[F^{++}] = 2$. For what we said, follows that $[\dd{\zeta^{(-4)}}] = 4$.




A possible action, gauge and Lorentz invariant, is then
\begin{equation}\label{hd-action-superspace}
S = \frac{1}{2 f } \int \dd{\zeta^{-4}} \dd{u} 
	\Tr \left[ \left(  F^{++} \right)^2 \right]
\end{equation}
with $f$ a dimensionless constant.






\subsection{Action in terms of components fields}




The gauge symmetry transformation \eqref{6d-gauge-V++-gauge-transf} allows us to get rid of all contributions from higher harmonics, choosing the so-called Wess-Zumino gauge. The infinitesimal form of the gauge transformation reads \( \delta V^{++} = - \covD^{++} \lambda \); using it, with some work one can show that is possible to cast $V^{++}$ in the form
\begin{equation}\label{wz-gauge}
V^{++}_{\WZ}(x,\theta^+)
	=
- \theta^{ + a } \theta^{ + b } A_{a b }(x) 
+ 2 \sqrt{2} (\theta^+)^3_a \Psi^{ - a}(x)
- 3 (\theta^+)^4 D^{--}(x)
\end{equation}
where there is still the usual gauge invariance for $A_{ab}$ (\ie $A_M$), and we used the notation
\begin{align}
&\left(
	\theta^+
\right)^3_d
	=
\frac{1}{6} \varepsilon_{a b c d} 
\theta^{ + a } \theta^{ + b } \theta^{ + c },
\quad
&\left(
	\theta^+
\right)^4
	=
- \frac{1}{24} \varepsilon_{a b c d} \theta^{ + a } \theta^{ + b } \theta^{ + c } \theta^{ + \delta },
\end{align}
and the fields read
\begin{equation}
\Psi^{-a} = \Psi^{ai} u^-_i,
%
\qquad\qquad
%
D^{--} = D^{ik}u^-_i u^-_k.
\end{equation}
We will not go into this in detail, the interested reader can check \cite{Galperin:book}, but we motivate this result. Both $V^{++}$ and $\lambda$ are a analytic superfield, so they depend on the variables $(x_A^\mu, \theta^+_a, u^\pm_i)$; they therefore admit a finite expansion in powers of $\theta^+$, in which the coefficients are functions of $x$ and $u$; the $\GroupName{U}(1)$ weight of the coefficients is fixed requiring that the superfields have weight respectively $+2$ and $0$. In this way, comparing the harmonic expansions (3.2.44), one can check, explicitly at least in the Abelian case, that using $\delta V^{++} = - \covDs^{++} \lambda $ it is possible to get rid of all component but those in \eqref{wz-gauge}, fixing $\lambda$ up to the function $\lambda(x) = \lambda |_{u=0=\theta^+}$ that generates usual gauge transformation for $A_{ab}$. Indeed, it is immediate to verify that $\covDs^{++} \lambda(x) = \theta^{+a} \, \theta^{+b} \, \partial_{ab} \lambda (x).$



In this expansion the only fields that appear are the gauge field $A^M$, the gluino $\Psi^{ai}$ and the triplet of auxiliary fields $D^{ik}$, so that this is exactly an off-shell representation of the vector supermultiplet. Notice that the fermionic ($4$ complex) and bosonic ($5$ + $3$ real) degrees of freedom indeed agree, as anticipated.


In order to express the action \eqref{hd-action-superspace} in terms of components fields, we have to find the $V^{--}$ superfield associated to $V^{++}_\WZ$, act on it with $(D^+)^4$ in order to find $F^{++}$ and then perform the integration over the Grassmann and harmonic variables. The algebra is quite lengthy but straightforward; we will only outline the steps.

$V^{--}$ is not a G-analytic superfield, so we can decompose it as
\begin{equation}
V^{--} 
	=
v^{--}
+ \theta^{- b} v_{ b }^-
+ \theta^{-c} \theta^{-d} v_{cd}
+ (\theta^-)^3_d v^{+d}
+ (\theta^-)^4 v^{++}
\end{equation}
where the coefficients are functions of \((x_A, \theta^+, u)\), \ie are G-analytic;
noticing that
\begin{equation}
F^{++} =  (\covDs^+)^4 V^{--} = v^{++} 	
\end{equation}
we conclude that we only need to solve for  the lowest  weight component $v^{--}$. 
$F^{++} = v^{++}$, being  G-analytic, can be similarly expanded as
\begin{equation}\label{v++-expansio}
F^{++}
= v^{++} =
\lambda^{++}
+ \theta^{+a} \lambda^+_a
+ \theta^{+a} \theta^{+b} \lambda^+_{ab}
+ (\theta^{+})^3_a \lambda^{-a}
+ (\theta^{+})^4 \lambda^{--};
\end{equation}
 Applying now \eqref{6d-gauge-V--V++-differential-relation} and extracting the equation for the $v^{++}$ component only, we obtain
\begin{equation}
\begin{split}
\covDs^{++} v^{++}
&- \theta^{+a} \theta^{+b} \left[  A_{ab}  , v^{++}  \right]
%%\\
%%&
+ 2 \sqrt{2} (\theta^+)^3_a \left[  \Psi^{-a}  , v^{++}  \right]
- 3 (\theta^+)^4 \left[  \covDs^{--} ,  v^{++} \right]
	=
0;
\end{split}
\end{equation}
inserting the expansion \eqref{v++-expansio} into the previous equation and comparing the coefficients term by term in the Grassmann variables, with a bit of work one can find the solution to be
\begin{align}
	\lambda^{++}
& = 
	- \covDs^{++}
,\\
	\lambda^{+}_a
& = 
	i \sqrt{2} \left( \Sigma^M \right)_{ab} \covD_M \Psi^{+b}
,\\
	\lambda_{ab}
& = 
	\frac{1}{2} \left( \Sigma^M \right)_{ab}
		\left[ i \covD_M D^{+-} - \covD^N F_{NM} \right]
	+ \varepsilon_{abcd} i \left\{ \Psi^{-c},\Psi^{+d}\right\}
,\\
	\lambda^{-a}
& = 
	\sqrt{2} \covD^2 \Psi^{ai} u^-_i 
	- i \sqrt{2} F_{MN} \left( \Sigma^{MN} \right)^a_b \Psi^{bi} u^{ bi } u^-_i
	\\ & \qquad \qquad - \frac{4\sqrt{2}}{3} [\Psi^{ai}, D^l_i] u^-_l
	+ \sqrt{2} \, [\psi^{ai}, D^{kl}]\, u^-_{(i}u^-_ku^+_{l)}
,\\
	\lambda^{--}
& = 
	- \covD^2 D^{--} - 3 \left[  D^{--} , \covD D^{+-} \right]
	- 2i \left\{  \Psi^-, \Sigma^M \covD_M \Psi^- \right\}
.
\end{align}






%%Notice that
%%\begin{align}
%%\theta \Sigma_M \theta \, \theta \Sigma_N \theta & = - 8 \theta^4 \eta_{MN}
%%\\
%%\int \dd{ \zeta^{-4} } \left( \theta^+ \right)^4 & = \int \dd{^6 x_A }
%%\\
%%
%%\\
%%\end{align}



We are now ready to consider the action \eqref{hd-action-superspace}. First we can integrate the $\theta^{+a}$, that according to \eqref{integr-zeta(-4)} singles out only the $(\theta^+)^4$-component of $F^{++}$, and we get
\begin{equation}\label{step-component field}
S = \frac{1}{2 g^2}
\int \dd{^6 x} \dd{ u }
\left(
	2 \lambda^{++} \lambda^{--}
	-2 \lambda^{+a}  \lambda^{-a}
	- \varepsilon^{abcd} \lambda_{ab} \lambda_{cd}
\right).
\end{equation}
Using \eqref{integratioon-harmonics} and splitting symmetric and antisymmetric part of the product of $u$'s, the following relations are easy to verify,
\begin{equation}
\int \dd{ u } u^+_i u^j_j  = \frac{1}{2} \varepsilon_{ij}
\qquad
\int \dd{ u } u^+_i u^+_j  u^-_k u^-_l
	 =
\frac{1}{6} 
\left( \varepsilon_{ik} \varepsilon_{jl} -  \varepsilon_{il} \varepsilon_{jk}\right)
\end{equation}
now the integration over the harmonics in \eqref{step-component field} can be performed, finally obtaining
\begin{equation}\label{lagr-hd-sym-6d}
\begin{split}
	\Lagr_{\SYM}^{6d,\HD}
		=
	\frac{1}{2f^2}
		\tr \bigg[
	&			 \left( \covD^M F_{ML} \right)^2
				+ i \Psi^i \Sigma^M \covD_M
					\left( \covD \right)^2 \Psi_i
				+ \frac{1}{2} \left( \covD_M D_{ij} \right)^2
	\\
	&		\quad	
				+ D_{ik} D^{kj} D\indices{_j^i}
				+ \left( \Psi^i \Sigma_M \Psi_i \right)^2
	\\
	&		\quad
				- 2 iD_{jk}
					\left( \Psi^j \Sigma^M \covD_M  \Psi^k
						- \covD_M \Psi^j \Sigma^M \Psi^k \right)
	\\
	&		\quad
				+ \frac{i}{2}  \covD_M \Psi^i \Sigma^M \Sigma^{NS} 
					\left[ F_{NS} , \Psi_j \right]
				- 2i \covD^M F_{MN} \Psi^j \Sigma^N \Psi_j
			\bigg].
\end{split}
\end{equation}







\section[Supersymmetric higher-derivative gauge theories in \texorpdfstring{${ d=4 }$}{d=4}]{Supersymmetric higher-derivative gauge theories in \texorpdfstring{$\boldsymbol{ d=4 }$}{d=4}}


In this Section we are going to compute the relevant contributions to the four dimensional Lagrangian density that will allow us to evaluate the one-loop divergence of the supersymmetric extensions of the higher derivative \ym{} theory in four spacetime dimensions.
We will do this performing trivial dimensional reduction to the theory described by \eqref{lagr-hd-sym-6d}. 



Already from a superficial inspection of \( \Lagr_{\SYM}^{6d,\HD} \), one could guess that the full dimensionally reduced Lagrangian contains quite a number of terms. Indeed, writing down all of them would take roughly ten pages and is not particularly instructive. Moreover, all we need is the counterterm to the kinetic term $\tr F_{\mu\nu}F_{\mu\nu}$, and most of the terms do not contribute to it in a convenient application of the background field method, similarly to what happened with matter fields in the non supersymmetric case. 



In order to do this let us anticipate that we will expand the Lagrangian about a classical background analogous to that considered in \eqref{bkg-with-matter}, namely with a non-vanishing gauge fields only. 
As we will see explicitly, the dimensional reduction transforms the fields as
\begin{align*}
\text{Gauge field } A_M 
&	\longrightarrow 
		\text{gauge field } A_\mu + 
		\text{ complex scalar } \phi = \frac{A + iB }{\sqrt{2} }
\\
\text{Weyl spinor } \Psi^i_a
&	\longrightarrow 
		\text{Weyl spinors } \psi_{ \alpha } + \bar \lambda_{\dot \alpha} 
\\
\text{Auxiliary field } D^{ij}
&	\longrightarrow 
		\text{real + complex auxiliary fields } D + F.	
\end{align*}
where $A$ and $B$ are real scalars, as the notation suggests.
This means that terms at least cubic in the spinors, auxiliary fields and in the induced scalar do not contribute to the one-loop counterterm; moreover in terms quadratic in them and containing gauge fields, the latter is just evaluated at the background value. These considerations allow us to truncate away a lot of contributions.








%%\subparagraph{6d pure higher derivative \sym{}}
%%\begin{equation}\label{lagr-hd-sym-6d}
%%\begin{split}
%%	\Lagr_{\SYM}^{6d,\HD}
%%		=
%%	\frac{1}{f^2}
%%		\tr \bigg[
%%	&			- \left( \covD^M F_{ML} \right)^2
%%				+ i \Psi^i \Sigma^M \covD_M
%%					\left( \covD \right)^2 \Psi_i
%%				+ \frac{1}{2} \left( \covD_M D_{ij} \right)^2
%%	\\
%%	&			
%%				+ D_{ik} D^{kj} D\indices{_j^i}
%%				- 2 iD_{jk}
%%					\left( \Psi^j \Sigma^M \covD_M  \Psi^k
%%						- \covD_M \Psi^j \Sigma^M \Psi^k \right)
%%				+ \left( \Psi^i \Sigma_M \Psi_i \right)^2
%%	\\
%%	&
%%				+ \frac{i}{2}  \covD_M \Psi^i \Sigma^M \Sigma^{NS} 
%%					\left[ F_{NS} , \Psi_j \right]
%%				- 2i \covD^M F_{MN} \Psi^j \Sigma^N \Psi_j
%%			\bigg];
%%\end{split}
%%\end{equation}
%%$f$ is a dimensionless constant. Notice that we added an overall minus sign wih respect to the reference in order to match with the non supersymmetric case for the gauge field.



\subsection[Dimensional reduction of the theory in \texorpdfstring{${ d=6}$}{d=6} ]{Dimensional reduction of the theory in \texorpdfstring{$\boldsymbol{ d=6}$}{d=6} }

\subsubsection{Generalities on trivial dimensional reduction }

We perform here a trivial dimensional reduction of the six dimensional Lagrangian \eqref{lagr-hd-sym-6d}.
By this we mean that we simply drop the dependence on the two coordinates $x^4$ and $x^5$ so that we formally set
\begin{equation}\label{dimensional-reduction-derivative}
\frac{\partial}{\partial x^4} = 0,
\qquad
\frac{\partial}{\partial x^5} = 0
\end{equation}
in \eqref{lagr-hd-sym-6d} and in the definitions in \eqref{6dim-covd-fmunu}. After this simplification, the covariant derivatives in the adjoint representation reads, written in terms of the generator of the fundamental one,
\begin{equation}\label{dimensional-reduction-covD}
\covD_M = 
\left\lbrace
\begin{split}
 & \covD_\mu 	=\partial_\mu + [A_\mu,\#]	\qquad &  M = \mu = 0,1,2,3 \\
 & [A, \#] 									& M = 4 \\			
 & [B, \#]									& M = 5
\end{split}
\right.
\end{equation}
where we denoted $A_{4,5} = (A,B)$.
For future reference we also write the d'Alembertian 
%%(with abuse, \(\left( \covD \right)^2\) indicates both \(\covD^M \covD_M\) and  \(\covD^\mu \covD_\mu\) depending on the context)
in the adjoint representation
\begin{equation}\label{dimensional-reduction-covDcovD}
\tilde\covD^2 =
\covD_M \covD^M 
	=
	\covD^2 
	+ \left[A, [A ,\#]\right]
	+ \left[B, [B,\#]\right],
\end{equation}
% the $\sim$ distinguishes the six dimensional Laplacian to that in four dimensions
where $\covD^2 = \covD^\mu \covD_\mu.$


Let us apply this to the transformation properties in \eqref{6dim-covd-fmunu-gaugetrans}. For $M = \mu$, \eqref{dimensional-reduction-derivative} is ineffective, and so the first four components of $A^M$  become  the four-dimensional \ym{} potential $A_\mu$. For $M= 4,5$ the derivative drops and what is left is 
\begin{align}
 \delta_\omega A = [\omega, A], \hspace{3em}
 \delta_\omega B = [\omega, B],
\end{align}
that means that $A$ and $B$ are Lorentz-scalar fields transforming under the adjoint representation of the gauge group. 
Therefore, the six-dimensional \ym{} potential $A_{M}$ splits according to 
\begin{equation}
A_M = ( A_\mu , A , B ).
\end{equation}
The six-dimensional field strength  $F_{MN}$ reduces accordingly. Indeed, the components $(M,N) = (\mu, \nu)$ become the four-dimensional tensor $F_{\mu\nu}$. The other components read
\begin{align}
F_{ \mu 4 } & = \covD_\mu A, \\
F_{ \mu 5 } & = \covD_\mu B, \\
F_{ 4 5 } & =  [ A_4 , A_5 ];
\end{align}
the remaining components are related to these by the antisymmetry of $F_{MN}$.



As far as spinors are concerned, the dimensional reduction \( 6d \rightarrow 4d \) is suggested by the manifestly covariant \( \GroupName{SL}(2,\mathbb{C}) \) notation we employed for the $\Sigma$ matrices. Indeed, spinor indices split according to as $a=(\alpha, \dot \alpha)$; the reduction corresponds to restrict the spinor representation to the diagonal subgroup of $\GroupName{SU*}(4)$ 
\begin{equation}
\begin{pmatrix}
A & 0 \\
0 & A^*
\end{pmatrix},
\qquad
A \in \GroupName{SU}(2,\mathbb{C}).
\end{equation}
The symbol $\varepsilon^{abcd}$ factorizes into two-index components according to
\begin{equation}
\varepsilon^{abcd} \rightarrow \varepsilon^{\alpha \beta} \varepsilon^{\dot \gamma \dot \delta}
\end{equation}
and antisymmetry of such tensors.




The spinor  $\Psi^1$ therefore splits into two four-dimensional chiral spinors 
\begin{equation}
(\Psi^1)^a = 
\begin{pmatrix}
\psi^\alpha \\
\bar \lambda ^{\dot{\alpha}}
\end{pmatrix},
\qquad\qquad
(\Psi^2)^a = 
\begin{pmatrix}
\lambda^{\alpha}\\
- \bar \psi^{\dot{\alpha}}
\end{pmatrix}
\end{equation}
where \( \Psi^2 \) is fixed by the pseudo-Majorana constraint.


The tensor of auxiliary fields $D^{ij}$ splits into the auxiliary fields of the vector ($D$, real) and chiral ($F$, complex) supermultiplets:
\begin{equation}\label{dim-red-Dij}
	D^{ij} 
		=
	\begin{pmatrix}
		F^*	&	-D \\
		- D	&	- F	
	\end{pmatrix}.
\end{equation}


The overall coefficient $f$ in four dimensional spaceime has then dimension 2, and we write it in terms of the dimensionless \ym{} coupling $g$ as  $f = m^2 g^2$.

%%
%%
%%\subsubsection{Dimensional reduction of \ym{} theory}
%%
%%Let us now apply the dimensional reduction outlined in the preivous Section in order to formulate theories in four dimensions starting with a six dimensional Lagrangian.
%%
%%Before turning to the higher derivative case, let us consider as a warm-up the usual \ym{} theory.
%%We therefore now reduce the terms present in the Lagrangian density \eqref{6 DIM YM}.
%%
%%The gauge field kinetic term reads, using \eqref{dimensional-reduction-covD},
%%\begin{equation}\label{dim-red-FmunuFmunu}
%%\begin{split}
%%- \frac{1}{2} F_{MN}F_{MN} 
%%	& =
%%		- \frac{1}{2} F_{ \mu \nu } F^{ \mu \nu }
%%		-  \covD_\mu A\ \covD^\mu A
%%		-  \covD_\mu B\ \covD^\mu B
%%\\
%%	& \qquad
%%		-  [A,B]^2,
%%\end{split}
%%\end{equation}
%%that is the kinetic term of the usual \ym{} theory and those of the real scalars\footnote{Notice that the scalars have the correct normalization since a factor of $1/2$ comes from the trace over the generators according to \eqref{notation-tr-generatord-fundam}.\label{footnote-normalizations-kinetic-term}} plus a gauge covariant interaction between the scalars.
%%
%%Thanks to the decomposition \eqref{dim-red-Dij} the auxiliary field contribution gives
%%\begin{equation}\label{dim-red-DikDik}
%%\begin{split}
%%- \frac{1}{2} D^{ik} D_{ik}
%%	& =
%%		- \frac{1}{2} D^{11} D^{22}
%%		+ D^{12} D^{21}
%%		- \frac{1}{2} D^{22} D^{11}
%%\\
%%	& =
%%		 F^* F
%%		+ D^2
%%\end{split}
%%\end{equation}
%%where in the first line indices have been raised by means of $\varepsilon.$ This is the usual non-dynamical contribution of the auxiliary fields.
%%
%%The third contribution containing the spinors can be written as
%%\begin{equation}\label{dim-red-psisigmacovdpsi}
%%\begin{split}
%%i \Psi^i \Sigma^M \covD_M \Psi_i
%%	& =
%%		2 i \left( \Psi^1 \right)^a
%%			\Sigma^\mu_{ab} \covD_\mu \left( \Psi^2 \right)^b 
%%\\
%%& \qquad
%%		+ 2 i \left( \Psi^1 \right)^a
%%			\Sigma^4_{ab} \covD_4 \left( \Psi^2 \right)^b 
%%		+ 2 i \left( \Psi^1 \right)^a
%%			\Sigma^5_{ab} \covD_5 \left( \Psi^2 \right)^b 
%%\\
%%	 & =
%%		- 2i \psi \sigma^\mu \covD_\mu \bar \psi
%%		- 2i \lambda \sigma^\mu \covD_\mu \bar \lambda
%%\\
%%& \qquad	
%%			,
%%\end{split}
%%\end{equation}
%%that is the kinetic term for the two spinors and an interaction with the scalars.
%%
%%We can now recognise in the dimensional reduction of the Lagrangian density REF the structure of Super \ym{}. Indeed, we obtain the Lagrangian for $N=2$ \sym{}
%%\begin{equation}
%%\Lagr^{N=2}_\SYM = 
%%	\tr
%%	\left[
%%		- \frac{1}{2} F_{ \mu \nu } F^{ \mu \nu }
%%		- 2i \psi \sigma^\mu \covD_\mu \bar \psi
%%		+ D^2	
%%	\right]
%%	.
%%\end{equation}
%%Truncating away the spinor $\lambda$, the auxiliary field $F$ and the scalars $A$ and $B$ we can read from the previous one the Lagrangian density for $N=1$ \sym{}
%%\begin{equation}\label{lagr-N=1-sym}
%%\Lagr^{N=1}_\SYM = 
%%	\tr
%%	\left[
%%		- \frac{1}{2} F_{ \mu \nu } F^{ \mu \nu }
%%		- 2i \psi \sigma^\mu \covD_\mu \bar \psi
%%		+ D^2	
%%	\right]
%%\end{equation}







%%
%%\subsubsection{Backup}
%%
%%
%%If we first restrict to $N=1$ supersymmetry,\footnote{The Lagrangian, however, has been expanded considering the full $N=2$; but since here only the unextended case is considered, only the truncated one is reported.} we can eliminate many contributions, in particular \( A_{4,5} \), \( F \) and \( \lambda \) are formally set to vanish. In other words only \( F_{\mu\nu} \), \(\psi^{\alpha}\) and \(D\) will be considered from now.
%%
%%Bearing this in mind, all terms can be dimensionally reduced as follows. We will ignore total (covariant) derivatives as usual.
%%
%%Let us start with the usual \sym{} Lagrangian:
%%\begin{align*}
%%- \frac{1}{2} F_{MN}F^{MN} 
%%	& \longrightarrow
%%		- \frac{1}{2} F_{\mu\nu}F^{\mu\nu}
%%	\\
%%- \frac{1}{2}D^{ij}D_{ij}
%%	& \longrightarrow
%%		D^2
%%	\\		
%%i \Psi^i \Sigma^M \covD_M \Psi_i
%%	& \longrightarrow
%%		2 i \left( \Psi^1 \right)^a
%%			\Sigma^\mu_{ab} \covD_\mu \left( \Psi^2 \right)^b 
%%%		= 2 i \left( \Psi^1 \right)^a
%%%			\Sigma^M_{a \dot{b}} 
%%%			\covD_M \left( \Psi^1 \right)^{\dot{b}}
%%		= - 2i \psi \sigma^\mu \covD_\mu \bar \psi
%%\end{align*}
%%yielding the well known Lagrangian REF, with $g$ now dimensionless, and fixing normalizations and relative coefficients consistently with the higher derivative sector. 
%%




\subsubsection{Dimensional reduction of higher derivative Lagrangian}


Let us now apply the described techniques  to dimensionally reduce the Lagrangian \eqref{lagr-hd-sym-6d}. As mentioned, we will only give those terms that will be relevant for the computation.
In particular, all the terms in the second and third %%% SPLIT THE FORMULA BY SENDING TO A NEW LINE THE 5TH TERM
 line, 
%% that are
%%\begin{equation}
%% D_{ik} D^{kj} D\indices{_j^i},
%% \qquad
%% \left( \Psi^i \Sigma^M \Psi_i \right)^2
%%\end{equation}
%%and
%%\begin{equation}
%%- 2 i D_{jk}
%%\left( \Psi^j \Sigma^M \covD_M  \Psi^k
%%- \covD_M \Psi^j \Sigma^M \Psi^k \right)
%%\end{equation}
are, respectively, third order on the auxiliary fields, fourth order on the spinors and third order on spinors and auxiliary fields, so that they do not contribute to the operator relevant for our computation and will not be considered from now on. 


The higher derivative kinetic term for the gauge field decomposes as
\begin{equation}\label{dim-red-covdF2}
\begin{split}
\left( \covD^M F_{ML} \right)^2
	& \rightarrow
		\left( \covD^\mu F_{\mu\nu} \right)^2
		+ ( \covD^2 A )^2
		+ ( \covD^2 B )^2
	\\
	&
	\qquad
	- 2 ( \covD^\mu F_{ \mu \nu } )[ A , \covD^\nu A ]
	- 2 ( \covD^\mu F_{ \mu \nu } )[ B , \covD^\nu B ]
\end{split}
\end{equation}
that is the higher derivative kinetic term for $F_{\mu\nu}$ and for the two scalars. It also produces higher order interactions between the fields, that have been discarded. Notice that the scalar contribution can be written in terms of the complex field as
\begin{equation}
( \covD^2 A )^2
+
( \covD^2 B )^2
=
2 ( \covD^2 \phi )^2
\end{equation}



In a somewhat similar way, the higher derivative kinetic term for the spinor field reads
\begin{equation}\label{dim-red-psisigmacovDcovDpsi}
\begin{split}
i \Psi^i \Sigma^M \covD_M \tilde \covD^2 \Psi_i
	& \rightarrow 
		i \left( \Psi^1 \right)^a \Sigma^\mu_{ab} 
			\covD_\mu \left( \covD \right)^2 \left( \Psi^2 \right)^b 
%%	\\
%%	& \qquad
			- i \left( \Psi^2 \right)^a	\Sigma^\mu_{ab} 
			\covD_\mu \left( \covD \right)^2 \left( \Psi^1 \right)^b 
	\\
	& \quad = 
		- i \psi \left\{ \covD_\mu, \covD^2 \right\}
			\sigma^\mu \bar \psi
%%	\\
%%	& \quad \qquad  
		- i \bar \lambda \bar \sigma^\mu  
			\left\{ \covD_\mu, \covD^2 \right\}	 \lambda
\end{split}
\end{equation}
the other terms produced by \eqref{dimensional-reduction-covDcovD} and \eqref{dimensional-reduction-covD} do not contribute to the quadratic operator, since they are multiplied by two spinor fields.

The kinetic term for the auxiliary fields becomes
\begin{equation}\label{dim-red-covDDij2}
\begin{split}
\frac{1}{2} \left( \covD_M D_{ij} \right)^2
	& \rightarrow 
		\frac{1}{2} ( \covD_\mu D_{ij}) (\covD^{\mu} D^{ij} )
	\\
	& \quad = 
		- ( \covD_\mu F^* ) ( \covD^\mu F )  
		- ( \covD_\mu D ) ( \covD_\mu D )
\end{split}
\end{equation}
where we again dropped interactions with scalars. We also note here that the auxiliary field $F$ is not canonically normalized% (cf.\ footnote \ref{footnote-normalizations-kinetic-term}).



We are now left with only two other contributions.
The first interaction between spinors and gauge fields produces
\begin{equation}\label{dim-red-covDpsisigmasigaFpsi}
\begin{split}
\frac{i}{2} \covD_M \Psi^j \Sigma^M \Sigma^{NS} 
\left[ F_{NS} , \Psi_j \right]
	& \rightarrow 
	\frac{i}{2} \covD_\mu \Psi^1 \Sigma^\mu \Sigma^{\nu\rho} 
	\left[ F_{ \nu \rho } , \Psi^2 \right]
%%\\
%%& \qquad	
	- \frac{i}{2} \covD_\mu \Psi^2 \Sigma^\mu \Sigma^{\nu\rho} 
		\left[ F_{ \nu \rho } , \Psi^1 \right]
\\
& \quad
		=
		- \frac{i}{2} 
			(\covD_\mu \psi) \sigma^\mu \bar \sigma^{\rho \sigma}
			\left[ F_{ \rho \sigma } , 	 \bar \psi \right]
%%\\
%%& \qquad \quad
		- \frac{i}{2}
			(\covD_\mu \bar \psi) \bar \sigma^\mu \sigma^{\rho \sigma}
			\left[ F_{ \rho \sigma } ,  \psi \right]
\\
& \qquad \quad
		- \frac{i}{2} 
			(\covD_\mu \lambda) \sigma^\mu \bar \sigma^{\rho \sigma}
			\left[ F_{ \rho \sigma } , 	 \bar \lambda \right]
%%\\
%%& \qquad \quad
		- \frac{i}{2}
			(\covD_\mu \bar \lambda) \bar \sigma^\mu \sigma^{\rho \sigma}
			\left[ F_{ \rho \sigma } ,  \lambda \right]
\end{split}
\end{equation}
where we discarded interactions with scalars.
  %%-- ELIMINARE INDICI SPINORIALI?

Finally, the second interaction between spinor and gauge boson reduces as
\begin{equation}\label{dim-red-covDFpsisigmapsi}
\begin{split}
2 i \left( \covD^M F_{MN} \right) \Psi^j \Sigma^N \Psi_j
	& \rightarrow 
		- 2 ( \covD^\mu F_{ \mu \nu } )
			\left(
				 \Psi^1 \Sigma^\nu \Psi^2
				 -
				 \Psi^2 \Sigma^\nu \Psi^1
			\right)
\\
	& \quad
		 = 
		- 2\ ( \covD^\mu F_{ \mu \nu } ) %\cdot \phantom{a}
%%		\\
%%	& \hspace{4em}	
%%		\cdot
			\left[
			 \psi \sigma^\nu \bar \psi 
				+ \bar \psi \bar \sigma^\nu \psi 
				+ \lambda \sigma^\nu \bar \lambda 
				+ \bar \lambda \bar \sigma^\nu \lambda 				
				\right],
\end{split}
\end{equation}
up to the by-now usual higher order contributions.




We are now ready to write the supersymmetric extension to the higher derivative contribution to \sym{}, truncated at second order in the matter fields. In principle we are able to write the case of extended $N=2$ supersymmetry, but first, for clarity and simplicity, we write down the $N=1$ invariant Lagrangian. 





The truncated  $N=1$ higher derivative Lagrangian  is then obtained putting together the contributions
\eqref{dim-red-covdF2}-\eqref{dim-red-covDFpsisigmapsi} according to the six-dimensional Lagrangian density \eqref{lagr-hd-sym-6d}, setting to zero the auxiliary field $F$, the spinor $\lambda$ and the scalars $A$ and $B$. 
It therefore reads
\begin{equation}\label{lagr-HDonly-N=1-sym-4d}
\begin{split}
\left.\phantom{\int}\Lagr^{N=1}_{\SYM}\right|_{\HD}
	=
 \frac{1}{m^2 g^2}
 \tr &\bigg[
		\left( \covD^\mu F_{\mu\nu} \right)^2
		- i \psi \left\{ \covD_\mu, \covD^2 \right\}
			\sigma^\mu \bar \psi
		- ( \covD_\mu D ) ( \covD_\mu D )
\\
& \quad
		-\frac{i}{2} 
			(\covD_\mu \psi) \sigma^\mu \bar \sigma^{\rho \sigma}
			\left[ F_{ \rho \sigma } , 	 \bar \psi \right]
		- \frac{i}{2}
			(\covD_\mu \bar \psi) \bar \sigma^\mu \sigma^{\rho \sigma}
			\left[ F_{ \rho \sigma } ,  \psi \right]
\\
& \quad
		- 2 i\ ( \covD^\mu F_{ \mu \nu } ) 
			\left[
			 \psi \sigma^\nu \bar \psi 
				+ \bar \psi \bar \sigma^\nu \psi 
		\right]
\bigg].
\end{split}
\end{equation}
The full $N=1$ higher-derivative \sym{} Lagrangian is obtained  summing  the $N=1$ \sym{} action \eqref{lagr-N1sym-4d} with \eqref{lagr-HDonly-N=1-sym-4d}, that is
\begin{equation}\label{lagr-N=1-HDsym-4d}
\Lagr^{N=1}_{\HD\SYM}
	=
\Lagr^{N=1}_\SYM 
+
\left.\phantom{\int}\hspace{-1em}\Lagr^{N=1}_{\SYM}\right|_{\HD}.
\end{equation}
 By inspecting the contributions to the Lagrangian density \eqref{lagr-N=1-HDsym-4d}, we can recognise the terms related to the na\"ive insertion of $\covD^2$ inside the terms, but also non-trivial contributions dictated by gauge invariance. As anticipated in the introduction to the chapter, the auxiliary field becomes dynamical; by inspecting the sign in front of its kinetic and `mass' term -- that by rescaling the field by $m$ is actually $D^2$ in the usual \sym{} Lagrangian density \eqref{lagr-N1sym-4d} -- we can recognise in it a ghost field. We will come back on this later.




The whole dimensionally reduced Lagrangian gives us the truncated higher-derivative sector of the higher-derivative $N=2$ \sym{} theory. For simplicity of notation we collect  the terms related to higher derivative contributions of the chiral multiplet sector of the full $N=2$  Lagrangian density
\begin{equation}\label{lagr-HDonly-chiral-4d}
\begin{split}
\left.\phantom{\int}\Lagr_{\CH}\right|_{\HD}
	=
%%
\frac{1}{m^2 g^2} \tr & \bigg[
		 2 ( \covD^2 \phi )^2
		- i \lambda \left\{ \covD_\mu, \covD^2 \right\}	\sigma^\mu \bar \lambda
		- ( \covD_\mu F^* ) ( \covD^\mu F )  		
	\\
	& \quad
		- \frac{i}{2} 
			(\covD_\mu \lambda) \sigma^\mu \bar \sigma^{\rho \sigma}
			\left[ F_{ \rho \sigma } , 	 \bar \lambda \right]
		- \frac{i}{2}
			(\covD_\mu \bar \lambda) \bar \sigma^\mu \sigma^{\rho \sigma}
			\left[ F_{ \rho \sigma } ,  \lambda \right]	
	\\
	& \quad
		- 2\ ( \covD^\mu F_{ \mu \nu } )
			\left[
				\lambda \sigma^\nu \bar \lambda 
				+ \bar \lambda \bar \sigma^\nu \lambda 				
				\right]
\bigg].
\end{split}
\end{equation}
The truncated $N=2$ Lagrangian density can then be written
\begin{equation}\label{lagr-N=2-HDsym-4d}
\begin{split}
\Lagr^{ N = 2 }_{ \HD \SYM }
	& =
\Lagr^{N=2}_\SYM 
+ \left.\phantom{\int}\hspace{-1em}\Lagr^{ N = 1 }_{\SYM}\right|_{\HD}
+ \left.\phantom{\int}\hspace{-1em}\Lagr_{ \CH }\right|_{\HD},
\end{split}
\end{equation}
where all the contributions have been just defined.



In order to proceed to a sensible quantization of the theory, namely in order to produce a convergent functional integral as discussed in Chapter~1, we perform a Wick-rotation of the theory. Without facing the details of the procedure, we formally apply the substitution $t \rightarrow i \tau $; the metric then rotates to $\eta^{\mu\nu} \rightarrow \delta^{\mu\nu}$ and the spinor representations are those of $\GroupName{Spin}(4) \approx \GroupName{SU}(2) \times \GroupName{SU}(2) $. As a consequence of the change of the metric, now $\sigma^0 = i \1$. Formally, since we were using a mostly positive metric, we simply cease to distinguish between upper and lower indices, and add an extra minus sign in front of the Lagrangian density, that now equals the energy.




\subsection{Multiplet structure}

In \eqref{lagr-HD-ym-splitted-dof}, \eqref{lagr-HD-scalar-splitted-dof} and \eqref{lagr-HD-spinor-splitted-dof} we have shown that we can rewrite the higher derivative Lagrangian density for massless gauge, scalar and spinor field in terms or a massless field and a massive ghost field, both described by a usual-derivative Lagrangian, the latter being though a ghost.


It is now interesting to check how the degrees of freedom of the involved supermultiplets split within this procedure.
We will analyse the vector and the chiral multiplet, since all other supersymmetric theories of interest for this work can be formulated using them.


We stress here that these  considerations are purely based on the structure of the kinetic term; a more accurate study would be necessary to understand how such grouping relates with supersymmetry transformations.
A hint into this direction was given in \cite{Ferrara:1977mv} for a simple model of the chiral multiplet.


\subsubsection{Vector multiplet}

The vector multiplet contains one vector boson with gauge invariance, one Weyl fermion and one real auxiliary field.

Considering ordinary derivative theories, off shell such fields describe therefore $3_B$, $4_F$ and $1_B$ degrees of freedom; on shell $2_B$, $2_F$ and zero.

Considering higher derivative theories, we have $5+1$ bosonic and $6$ fermionic degrees of freedom on shell. Indeed the multiplet decomposes into massless vector + massless spinor  + massive vector boson + massive fermion + real dynamical auxiliary fields. Off-shell we therefore have $3_B + 4_F$ massless and $3_B + 8_F + 1_B$ massive degrees of freedom; on-shell $2_F + 2_B $ and $3_B + 4_F + 1_B$.

We recognise the structure for the degrees of freedom consisting of the massless vector multiplet without auxiliary field + massive multiplet of ghost fields, the latter consisting of a vector boson + a Dirac fermion and a real dynamical auxiliary field.


The degree-of-freedom counting is summarized in the following table.
 (OD = ordinary derivative; HD = higher derivative; equiv.\ = equivalent defintion in terms of ordinary derivatives only)
\begin{table}[!ht]
\centering
\begin{tabular}{cccccc}
\hline \hline
  &   &   &   & \multicolumn{2}{c}{HD-equiv.} \\ 
%%\hline 
field & off-sh.  & on-sh. (OD) & on-sh. (HD) & off-sh. & on-sh. \\  [0.75ex]
%%\hline  
$A_\mu$ (+ gauge inv.) & $3$ & $2$ & $5$ & $3+4$ & $2+3$ \\ 
%%\hline 
$\psi$ (Weyl) & $4$ & $2$ & $6$ & $4+8$ & $2+4$ \\ 
%%\hline 
$D$ (real) &  $1$ & $0$ & $1$ & $0+1$ & $0+1$ \\ 
\hline \hline 
\end{tabular} 
\end{table}

We notice that the number of on shell bosonic and fermionic degrees of freedom coincide separately in the two categories of fields (massless vs.\ massive ghosts), even though they fail to agree off-shell.


\subsubsection{Chiral multiplet}

The chiral multiplet contains one complex scalar, one Weyl fermion and one complex auxiliary field.

Considering ordinary derivative theories, off shell such fields describe respectively $2_B$, $4_F$ and $2_B$ degrees of freedom; on shell $2_B$, $2_F$ and zero.

Considering higher derivative theories, we have $4+2$ bosonic and $6$ fermionic degrees of freedom on shell. Indeed in the ordinary derivative the multiplet decomposes into massless scalar and spinor + massive complex scalar + massive fermion + complex dynamical auxiliary fields. Off-shell we therefore have $2_B + 4_F$ massless and $2_B + 8_F + 2_B$ massive degrees of freedom; on-shell $2_F + 2_B  $ and $2_B + 4_F + 2_B$.

We can identify the degrees of freedom as the chiral multiplet without auxiliary field + massive multiplet of ghost fields, the latter consisting of a complex  scalar field, a Dirac fermion and a complex dynamical auxiliary field.

The degree-of-freedom counting is summarized in the following table.
 (OD = ordinary derivative; HD = higher derivative; equiv.\ = equivalent definition in terms of ordinary derivatives only)
\begin{table}[!ht]
\centering
\begin{tabular}{cccccc}
\hline \hline
  &   &   &   & \multicolumn{2}{c}{HD-equiv.} \\ 
%%\hline 
field & off-sh.  & on-sh. (OD) & on-sh. (HD) & off-sh. & on-sh. \\  [0.75ex]
%%\hline  
$\phi$ (complex) & 2 & 2 & 4 & 2+2 & 2+2 \\ 
%%\hline 
$\psi$ (Weyl) & 4 & 2 & 6 & 4+8 & 2+4 \\ 
%%\hline 
$F$ (complex) & 2 & 0 & 2 & 0+2 & 0+2 \\ 
\hline \hline 
\end{tabular} 
\end{table}






We notice, again, that the number of bosonic and fermionic degrees of freedom coincide separately in the two categories of fields (massless vs.\ massive ghosts), though this happens only on shell.





\section[Higher-derivative \texorpdfstring{${ N=1}$}{N=1} \sym{} in \texorpdfstring{${ d=4}$}{d=4}]{Higher-derivative \texorpdfstring{$\boldsymbol{ N=1}$}{N=1} \sym{} in \texorpdfstring{$\boldsymbol{ d=4}$}{d=4}}





In this section we are going to compute the one-loop correction to the gauge coupling for the higher derivative $N=1$ \sym{} theory. 


Consider now the $N=1$ supersymmetric Lagrangian density with higher derivative \eqref{lagr-N=1-HDsym-4d}. As done for the other Lagrangian densities that we considered, in order to compute the one-loop beta function we need only the divergent contribution to the \ym{} kinetic term, and this can be obtained by expanding the fields about the background solution
\begin{equation}\label{N=1-bkg-expansion}
A_\mu \rightarrow A_\mu + B_\mu ,
\hspace{3.5em}
\psi \rightarrow \psi,
\hspace{3.5em}
D \rightarrow D,
\end{equation}
where $A_\mu$, $\psi$, $D$ are now the quantum fluctuations, similarly to that we did in  \eqref{bkg-with-matter}.
This means that in the Lagrangian,
the terms that contain only the gauge field can be expanded in the same way that we did for \eqref{lagr-hdym} (with $\gamma = 0$), so we can directly read from \eqref{LW-expansion-intermeriate} the expansion. For this reason it is convenient to choose the same gauge fixing \( G[A+B] = \covD_\mu A_\mu \) with again the integration weight $H = 1/g^2 - \covD^2 /g^2 m^2$ we used there.
The terms containing the spinor and the auxiliary field in \eqref{lagr-HDonly-N=1-sym-4d} are already formally expanded, with the convention that the field $ F_{\rho\sigma} $ is evaluated in the background field configuration.


We are now interested in a decomposition for the Lagrangian to quadratic fluctuations of the type
\begin{equation}
\Lagr^{N=1}_{\HD\SYM,Q^2}
	=
\Lagr^{ N = 1 }_A
+
\Lagr^{ N = 1 }_\psi
+
\Lagr^{ N = 1 }_D
\end{equation}
where each $\Lagr_X$ contains terms up to quadratic order in the fluctuations of the field $X$. In this way we will be able to read the operators for the computation of the effective action and to evaluate their determinants.


As already mentioned, the field content is the vector supermultiplet; it therefore consists of the gauge field, a chiral spinor and a real auxiliary field, that cannot be ignored as we implicitly did for \sym{} theory in Section 1.5.3
since it is now dynamical. 


The effective action for this theory is therefore of the type described in \eqref{eff-act-1loop-gauge}, but it is not well defined because of the ghost character of the auxiliary field $D$, whose contribution makes the functional integral divergent. We will ignore this problem and apply \eqref{eff-act-1loop-gauge} anyway; formally this can be thought as the result of an analytic continuation of the auxiliary field $D \rightarrow iD$.\footnote{This is, of course, unacceptable from a superfield perspective, but since we are interested only in renormalization properties and $D$ is not a physical field we will not investigate this any further.}




In order to compute the beta function of the gauge coupling we therefore have to compute the total coefficient
\begin{equation}\label{b4-generic-N=1}
\left.b_4^\tot\right|_{ N = 1 }
	=
b_4(\Delta_A)
+
b_4(\Delta_D)
-
2 b_4(\Delta_\psi)
-
2 b_4(M_0)
-
 b_4(H)
\end{equation}
The Seeley-deWitt coefficients can be evaluated with the  previously exposed techniques. We will compute them in the next Section; here we directly give the result
\begin{equation}\label{b4-N=1}
\left.b_4^\tot\right|_{ N = 1 }
	=
-\frac{7}{4} C_2 F_{\mu\nu}^a F_{\mu\nu}^a
\end{equation}
yielding upon renormalization to the running of $g$ described by the $\beta$ function \eqref{beta-function-generic}
\begin{equation}
\beta_{N=1} = 
\frac{g^3}{32 \pi^2} \cdot \frac{7}{2}
\end{equation}


This $\beta$ function is positive, and corresponds to a theory perturbative at low energy and with a UV Landau pole. This is completely different from what is found for the \ym{} field in its first supersymmetric extension (\ie $N=1$ \sym{}, that has negative beta function).


It is also interesting to notice that in the expression \eqref{b4-N=1} for $\left.b_4^\tot\right|_{ N = 1 }$ the contribution proportional to $m^4$ cancels out, so that there is no cosmological constant contribution from this model, as prescribed by unbroken supersymmetry.



We now turn to the computation and evaluation of the Seeley-DeWitt coefficients for the involved operators. 
%%As we already mentioned, the chosen background field configurations is such that the spinors and the auxiliary fields are directly the fluctuation, so the gauge fields multyplying them in the covariant derivatives and in the $F_{\rho \sigma}$ factors are evaluated in the background field.


%%
%%In what follows we will describe the computations of the Seeley-deWitt coefficients.


\subsection{Gauge field}

Following what we said in the previous paragraphs, we can read the relevant contribution for the gauge field from what we studied in Chapter~2, specialising the result to the case $\gamma = 0$. The relevant Seeley-deWitt coefficient can directly read from \eqref{b4-tot-hd-gauge}, that gives the overall contribution
\begin{equation}
\begin{split}
b_4^{ \text{gauge} } 
& := b_4(\Delta_A) - 2 b_4(M_0) - b_4(H)
%%\\
%%&
 = \frac{43}{12} C_2 F_{\mu\nu}^a F_{\mu\nu}^a + \frac{3}{2} m^2 d_\ad.
\end{split}
\end{equation}
Of course back then this was the total coefficient, whereas here it is just one of the contributions.









\subsection{Spinor field}
Let us define the spinor operator representing the Lagrangian as
\begin{equation}
		\Lagr_{\psi}
	= 
		\frac{1}{2g^2m^2} 
		 \psi \Delta_{\psi} \bar \psi.
\end{equation}
The spinor operator is readily read off the Lagrangian as
\begin{equation}\label{operator_spinor}
\begin{split}
(\Delta_\psi)
  = 
& \ i  \sigma^\mu \covD^2 \covD_\mu 
%				
%%\\
%%& \quad	
				+ \bigg[
				 i F_{ \rho \mu }  \sigma^\rho	
				- \frac{i}{4} F_{\tau \kappa}
						\sigma^\mu
						\bar \sigma\indices{^{ \tau \kappa}}
%
				- \frac{i}{4} F_{ \tau \kappa } 
			\sigma\indices{^{ \tau \kappa} }
				\sigma^{\mu}
%
			 - i m^2  \sigma^\mu
			\bigg] \covD_{\mu} ;
%%			\right]^{ik}
\end{split}
\end{equation}
we ignored the contribution from the term in the last line of \eqref{lagr-HDonly-N=1-sym-4d} that is a boundary term (it is $B$-like in the notation of \eqref{3rd-order-operator-generic}).
%%\begin{equation}\label{operator_spinor} 
%%\begin{split}
%%(\Delta_\psi)^{ik}_{\alpha\dot\alpha}
%%  = 
%%& \ i  \sigma^\mu_{\alpha\dot\alpha} \covD^2 \covD_\mu 
%%%				
%%\\
%%& \quad	
%%				+ \bigg[ - i m^2  \sigma^\mu_{\alpha\dot\alpha} 
%%				+ i F_{ \rho \mu }  \sigma^\rho	
%%				- \frac{i}{4} F_{\tau \kappa}
%%						\sigma^\mu_{\alpha\dot\beta}
%%						\bar \sigma\indices{^{ \tau \kappa\;}^{\dot\beta}_{\dot \alpha} }
%%%
%%				- \frac{i}{4} F_{ \tau \kappa } 
%%			\sigma\indices{^{ \tau \kappa\;}_{\alpha}^{\beta} }
%%				\sigma^{\mu}_{\beta  \dot \alpha} 
%%%			
%%			\bigg] \covD_{\mu} 
%%%%			\right]^{ik};
%%\end{split}
%%\end{equation}
Its Seeley-deWitt coefficient reads
\begin{equation}\label{b-4-spinor-n=1}
b_4(\Delta_\psi)
	=
\frac{21}{8} C_2 F_{\mu\nu}^a F_{\mu\nu}^a
+m^2 d_\ad
\end{equation}
as we are now going to verify. The derivation is lengthy and does not provide any particular insight, therefore the uninterested reader can skip it and go directly to the next Section.






\subsubsection{Computation of the determinant for the spinor field}


In order to get the coefficient for the spinor determinant, applying the idea outlined in Section~1.2.4, we compose \eqref{operator_spinor} with \(\bar \Delta_1 = i \bar \sigma^\nu \covD_\nu \). The result is
\begin{equation}
\begin{split}
	\bar \Delta_{\psi+1}
& :=
	\Delta_\psi \cdot \bar \Delta_1
= (\covD^2)^2
+
V_\psi^{\mu\nu} \covD_{\mu} \covD_{\nu}
+
U_\psi
\end{split}
\end{equation}
with coefficients
\begin{equation}\label{N=1-Vmunu}
V_\psi^{\mu\nu}
=
- \frac{3}{4}  F_{\tau\rho} \sigma^{\tau\rho} \delta^{\mu\nu}
+ F\indices{^{(\mu|}_\tau} \sigma^{\tau} \bar \sigma^{|\nu)}
+ \frac{1}{4} F_{\rho\kappa} \sigma^{(\mu|} \bar \sigma^{\rho \kappa} \bar \sigma^{|\nu)}
- m^2 \delta^{\mu\nu}
\end{equation}
and
\begin{equation}\label{Upsi}
\begin{split}
U_\psi
=
& 
\frac{1}{8} F_{\nu\rho} F_{\mu\kappa}
 	 \sigma^{ \mu } \bar \sigma^{\nu\rho} \bar \sigma^{\kappa }
+ \frac{1}{8} F_{\nu\rho} F_{\mu\kappa}
	 \sigma^{ \nu \rho } \sigma^{\mu \kappa}  
%%\\
%%& \qquad
- \frac{1}{2} F_{ \mu \rho } F_{ \rho \kappa }
	\sigma^{ \mu } \bar \sigma^{\kappa}
+ \frac{1}{2} m^2 F_{\mu\nu} \sigma^{\mu\nu} 	.
\end{split}
\end{equation}
As in the previous cases, we decomposed the derivatives according to \(\covD_\mu \covD_\nu = \covD_{(\mu}\covD_{\nu)} + \frac{1}{2} F_{\mu\nu} \) in order to ensure the correct symmetries of the coefficients. 




The trace of \( U_\psi \) reads
\begin{equation}
\begin{split}
\Tr U_\psi
&	=
\frac{1}{8} 
	\tr F_{ \nu \rho } F_{ \mu \kappa } 
	\, \str \bar \sigma^{\nu\rho} \bar \sigma^{ \kappa \mu }
+ \frac{1}{8}
	\tr F_{ \nu \rho } F_{ \mu \kappa } 
	\, \str \sigma^{\nu\rho} \sigma^{ \kappa \mu }
\\
& \qquad 
- \frac{1}{2}
	\tr F_{ \mu \rho } F_{ \rho \kappa } 
	\, \str \sigma^{\mu}  \sigma^{\kappa} 
\\
& =
- \tr F_{ \mu \nu} F_{ \mu \nu}
\\
& =
C_2  F^a_{ \mu \nu} F^a_{ \mu \nu}.
\end{split}
\end{equation}
It is interesting to notice that the two contributions in the first line of \eqref{Upsi} cancel against each other, since 
\(
\str \sigma\indices{^{[\mu|}}
\bar \sigma\indices{^{ \tau \kappa}}
\bar \sigma\indices{^{|\nu]}}
=
\str 
\bar\sigma\indices{^{\nu\mu }}
\bar \sigma\indices{^{ \tau \kappa\;}}
=
- 
\str\bar\sigma\indices{^{\mu \nu}}
\bar \sigma\indices{^{ \tau \kappa} }
\). The term $\sim m^2$ has vanishing trace because it is proportional to   \( \tr  \sigma^{\nu\rho} = 0\).

Let us now evaluate the trace $V_\psi^2$, with
\begin{equation}
\begin{split}
V_\psi
& =  V_\psi^{\mu\mu}
= - 4 F_{\nu\rho} \sigma^{\nu\rho} %%+ F_{\nu\rho} \sigma^{\nu\rho} 
	- \frac{1}{2} \bar \sigma^{\nu\rho} F_{\nu\rho} - 4 m^2,
\end{split}
\end{equation}
that contributes with
\begin{equation}
\begin{split}
\Tr V_\psi^2
& =
16 \tr F_{\nu\rho}  F_{\mu\kappa} \, \str\sigma^{\nu\rho}  \sigma^{\mu\kappa} 
	+ 32 m^4 d_\ad
\\
& = 
 64 \, C_2 \, F^a_{\mu\nu}  F^a_{\mu\nu} 
	+ 32 m^4 d_\ad;
\end{split}
\end{equation}
the mixed product vanishes again because \( \str \bar \sigma^{\nu\rho} = 0 \).

Now we consider  \( \Tr V_\psi^{\mu\nu} V_\psi^{\mu\nu} \) is much more complicated due to the various products of sigma matrices. Expanding the product we get
\begin{equation}\label{N=1-trVmunuVmunu}
\begin{split}
\Tr V_\psi^{\mu\nu} V_\psi^{\mu\nu} 
	= 
		\Tr
		\bigg[
		&	\frac{15}{4}
				F_{\nu\rho} F_{\mu\kappa} 
				\sigma^{\nu \rho} \sigma^{\mu \kappa}
			+ F_{\alpha\mu} F\indices{^{(\alpha|}_\rho}
				\sigma^\mu \bar \sigma^\beta  \sigma^\rho \bar \sigma^{|\beta)}
				+ 4 m^4
		\\
		&	\quad	
			+ \frac{1}{16} 
				F_{\nu\rho} F_{\mu\kappa} 
				\sigma^\alpha \bar \sigma^{\nu\rho}  \bar \sigma^\beta
				\sigma^{(\alpha|} \bar \sigma^{\mu\kappa} \bar \sigma^{|\beta)}
		\\
		&	\quad
			+ \frac{1}{2} 
				F_{\alpha\mu} F_{\nu\rho} 
				\sigma^\mu  \bar \sigma^{\beta}  \bar \sigma^\beta
				\sigma^{(\alpha|} \bar \sigma^{\mu\kappa} \bar \sigma^{|\beta)}
		\bigg]
\end{split}
\end{equation}
plus some other traceless terms (though this might not be evident). The first term in \eqref{N=1-trVmunuVmunu} is the sum of the square of the first term in \eqref{N=1-Vmunu} and of twice the mixed product between the first and the second one.
This expression can be manipulated with the techniques used in Section~2.3.4, and it is not really instructive to go through the details of the computation.  
Nonetheless, we mention that, after some algebra, one can find that the two terms in the second and third lines actually cancel against each other, so that the result is all due to to the contributions in the first line and reads
\begin{equation}
\begin{split}
\Tr V_\psi^{\mu\nu} V_\psi^{\mu\nu} 
	= 
		19\, C_2 F_{\mu\nu}^a F_{\mu\nu}^a
		+ 8 m^4 d_\ad.
\end{split}
\end{equation}
Applying \eqref{b4-coeff-3order-implicit}, we then get the result \eqref{b-4-spinor-n=1}.




\subsection{Auxiliary field}

The auxiliary field, up to the overall mentioned sign, contributes with the operator
\begin{equation}
\Delta_D = - \covD^2 + m^2,
\end{equation}
that is straightforward to evaluate, and its coefficient is
\begin{equation}
b_4(\Delta_D)
	= 
		\tr \left[
			\frac{1}{12}  F_{\mu\nu} F_{\mu\nu}
			+ \frac{1}{2} m^4
		\right]
	= 
		- \frac{1}{12} C_2 F^a_{\mu\nu} F^a_{\mu\nu}
		+ \frac{1}{2} m^4 d_\ad.
\end{equation}














\section[Higher-derivative \texorpdfstring{${ N=2}$}{N=1} \sym{} in \texorpdfstring{${ d=4}$}{d=4}]{Higher-derivative \texorpdfstring{$\boldsymbol{ N=2}$}{N=1} \sym{} in \texorpdfstring{$\boldsymbol{ d=4}$}{d=4}}




Here we consider the $N=2$ higher-derivative \sym{}, obtained, as anticipated, directly by dimensional reduction of the full theory \eqref{lagr-hd-sym-6d}.
The quantization, hence the linearisation about a background solution and the gauge fixing, can be carried out as in the case of $N=1$ supersmmetry, since there is no new contribution containing only the gauge field.


We therefore perform the shift \eqref{N=1-bkg-expansion} and we also expand the other spinor, the scalar and the auxiliary $F$ about a vanishing background solution.
The Lagrangian density  can be then cast in the form
\begin{equation}\label{lagr-quadr-N=2}
\Lagr^{N = 2}_{\HD\SYM, Q^2}
	=
\Lagr^{  }_A
+
\Lagr^{  }_\psi
+
\Lagr^{  }_\lambda
+
\Lagr^{  }_\phi
+
\Lagr^{  }_D
+
\Lagr^{  }_F
\end{equation}
where all the symbols have the usual meaning.


By $R$-symmetry the two spinors must enter symmetrically in the Lagrangian; since there is no interaction quadratic in the spinors, their contribution is exactly the same and there is no new mixing between them.



The general considerations for this model are quite similar to those of the $N=1$ case. The only comment that we will add here is that the auxiliary field $F$, now dynamical, has a ghost character, and of course shares the same problems of the other auxiliary field $D$. As we did before, we will apply \eqref{beta-function-generic} anyway, and therefore the computation of the beta function is equivalent to obtain the coefficient
\begin{equation}\label{b4-N=2}
\begin{split}
b_4^\tot
&	=
b_4(\Delta_A)
+
2 b_4(\Delta_\phi)
-
4 b_4(\Delta_{\psi/\lambda})
\\
& \quad
+
b_4(\Delta_D)
+
2 b_4(\Delta_F)
-
2 b_4(M_0)
-
b_4(H)\\
& =
\left.b_4^\tot\right|_{ N = 1 }
+
2 b_4(\Delta_\phi)
-
2 b_4(\Delta_\lambda)
+
2 b_4(\Delta_F)
\end{split}
\end{equation}
where the factors $2$ in front of the contributions of $F$ and $\phi$ are due to their complex nature.

The only heat kernel coefficients that we have to compute in order to `complete' the $N=1$ result  \eqref{b4-N=1} are those of the scalar field $\phi$ and of the auxiliary field $F$.
The Lagrangian density for them is quite simple, at least at the linearised level, and the operators are simply
\begin{align}
 \Delta_\phi  =  \covD^4  +  m^2 \covD^2  ,
\hspace{5em} \Delta_F  =  - \covD^2  +  m^2  ,
\end{align}
whose Seeley-deWitt $b_4$ coefficients are
\begin{align}
& b_4 (\Delta_\phi) = -\frac{1}{6} C_2 F_{\mu\nu}^a F_{\mu\nu}^a+\frac{1}{2} m^4 d_\ad,\\
& b_4 (\Delta_F) = -\frac{1}{12} C_2 F_{\mu\nu}^a F_{\mu\nu}^a + \frac{1}{2} m^4 d_\ad,
.
\end{align}


Putting together the different contributions, we get for the total coefficient \eqref{b4-N=2}
\begin{equation}
\left.b_4^\tot\right|_{ N = 2 }
	=
-\frac{41}{6} C_2 F_{\mu\nu}^a F_{\mu\nu}^a
\end{equation}
yielding upon renormalization to the running of $g$ described by the $\beta$ function \eqref{beta-function-generic}
\begin{equation}
\beta_{ N = 2 } = 
\frac{g^3}{32 \pi^2} \cdot \frac{41}{3} C_2.
\end{equation}
The $\beta$ function in this case is bigger than the unextended $N=1$ case, and it is again positive, corresponding to the presence of a UV Landau pole.
Also for the $N=2$ supersymmetric case, there is no contribution proportional to $m^4$; this is due to the vanishing of such coefficient in the $N=1$ case and for the scalar multiplet separately.





\section[Higher-derivative \texorpdfstring{${ N=4}$}{N=4} \sym{} in \texorpdfstring{${ d=4}$}{d=4}]{Higher-derivative \texorpdfstring{$\boldsymbol{ N=4}$}{N=4} \sym{} in \texorpdfstring{$\boldsymbol{ d=4}$}{d=4}}


%%\section{$\boldsymbol {d=4} $, $\boldsymbol {N=4}$ Super-Higher-Derivative gauge theory}


In terms of $N=1$ supermultiplets, we have the vector multiplet and three scalar multiplets. All fields with the same spin must enter completely symmetrically in the component field Lagrangian, therefore we can gain some information relevant for the new fields to be introduced by looking at the $N=2$ model, at least considering the Lagrangian expanded up to quadratic fluctuations.

Our construction of the Lagrangian will be guided by symmetry principles and by the fact that truncating two chiral multiplets out we have to recover the Lagrangian density for the $N=2$ case \eqref{lagr-quadr-N=2}.

Since there is no interaction term between the spinors in the $N=2$ supersymmetric case, no new term can therefore arise even in $N=4$.
Similarly, for the complex fields, interactions among the scalar fields are of higher order, and interactions with the gauge field like $F_{\mu\nu}F_{\mu\nu}\phi_i\phi_j$ would imply a contribution with the structure $F_{\mu\nu}F_{\mu\nu}\phi^2$ in the $N=2$ Lagrangian density. We notice that such term is absent is not the case, and this is not accidental: It is not possible since it would require a term at least cubic in the field strength in the Lagrangian in $d=6$, as explained in Section~3.1.

Following this line of thought, in order to compute this contribution we have just to repeat the computation for the $N=2$ case considering four times the contribution of fermionic fields and three times the scalar and auxiliary fields. 

Modulo the problem connected to the ghost nature of the auxiliary fields, we can apply \eqref{b4-total} and obtain the Seeley-deWitt coefficient
\begin{equation}
\begin{split}
\left.b_4^\tot\right|_{ N = 4 }
&	=
b_4(\Delta_A)
+
6 b_4(\Delta_\phi)
-
4 b_4(\Delta_{\psi/\lambda})
\\
& \quad
+
b_4(\Delta_D)
+
6 b_4(\Delta_F)
-
2 b_4(M_0)
-
b_4(H)\\
& =
\left.b_4^\tot\right|_{ N = 2 }
+
4 b_4(\Delta_\phi)
-
4 b_4(\Delta_\lambda)
+
4 b_4(\Delta_F)
\end{split}
\end{equation}
and the result is
\begin{equation}
\left.b_4^\tot\right|_{ N = 4 }
=
 -17 C_2 F^a_{ \mu \nu }  F^a_{ \mu \nu } .
\end{equation}


The divergence can as usual be reabsorbed renormalising the coupling constant $g$, whose flow is regulated by the $\beta$  function
\begin{equation}
\beta(g) =  \frac{g_\mu^3}{16 \pi^2} 34 C_2 ,
\end{equation}
is even bigger than the $N=2$ case, so that the coupling $g$ grows much faster with the scale than in the previously considered cases.

Also in this case, for what we said in the $N=2$ case, there is no contribution from $m^4.$




\section{Concluding remarks}

In this Chapter we managed to formulate, at least to a linearised level, the $N=1$, $2$ and $4$ supersymmetric extension of the higher derivative \ym{} theory \eqref{lagr-hdym}. We evaluated the $\beta$ functions of the theories, obtaining
\begin{align}
\beta_{N=1} =  \frac{g^3}{16 \pi^2} \frac{7}{4} C_2,
\qquad\quad
\beta_{N=2} =  \frac{g^3}{16 \pi^2} \frac{41}{3} C_2 ,
\qquad\quad
\beta_{N=4} =  \frac{g^3}{16 \pi^2} 34 C_2 .
\end{align}

Comparing with the result for the non-supersymemtric case, that is
\begin{align}
\beta_{\HD\YM} = - \frac{g^3}{16 \pi^2} \frac{43}{6} C_2,
\end{align}
we see that already in the $N=1$ case the asymptotic freedom is broken, and the theories are not ultraviolet complete any more, but, at least according to this one loop analysis, the coupling increases and diverges in a so-called Landau pole. This is the same behaviour observed in QED (see \eg{} \cite{Ram}). The extended supersymmetry models give an even faster running for the coupling $g$, so that the Landau point is expected to be at lower energy. In the usual-derivative supersymmetric theories, as computed in Section~1.5, the $\beta$ functions never turn positive; moreover, none of the considered theories maintains the conformal symmetry at the quantum level, while this happens for $N=4$ \sym{}.


Another feature that we pointed out during the analysis, is that in all the supersymmetric invariant Lagrangians there is not contribution to the cosmological constant, that is all divergences are reabsorbed renormalising the coupling, while in the case of  \ym{} field only, such a contribution is present.





