% !TeX program = lualatex
% !TeX encoding = UTF-8
% !TeX root = ./thesis.tex

\usepackage[utf8]{inputenc}
\usepackage[english]{babel}
\hyphenation{equiv-a-lence}


%
%\usepackage{xparse}
%\NewDocumentCommand{\verbl}{v}{%
%	#1%
%}
\usepackage{listings}
\usepackage{color}
%\usepackage{minted}
%\RequirePackage{shellesc}                % implement \write18
%
%\RequirePackage{pdftexcmds}              % fake \pdfshellescape so
%\makeatletter                            % standalone knows about it
%\let\pdfshellescape\pdf@shellescape
%\makeatother
%
%\RequirePackage{luatex85}


\usepackage{amsmath}
\usepackage{amsfonts}
\usepackage{amssymb}
\usepackage{dsfont}\newcommand{\singleton}{\mathds{1}}
%\usepackage{bbm}\newcommand{\singleton}{\mathbbm{1}}
%\newcommand{\singleton}{{1}}

\usepackage{stix}
\newcommand{\trad}[1]{\ensuremath{ 
		{\lBrack {#1} \rBrack} 
	}}
%\usepackage{textcomp}\newcommand{\trad}[1]{\textlbrackdbl{#1}\textrbrackdbl}
%\usepackage{stmaryrd}\newcommand{\trad}[1]{\left\llbracket#1\right\rrbracket}
%\allowdisplaybreaks % allows pagebreak inside align; prevent with: \\*
%\usepackage{amsopn}
%\usepackage{csquotes}
%\usepackage{tensor} %provides command \indices
%\usepackage{bbm} % provides \mathbbm{1}
%\usepackage{slashed} % proves \slashed{#}
%\usepackage{mathdesign} % packages to modify the look of math mode
%\usepackage{mathastext}
%\usepackage{ccfonts}
%\usepackage[math]{iwona}

%\usepackage{pdfpages}
%\usepackage{graphicx}
%\usepackage[
%%width=2.00cm, height=10.00cm, 
%a6paper, left=0.50cm, right=0.80cm, top=1.00cm, bottom=1.00cm]{geometry}
%%\usepackage{showkeys} %SHOW LABELS
%%\usepackage{tikz}
%\usepackage[small]{caption}

\newcommand{\ie}{{i.e.}{\ }}
\newcommand{\eg}{e.g.{\ }}

\numberwithin{equation}{section} % numbering Chpater.section

%\usepackage[nottoc]{tocbibind} % Show biblio in toc
%\usepackage{natbib} % per selezionare lo stile alpha
%\usepackage{filecontents}
\usepackage[style=alphabetic,backend=biber]{biblatex}
	%\addbibresource{\jobname.bib} %to be used with \begin{filecontents}{\jobname.bib}
	\addbibresource{./bibo.bib}
%\usepackage{fancyhdr}
%\usepackage{pgfplots}
%\usepackage{color} % not to have ugli red squares
%\usepackage{hyperref} % internal links from toc to document
%\hypersetup{
%	colorlinks=true, % make the links colored
%	linkcolor=black, % color TOC links in blue
%	urlcolor=black, % color URLs in red
%	citecolor = black,
%	linktoc=all % 'all' will create links for everything in the TOC
%}


\usepackage{ebproof}%\usepackage{proof}

\author{Alberto Fiori}
\title{Towards an implementation in LambdaProlog of the two level Minimalist Foundation}


%\setlength{\headheight}{13.6pt} 
%\usepackage[nouppercase,swapnames]{frontespizio}




\newcommand{\noop}[1]{}
\newcommand{\s}[1]{\noop\textsf{\text{\ensuremath{\mathsf{#1}}}}}
\newcommand{\type}{\s{type}}


\definecolor{mygreen}{rgb}{0,0.6,0}
\definecolor{mygray}{rgb}{0.5,0.5,0.5}
\definecolor{mymauve}{rgb}{0.58,0,0.82}
\lstset{showstringspaces=false, frame=single,
	literate = {->}{{{$\rightarrow$}}}{1}
		{=>}{{{$\Rightarrow$}}}{1}
		{:-}{{{$\vdash$}}}{1}
		{'}{{\!\textquotesingle}}{1},
	backgroundcolor=\color{white},   % choose the background color; you must add \usepackage{color} or \usepackage{xcolor}; should come as last argument
	basicstyle=\footnotesize,        % the size of the fonts that are used for the code
	breakatwhitespace=true,          % sets if automatic breaks should only happen at whitespace
	breaklines=true,                 % sets automatic line breaking
	captionpos=b,                    % sets the caption-position to bottom
	commentstyle=\color{mygreen},    % comment style
%	deletekeywords={...},            % if you want to delete keywords from the given language
	escapeinside={\%*}{*)},          % if you want to add LaTeX within your code
	extendedchars=true,              % lets you use non-ASCII characters; for 8-bits encodings only, does not work with UTF-8
	frame=single,	                   % adds a frame around the code
	keepspaces=true,                 % keeps spaces in text, useful for keeping indentation of code (possibly needs columns=flexible)
	keywordstyle=\color{blue},       % keyword style
%	language=Octave,                 % the language of the code
	morekeywords={*,
	mttType, mttTerm, mttKind, mttLevel, int ext
	set, col, props, propc, 
	setPi, app, lambda,
	setSigma, pair, elim_singleton,
	elim_setSigma
	forall, and, pair_and, p1_and, p2_and, exist, pair_exist, elim_exist, forall, forall_lam, forall_app,
	propId, id, elim_id, implies, impl_app, impl_lam, letIn, or, inl_or, inr_or, elim_or,
	setSum, inl, inr, elim_setSum, singleton, star
	},            % if you want to add more keywords to the set
	numbers=left,                    % where to put the line-numbers; possible values are (none, left, right)
	numbersep=5pt,                   % how far the line-numbers are from the code
	numberstyle=\tiny\color{mygray}, % the style that is used for the line-numbers
	rulecolor=\color{black},         % if not set, the frame-color may be changed on line-breaks within not-black text (e.g. comments (green here))
	showspaces=false,                % show spaces everywhere adding particular underscores; it overrides 'showstringspaces'
	showstringspaces=false,          % underline spaces within strings only
	showtabs=false,                  % show tabs within strings adding particular underscores
	stepnumber=2,                    % the step between two line-numbers. If it's 1, each line will be numbered
	stringstyle=\color{mymauve},     % string literal style
	tabsize=2,	                   % sets default tabsize to 2 spaces
	title=\lstname                   % show the filename of files included with \lstinputlisting; also try caption instead of title
}

\lstset{identifierstyle=\idstyle}

\makeatletter
\newcommand*\idstyle{%
	\expandafter\id@style\the\lst@token\relax
}
\def\id@style#1#2\relax{%
	\ifcat#1\relax\else
	\ifnum`#1=\uccode`#1%
	\bfseries
	\fi
	\fi
}
\makeatother



