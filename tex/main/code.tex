% !TeX program = lualatex
% !TeX encoding = UTF-8
% !TeX root = ./thesis.tex

\section{The Code}
	All the code was written in $\lambda$Prolog (in particular it was written for the ELPI implementation \cite{dunchev2015elpi} of $\lambda$Prolog) and make extensive use of many high level features that in other languages would not be available. Still our code does not use many of the more advanced features offered by ELPI like constraint programming which we plan to use in the next phase of the project when we will start to implement a kernel for the planned proof checker

	In this section we will show some extract of the code and explain how we obtained the desired functionalities.
	\subsection{Project Structure and Extensibility}
		We have divided the code 4 categories \verb|main|, \verb|calc|, \verb|test|, \verb|debug|.
		The first, \verb|main| holds type declaration for the other modules and define the general functionality of the theory shared by all type constructor of both levels.
		the second is a folder containing a \verb|.elpi| file for each type constructor; in each of these files we implement the typing and conversion rules for that particular type constructor. For example in \verb|calc/setPi.elpi| we keep all definitions related to dependent products, like the implementation of $\Pi$-I or the interpretation of a function term.
		\begin{verbatim}
		of (lambda B F) (setPi B C) IE
		    :- ofType B _ IE
		    ,  (pi x\ locDecl x B => isa (F x) (C x) IE)
		    .		
		\end{verbatim}

		\begin{verbatim}
		interp (lambda B F) R
		    :- spy(of (lambda B F) (setPi B C) ext)
		    ,  spy(interp (setPi B C) (setSigma (setPi Bi Ci) H ))
		    ,  macro_interp B ( x\_\_\xi\_\_\ interp (F x) (Fi xi))
		    ,  setoid_eq B EquB
		    ,  macro_interp B (x1\x2\h\x1i\x2i\hi\tau_proof_eq 
		        (F x1) 
		        (F x2) 
		        (C x2) 
		        (K_EQU x1i x2i hi))
		    ,  R = pair (setPi Bi Ci) (H) (lambda Bi Fi)
		        (forall_lam Bi x1\ 
		        	forall_lam Bi x2\ 
		        	    forall_lam (EquB x1 x2) h\ K_EQU x1 x2 h)
		    .
		\end{verbatim}
		
		The last two, \verb|test| and \verb|debug|,  hold utilities to quickly test the predicates from \verb|main| and \verb|calc| in addition to useful predicates needed to inspect the runtime behaviour of the type checking and of the interpretation.
	\subsection{Explanation of the Predicates}
	We can divide the predicate defined in our implementation in three groups: judgement predicates, conversion predicates and interpretation predicates.
	\subsubsection{\texttt{of}, \texttt{isa}, \texttt{ofType}, \texttt{isaType} and \texttt{locDecl}}
	In the first group are the predicate corresponding to the judgements $A \ type\ [\Gamma]$ or $a \in A\ [\Gamma]$. We use the \verb|of Term Type Level| predicate to represent the judgement $Term \in Type\ [\Gamma]$ at the level indicated by the third argument, we also define another predicate \verb|isa Term Type Level| to check if the inferred type of \verb|Term| is convertible with \verb|Type| at level \verb|Level|; a major difference between these two predicate is that for \verb|of| the first and last argument are inputs and the middle is an output, while for \verb|isa| all three arguments are inputs. Similarly \verb|ofType Type Kind Level| and \verb|isaType Type Kind Level| implement the corresponding functionality at the type level.
	
	\verb|locDecl| is used to form the context where \verb|of| will look up the types of variables via the clause \verb|of X Y _ :- locDecl X Y.|. We have no need to have a parameter to indicate the level for \verb|locDecl| because extensional and intensional variables are disjoint and we can assume that we will never try to compute the extensional type of an intensional variable. Moreover as of now our implementation validates the $\xi$-rule at the intensional level so that also the conversion can be level agnostic.
	
	\subsubsection{\texttt{pts\char`_*} Predicates}
	In the Minimalist Foundation in general the kind of a type is not unique but admit a most general unifier, this unifier will be the intended output of \verb|ofType| and the logic to find it is implemented in the various PTS predicates. As an example a dependent product $\Pi(B,C)$ is never a proposition  and is a set if both $A\ set$ and $B(x)\ set\ [x\in A]$ hold.
	
	This can be expressed with the following code:
	\begin{verbatim}
		pts_leq A A.
		pts_leq set col.
		pts_leq props set.
		pts_leq propc col.
		pts_leq props propc.
		pts_leq props col.
		
		pts_fun A B set :- pts_leq A set, pts_leq B set, !.
		pts_fun _ _ col.
	\end{verbatim}
	Other variations allow to decide the most general kind of universal/existential quantifiers, connectives or lists.
	\subsubsection{\texttt{conv}, \texttt{dconv}, \texttt{hnf} and \texttt{hstep}}
	While conversion at the intensional level is almost straightforward at the extensional level we need to be more careful as a naive implementation would be extremely inefficient due to the rule Eq-E \[\begin{prooftree}\hypo{\s{true}\in\s{Eq}(C,c,d)\ [c\in C, d\in C]}\infer1{c = d \in C}\end{prooftree}\] 
	
	The solution we have chose to implement consist of two step, first we weaken the Eq-E rule to perform a context lookup instead of trying to prove a judgement and then we divide the conversion in three different operations.
	
	\begin{verbatim}
		conv A A :- !.
		conv A B :- locDecl _ (propEq _ A B).
		conv A B :- (hnf A A'), (hnf B B'), (dconv A' B').
	\end{verbatim}
	
	The predicate that will actually implement the conversion judgement will be \verb|conv|, for this predicate both arguments will be inputs. When called between two terms the predicate will first check if the two terms are unifiable at the meta-level, if this fail then it will search for a context declaration that states the equality it wants; if this also fail then if will reduces both term to head normal form and apply a \verb|dconv| step on the head normal forms. \verb|dconv| role is to propagate the conversion to subterms; it assumes to receive two terms in head normal form (as it will only ever be called on output of \verb|hnf|) and and will check for conversion on the corresponding subterms of its arguments.
	\begin{verbatim}
	dconv (setSigma B C) (setSigma B' C') 
	    :- (conv B B')
	    ,  (pi x\ locDecl x B => conv (C x) (C' x))
	    .
	dconv (pair B C BB CC) (pair B' C' BB' CC') 
	    :- (conv B B')
	    ,  (pi x\ locDecl x B => conv (C x) (C' x))
	    ,  (conv BB BB)
	    ,  (conv CC CC')
	    .
	dconv (elim_setSigma Pair M MM) (elim_setSigma Pair' M' MM') 
	    :- (conv Pair Pair')
	    ,  (of Pair (setSigma B C))
	    ,  (pi z\ locDecl z (setSigma B C) => conv (M z) (M' z))
	    ,  (pi x\ pi y\  locDecl x B 
	        => locDecl y (C x) 
	        => conv (MM x y) (MM' x y))
	    .
	\end{verbatim}
	In general it will be needed to declare a \verb|dconv| clause for each new non-constant constructor (constant constructors are all handled by a \verb|dconv A A :- !.| clause) the main reason to use \verb|dconv| is that is does not trigger the context lookup associated with the Eq-E rule.
	\subsubsection{\texttt{tau\char`_*}, \texttt{setoid\char`_*} and \texttt{interp}}
	The main entry point of the interpretation is the predicate \verb|interpr|, for example the typical structure of our test is \begin{enumerate}
		\item Construct a suitable extensional type
		\item Typecheck the type at the exstensional level
		\item  Interpret the type to the instensional level
		\item Typecheck this last result.
	\end{enumerate}
	During the interpretation many other auxiliary predicates are needed. Here we will give a description of the most important ones.
	
	As explained above during the interpretation we need to correct the types of interpreted terms to compensate for the different rules at the two levels; in our code we have chosen to implement this functionality with the predicate \verb|tau T1 T2 P|. \verb|tau| takes in input 2 mutually convertible extensional types and return a $\lambda$Prolog function which convert elements of type \verb|T1| to elements of type \verb|T2|. It is important to note that \verb|P| is a pure $\lambda$-function at the meta-level and thus in the conversion between the two types \verb|T1| and \verb|T2| we can not use backtracking or patter matching; thankfully a pure function is all we need because canonical isomorphism only apply $\eta$-expansions and do not actually inspect the term they are applied to, only its type and that is completely know to \verb|tau|. As an example we show how this type correction would look like for a \emph{disjoint sum} for a variable $w \in \trad{B(x)+C(x)} \ [\trad{x \in A}]$ when transitioning to a type $\trad{B(x')+C(x')}$  \[\sigma_{x}^{x'}(w) \equiv \s{Elim}_+(w, (y_1).\sigma_{x}^{x'}(y_1), (y_2).\sigma_{x}^{x'}(y_2))\].
	 
	We can see that we have applied structural recursion over the types even when computing a function over terms, this is helpful when treating types with more than one canonical constructor and  when dealing with variables, but in general it is necessary since we cannot assume that all terms will have a canonical head normal form.

	It is important to note that all predicates working at the interpretation level will always assume to handle well formed and well typed terms and types so that we do not need to fill our code with an enormous amount of runtime checks.
		
	A related predicate is \verb|tau_trasp|, which at the moment is conceptually the most complex predicate in the code and its aim is to implement the second and third property of required of substitution morphisms: preserving equalities of Q(mTT). In formulas this means that the following hypothetical judgement should be derivable for some choice of $d_2$ 
	\begin{multline}
	d_2 \in \sigma_{x_1}^{x_2}(y) =_{B(x_2)} \sigma_{x_1}^{x_2}(y') \ [x_1 \in A,\ x_2 \in A,\ d_1 \in x_1 = x_2,\\ y\in B(x_1),\ y\in B(x_2), w\in y =_{B(x_1)} y']\end{multline}
	\verb|tau_trasp| role is exactly to find a constructive and functional (after taking as inputs $B(x_1)$ and $B(x_2)$) transformation that produces such $d_2$ from $y$, $y'$ and $d_1$.
	
	This predicate becomes most useful when during the interpretation of dependent products, since in Q(mTT) a dependent function is represented as a pair containing a mTT function $f$ and a proof that $f$ preserve the extensionality of Q(mTT), similarly to how it is done in \cite{necula1997proof}.
	
	Another problem is that in dealing with dependent products is that we need to automatically generate a proof that the interpretations of emTT function will preserve the equalities of Q(mTT). The generation of these proofs is handled by the predicate \verb|tau_proof_eq|.
	
	A family of related predicates is built around the setoids of Q(mTT), the principal one \verb|setoid_eq| takes an extensional type in input and return an mTT proposition representing the setoid equality of the interpretation of the given type, three related predicates \verb|setoid_refl|, \verb|setoid_symm| and \verb|setoid_trans| produce proofs that the proposition returned by \verb|setoid_eq| is indeed an equivalence relation.
