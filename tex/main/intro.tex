
%\chapter*{Introduction}
%		\addcontentsline{toc}{chapter}{Introduction}\markboth{Introduction}{}
%
%		\fancyhead{} % cancella tutti i campi
%			\fancyhead[LE]{\scshape \leftmark}
%			\fancyhead[RO]{\scshape \rightmark}
%	\fancyfoot[CE,CO]{\thepage}
%	\fancyfoot[LO,RO]{ }
%	\fancyfoot[LE,RE]{ }
%		 		\renewcommand{\headrulewidth}{0.4pt}
%			\renewcommand{\footrulewidth}{0.4pt}
%		\pagestyle{fancy}
%		\renewcommand{\sectionmark}[1]{\markright{\thesection.\ #1}}
%		\renewcommand{\chaptermark}[1]{\markboth{\chaptername\ \thechapter.\ #1}{}}
%	\fancyfoot[CE,CO]{\thepage}
%	\fancyfoot[LO,RO]{ }
%	\fancyfoot[LE,RE]{ }

\section*{Introduction}
\addcontentsline{toc}{section}{Introduction}

In 2005 M. Maietti and G. Sambin \cite{maietti2005toward} argued about the necessity of building a foundation for constructive mathematics to be taken as a common core among relevant existing foundations in axiomatic set theory, such as Aczel-Myhill's CZF and Mizar's Tarski-Grothendieck set theory, or in category theory, such as the internal theory of a topos, or in type theory, such as Martin-L\"of's type theory and Coquand's Calculus of Inductive Constructions. 
Moreover, they asked the foundation to satisfy the ``proofs-as-programs'' paradigm, namely the existence of a realizability model where to extract programs from proofs. 
Finally, the authors wanted the theory to be appealing to standard mathematicians and therefore they wanted extensionality in the theory, \eg to reason with quotient types and to avoid the intricacies of dependent types in intensional theories.

In the same paper they noticed that theories satisfying extensional properties, like extensionality of functions, cannot satisfy the proofs-as-programs requirement.
Therefore they concluded that in a proofs-as-programs theory one can only represent extensional concepts by modelling them via intensional ones in a suitable way, as already partially shown in some categorical contexts.

Finally, they ended up proposing a constructive foundation for mathematics equipped with two levels: an intensional level that acts as a programming language and is the actual proofs-as-programs theory, and an extensional level that acts as the set theory where to formalize mathematical proofs. Then, the constructivity of the whole foundation relies on the fact that the extensional level must be implemented over the intensional level, but not only this. Indeed, following Sambin's forget-restore principle, they also required that extensional concepts must be abstractions of intensional ones as result of forgetting irrelevant computational information.
Such information is then restored when extensional concepts are translated back at the intensional level.


In 2009 M. Maietti \cite{maietti2009minimalist} presented the syntax and judgements of the two levels, together with a proof that a suitable completion of the intentional level provides a model of the extensional one. The proof is constructive and based on a sequence of categorical constructions, the most important being the construction of a quotient model within the intensional level, where setoids are used to encode extensional equality in an intensional language, and a notion of canonical isomorphism between intensional dependent setoids.


In this work we will present an implementation of the following software components:
\begin{enumerate}
	\item a type checker for the intensional level
	\item a reformulation of the extensional level that allows to store sintactically proof objects that are later used to provide the information to be restored when going from the extensional to the intensional level so to avoid general proof search during the interpretation
	\item a type checker for the obtained extensional level
	\item a translator from well-typed extensional terms and types to well-typed intensional terms and types
\end{enumerate}


As a future extension we hope to implement also other software components:
\begin{enumerate}
	\item a formal validation (in Abella) of our reformulation of the extensional level and of the correctness of the implementation
	\item code extraction at the intensional level (in Haskell/Lisp)
	\item a proof assistant at the extensional level (using the constraint programming functionality offered by ELPI)
\end{enumerate}
Combining the translator with a proof extraction component it will be possible to extract programs from proofs written in the extensional level.

We have chosen $\lambda$Prolog \cite{nadathur1988overview} (and in particular its recent implementation ELPI \cite{dunchev2015elpi}) as the programming language to write the two type checkers and the interpreter. The benefits are that $\lambda$Prolog takes care of the intricacies of dealing with binders and alpha-conversion and moreover a $\lambda$Prolog implementation of a syntax-directed judgement is just made of simple clauses that are almost literal translations of the judgemental rules. This allows humans (and logicians in particular) to easily inspect the code to spot possible errors.










%
%It can be shown that, under quite general hypotheses, a theory is renormalizable if all the couplings have non-negative mass dimension, and this constraints the form of the Lagrangian densities for renormalizable theories. Indeed, since the fields have positive canonical dimension and the derivative operator has mass dimension $1$, only a small number of independent terms can be constructed. Therefore, the presence of derivative interactions is often linked to non-renormalizable theories. 
%
%
%
%A loophole in the previous result is the fact that it assumes that the propagator scales, for high energies, as the inverse of the squared momentum, but with a carefully chosen derivative term in the equations of motion one can improve the ultraviolet properties of the theory. Indeed, one way to deal with divergent computations in QED, introduced by Pauli and Villars, is the substitution of the photon propagator with
%\begin{equation*}%\tag{\star}\label{PV-regularization}
% \frac{i \eta_{\mu\nu}}{p^2} \longrightarrow 
% \frac{i \eta_{\mu\nu}}{p^2}  -  \frac{i \eta_{\mu\nu}}{p^2 - m^2}
% =
%- \frac{i \eta_{\mu\nu} m^2}{p^2(p^2-m^2)}
%\sim \frac{i \eta_{\mu\nu}}{\left( p^2 \right)^2 },
%\end{equation*}
%%In this thesis we will study the higher-derivative \ym{} theory and will compute its $\beta$ function.
%recovering the original theory taking the limit in $m^2 \rightarrow \infty$ at the end of the computation. Lee and Wick, in \cite{Lee:1969fy,Lee:1970iw}, investigated the properties of promoting  the new term arising from such substitution to a fundamental degree of freedom. In the Abelian gauge theory one can obtain such result by adding to the Lagrangian a term with the structure
%\(
%\sim  F_{\mu\nu} \partial^2  F^{\mu\nu}/ m^2
%\). Notice that this kind of modification is peculiar for it is quadratic in the fields, that is the reason why it modifies the propagator.
%
%
%
%
%These motivations suggested Grinstein, O'Connell and Wise  to  discuss in \cite{Grinstein:2007mp,Grinstein:2007iz,Grinstein:2008qq,Wise:2009mi} a whole extension of the Standard Model  inserting a quadratic differential operator in the conventional kinetic term. This was done for all the fields in the theory.
%This model enjoys many interesting properties thanks to the improved ultraviolet behaviour; for instance, the Higgs boson mass is free of quadratic divergences in the cut-off, so that the hierarchy problem is solved, and this theory is also compatible with other known possible extensions of the Standard Model such as the see-saw mechanism to account for neutrino masses, as discussed in \cite{Espinosa:2007ny}. Despite the desirable properties, results from LHC seem not to be compatible with an extension of this form (\cite{Rizzo:2007ae}).
%However, such models are interesting from a theoretical perspective for their properties and because they arise in various contexts. They also serve as toy models or low energy approximations of other theories, as we are going to discuss. 
%
%
%
%
%%%These kind of theories with higher derivatives appear in other contexts as well.
%Quite remarkably, it was discovered (see \cite{Stelle:1976gc,Fradkin:1981iu,Avram}) that there exists an extension (actually, a two-parameter family of extensions) of General Relativity in which the Lagrangian contains quadratic terms in the Ricci tensor that turns out to be renormalizable. This is once again a reflection of the fact that these kind of operators improve the ultraviolet properties of the theory. Also in modern developments of theoretical physics, higher derivative couplings arise as low energy expansions of string theory models, such as the low energy expansion of conformal theories (see for instance \cite{Stelle:1978, Buchbinder:1999jn}).
%
%
%
%
%
%All these nice properties of the higher derivative theories that we are considering come at some cost. 
%We can understand this already at the classical level. The equations of motion  are fourth-order differential equations, so that they admit not only oscillatory solutions, but also solutions that grow exponentially with time (both in the past and in the future). This instability is manifested in the quantum mechanical formulation for the presence of ghost states: Indeed, as suggested in the expansion of the propagator considered above, the four-derivative theory can be expanded into a couple of ordinary-derivative theories, in which one of the two fields has an extra a negative sign in front of the kinetic term, that implies, upon canonical quantization, the presence of negative norm states.
%The presence of these extra propagating degrees of freedom is represented also in the fact that the propagator has more poles than in the usual-derivative case; these actually make the Wick rotation problematic, since a proper definition of the contour of integration is a difficult task.
%
%
%
%
%%Notice then that the euclidean path integral obtained after  Wick rotation is divergent if one considers the formulation of the theory  in terms of ordinary-derivative ones, since it explicitly contains the ghost particle. On the other hand, a formal application of the rule $t\rightarrow i \tau $ gives an apparently convergent result if one considers the higher derivative formulation, since the operator $(\partial^2)^2$ is positive. This contradiction lies in the fact that we ignored the position of the extra poles in the complex plane of the momentum, that actually make the Wick rotation ill-defined.
%
%
%
%These problems can be partially cured in different ways. One is to impose future and past boundary conditions on the solutions, requiring that in the initial and final states no ghost particle are present. This can be done with a careful definition of the contour of integration that defines the Feynman propagator; this violates the usual causal relations, but ghost partner particles would then contribute only to virtual processes without spoiling the unitarity of the theory. Other approaches, like in \cite{Hawking:2001yt,Ghilencea:2007ex}, allow for a violation of unitarity in spite of maintaining the causal structure of the theory. However such problems might not be an issue if the theory is an effective theory in which the violation of causality or unitarity arises in a region of the parameter space that is out of the regime of applicability of the theory.
%
%
%
%
%\vspace{1.5em}
%
%
%
%Another popular candidate (\cite{Tommaso, WB, Sohnius:1985qm}) to describe physics beyond the Standard Model is supersymmetry. It is a symmetry relating the bosonic and fermionic fields in a given model, so that the two kind of degrees of freedom, in a quite  broad meaning, balance each other. 
%Supersymmetry in this way provides candidates for dark matter; since bosons and fermions tend to give an opposite contribution to a given process, it also improves the ultraviolet properties of the theory, solving in particular the hierarchy problem discussed above. Supersymmetry also allows for a systematic cancellation of the vacuum energy contribution between the fields, thus hinting to a possible explanation to the smallness of the observed cosmological constant. Supersymmetry is not a manifest symmetry of nature, and therefore it should be broken at our energies. It also ties very well with string theory, that is another widely believed framework in which to unify the Standard Model and General Relativity, and many aspects of modern fundamental physics are nowadays driven by supersymmetric principles. 
%It is then natural to wonder how supersymmetry relates with higher derivative theories, both in considering fundamental models and in dealing with low energy effective theories.
%
%
%
%
%A further motivation to work in this direction rises when considering renormalizable theories in higher dimensional spacetime.
%It is nowadays widely accepted the idea that our four-dimensional Universe is actually a submanifold embedded in some higher dimensional structure;  it is then of interest to consider the question of finding renormalizable and supersymmetric theories in such spacetime. As discussed in \cite{Smilga:2005pr,Ivanov:2005qf}, besides supersymmetry and renormalizability, another desirable requirement for the fundamental theory, that would ensure its ultraviolet completeness, is the invariance under conformal symmetry. This symmetry extends the Poincar\'e group essentially to include also circular inversions and dilatations. Superconformal algebras have been studied and classified (\cite{VanProeyen:1999ni} is a review), and it turns out that the highest possible dimension in which superconformal symmetry can be realised is six.
%Renormalizability is in general a property that is not preserved if one considers a theory in an extended dimensionality; for example, the pure \ym{} theory is non-renormalizable in six spacetime dimensions. On the other hand, adding a higher derivative term of the form discussed above improves the ultraviolet properties also in this case, and exploiting  this fact one can formulate renormalizable theories also in higher dimensions. Indeed, a theory with the structure $F_{\mu\nu} \partial^2 F^{\mu\nu}/f^2$ is renormalizable in six dimensions keeping the canonical dimension $1$ for the gauge field, with $f$ dimensionless.
%Therefore, a theory in four dimensions with an extra quadratic differential operator in the free sector, is also of interest as a low-energy manifestation of such more fundamental theories satisfying the aforementioned requirements.
%
%
%
%Another aspect to keep into consideration is that, adding a quadratic differential operator, the auxiliary fields, necessary for the supersymmetry algebra to close off-shell, in general become dynamical and appear as ghosts.
%However, the link between supersymmetry and higher derivative theories deserves to be studied more deeply. In simple models (some partial result is discussed in \cite{Ferrara:1977mv}) one can show that some kind of supermultiplet structure arises in considering higher derivative supersymmetric theories. However, one can construct supersymmetric theories with higher derivative terms without a dynamical auxiliary field; this is relevant because they are thought to be non-dynamical, and the Euclidean functional integral over the auxiliary field is divergent, while that on the physical fields is formally convergent. Also, in simple cases, it has been shown that the elimination of the dynamical ghost-like auxiliary field can provide mechanisms for supersymmetry breaking, as discussed in \cite{Fujimori:2016udq,Fujimori:2017kyi}.
%
%
%
%
%
%
%
%
%\vspace{1.5em}
%
%In this work we will study the main properties and the one-loop renormalization of a \ym{} theory in which the kinetic term contains also an extra quadratic differential operator of the type described above, as well as the usual contribution. Then, we will consider its supersymmetric extension. 
%
%In detail, we consider an Euclidean Lagrangian density of the form
%\begin{equation*}
%\Lagr
%	=
%- \frac{1}{2 g^2} \tr  F_{\mu\nu} F_{\mu\nu} %\right)
%	- \frac{1}{ m^2 g^2 }\tr
%		\left[
%			 \left( \covD_\mu F_{\mu\nu} \right)^2
%			+ \gamma   \: F_{\mu\nu}
%				\left[ F_{\mu\lambda},
%				F_{\nu\lambda} \right] 
%			\right],
%\end{equation*}
%in which we allowed for adding the most general contribution of mass dimension six. The $\beta$ function for this theory was computed for the first time  using heat kernels in \cite{Fradkin:1981iu}, that studied this Lagrangian as a toy model for higher derivative gravity. More recently, \cite{Grinstein:2008qq,Schuster} considered the theory as a viable extension of the Standard Model and performed the computation with a diagrammatic approach finding a different result; up to now there is no consensus on the correct value of $\beta$.
%We repeat in this thesis the computation using heat kernels, and we will confirm the latter result. We also consider some models of coupling with matter.
%
%We then study the supersymmetric extension of the Lagrangian given above; formulating a supersymmetric higher derivative theory is actually a nontrivial task because of the complicate structure of the higher derivative term.  As we will motivate, supersymmetry requires restricting to the case $\gamma = 0$.
%The supersymmetric case was somehow discussed in \cite{Buchbinder:1999jn} in terms of superfields, but some contributions have been ignored; \cite{Gama:2011ws} considered the higher-derivative extension of QED. However, an explicit and systematic formulation of the $N=1,2,4$ supersymmetric non-Abelian theory is still missing. Here we will make use of the (unextended) supersymmetric higher-derivative Lagrangian density for the \ym{} field in six dimensions obtained in \cite{Ivanov:2005qf}. In that case, the only  higher derivative term is $\sim  (\covD F)^2 $, weighted with a dimensionless constant,  as we mentioned above; by dimensional reduction  we will then obtain the $N=1$ and $N=2$ supersymmetric higher-derivative \sym{}  Lagrangian in four spacetime dimensions, and we will evaluate the one-loop $\beta$ function for such systems. Remarkably, we will be able to deduce the $\beta$ function also for the case $N=4$.
%
%
%
%
%
%\vspace{1.5em}
%
%
%The work is organised in three main Chapters and two Appendices, as follows.
%
%
%
%In Chapter 1 we introduce all the technical tools needed for the computation. The general setting of the path integral is introduced, with particular attention to the background field technique. The main technical tool that we will employing to perform the computations, namely the heat kernel method for the computation of functional determinants, is introduced.
%
%
%
%In Chapter 2 the higher-derivative extension of \ym{} theory is considered. Some of its properties are described, and in particular the one-loop renormalization is performed and the $\beta$ function of the \ym{} coupling is computed. Simple models of coupling with matter are discussed too, in particular we show that the mass of the scalar field in a class of higher-derivative theory is free of quadratic divergences. 
%
%
%
%
%Chapter 3 is devoted to the supersymmetric extension of the higher-derivative \ym{} theory studied in the previous Chapter. The nontrivial aspects of this generalization are introduced, and in order to overcome them the theory is then formulated in six spacetime dimensions. The relevant Lagrangian with $N=1$ and $2$ supersymmetries in four dimension is then obtained by dimensional reduction. The $\beta$ function of the $N=1$, $2$ and even $4$ supersymmetric extension is then computed.% The supersymmetric case was somehow discussed in \cite{Buchbinder:1999jn}; the renormalisable higher derivative theory was formulated in six spacetime dimensions in \cite{Ivanov:2005qf}, but the explicit formulation of the $N=1$ and $2$ in four spacetime dimensions and the computations of the $\beta$ functions is new.
%
%
%
%In Appendix~A the notation is established.  In Appendix~B some technical computations are reported, given here favouring the readability of the text.
%
%
%
%
%
%
